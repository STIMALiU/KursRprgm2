\batchmode
\makeatletter
\def\input@path{{"/Users/johal95/Documents/Teaching/R programmering/KursRprgm2/Labs/Project/"}}
\makeatother
\documentclass[swedish,english]{article}\usepackage[]{graphicx}\usepackage[]{xcolor}
% maxwidth is the original width if it is less than linewidth
% otherwise use linewidth (to make sure the graphics do not exceed the margin)
\makeatletter
\def\maxwidth{ %
  \ifdim\Gin@nat@width>\linewidth
    \linewidth
  \else
    \Gin@nat@width
  \fi
}
\makeatother

\definecolor{fgcolor}{rgb}{0.345, 0.345, 0.345}
\newcommand{\hlnum}[1]{\textcolor[rgb]{0.686,0.059,0.569}{#1}}%
\newcommand{\hlstr}[1]{\textcolor[rgb]{0.192,0.494,0.8}{#1}}%
\newcommand{\hlcom}[1]{\textcolor[rgb]{0.678,0.584,0.686}{\textit{#1}}}%
\newcommand{\hlopt}[1]{\textcolor[rgb]{0,0,0}{#1}}%
\newcommand{\hlstd}[1]{\textcolor[rgb]{0.345,0.345,0.345}{#1}}%
\newcommand{\hlkwa}[1]{\textcolor[rgb]{0.161,0.373,0.58}{\textbf{#1}}}%
\newcommand{\hlkwb}[1]{\textcolor[rgb]{0.69,0.353,0.396}{#1}}%
\newcommand{\hlkwc}[1]{\textcolor[rgb]{0.333,0.667,0.333}{#1}}%
\newcommand{\hlkwd}[1]{\textcolor[rgb]{0.737,0.353,0.396}{\textbf{#1}}}%
\let\hlipl\hlkwb

\usepackage{framed}
\makeatletter
\newenvironment{kframe}{%
 \def\at@end@of@kframe{}%
 \ifinner\ifhmode%
  \def\at@end@of@kframe{\end{minipage}}%
  \begin{minipage}{\columnwidth}%
 \fi\fi%
 \def\FrameCommand##1{\hskip\@totalleftmargin \hskip-\fboxsep
 \colorbox{shadecolor}{##1}\hskip-\fboxsep
     % There is no \\@totalrightmargin, so:
     \hskip-\linewidth \hskip-\@totalleftmargin \hskip\columnwidth}%
 \MakeFramed {\advance\hsize-\width
   \@totalleftmargin\z@ \linewidth\hsize
   \@setminipage}}%
 {\par\unskip\endMakeFramed%
 \at@end@of@kframe}
\makeatother

\definecolor{shadecolor}{rgb}{.97, .97, .97}
\definecolor{messagecolor}{rgb}{0, 0, 0}
\definecolor{warningcolor}{rgb}{1, 0, 1}
\definecolor{errorcolor}{rgb}{1, 0, 0}
\newenvironment{knitrout}{}{} % an empty environment to be redefined in TeX

\usepackage{alltt}
\usepackage{fontspec}
\usepackage{color}
\usepackage[unicode=true]
 {hyperref}

\makeatletter

%%%%%%%%%%%%%%%%%%%%%%%%%%%%%% LyX specific LaTeX commands.
\providecommand\textquotedblplain{%
  \bgroup\addfontfeatures{Mapping=}\char34\egroup}
%% Because html converters don't know tabularnewline
\providecommand{\tabularnewline}{\\}

\makeatother

\usepackage{polyglossia}
\setdefaultlanguage[variant=american]{english}
\setotherlanguage{swedish}
\IfFileExists{upquote.sty}{\usepackage{upquote}}{}
\begin{document}
\title{Projekt: Programmering i R}

\maketitle
Som en del av kursen i R-programmering ska ni göra en rapport i Rmarkdown.
Det handlar om att läsa in, presentera och bearbeta data.
\begin{itemize}
\item R-markdown ska användas. En mall kan ni hitta \textcolor{blue}{\href{https://raw.githubusercontent.com/STIMALiU/KursRprgm2/main/Labs/Project/TemplateProjekt.Rmd}{här}}.
\item Undvik att använda \textbf{å},\textbf{ä} eller \textbf{ö} i variabelnamn
i er R-kod.
\item Spara er fil i UTF-8 kodning. I Rstudio gör ni: “File” $\rightarrow$
“Save with Encoding” välj UTF-8 och klicka OK.
\item Ha en god kodstil och kommentera er kod. Se datorlaboration 4 för
detaljer. 
\item Rapporterna ska lämnas in som både \textbf{PDF} och \textbf{.Rmd}-fil.
Om ni har problem att skapa PDF så går det bra att lämna in som \textbf{HTML}-fil.
Notera att PDF är att föredra. Det är ok att skapa en HTML som ni
sedan sparar/skriver ut som PDF\footnote{Detta går att göra i de flesta webbläsare, välj skriv ut och sen skriv
till pdf.}. Filerna ska kallas: \texttt{}~\\
\texttt{{[}liu id 1{]}\_{[}liu id 2{]}\_project.pdf}. \\
Exempel på inlämning av projekt del 1 är följande \textbf{två} filer: 
\begin{itemize}
\item \texttt{joswi71\_manma97\_project.Rmd} och
\item \texttt{joswi71\_manma97\_project.pdf}. 
\end{itemize}
\item Samtliga material ska laddas in i R från webben som \textbf{externa
datakällor}. Vill ni använda ett eget material får ni lägga upp det
öppet på github, dropbox, google docs eller dylikt och läsa in det
därifrån i R. Syftet är att rapporten ska vara helt reproducerbar
och kunna återskapas på godtycklig dator.
\item \textbf{Inga output från R console/varningar/meddelanden/felmeddelanden
ska visas i dokumentet.} Antingen skapar ni tabeller (med \texttt{kable()})
eller grafer. T.ex. kan ni ange message=FALSE, warning=FALSE i chunk
options när ni skapar chunks med R-kod.
\item Ni ska använda ggplot2 för alla plottar.
\item \textbf{Rmd}-filen ska kunna köras och reproducera era resultat på
godtycklig dator. D.v.s. den ska innehålla all er kod som behövs för
att ladda ner data, era beräkningar och er rapporttext. 
\item \textbf{Namn}, \textbf{liu-id} och \textbf{gruppnummer} ska framgå
i början av rapporten.
\item \textbf{Tänk på att kommentera er kod och ha god kodstil!}
\end{itemize}

\section{Projektinstruktioner}

Till projektet behöver två olika typer av datamaterial, ett material
med kommunala data och ett material som innehåller en tidsserie. Det
är okej att välja data på lännivå eller regionnivå istället för kommunnivå
om ni vill. Beskrivningen nedan utgår från kommunala data. I projekt
ska ni använda pxweb\footnote{Se \href{https://cran.r-project.org/web/packages/pxweb/vignettes/pxweb.html}{här}
för mer info.}, i filen ska ni ha kod som fungerar och ger reproducerbara rapporter.
Om ni vill dölja varningar kan ni använda

\begin{swedish}%
\begin{knitrout}
\definecolor{shadecolor}{rgb}{0.969, 0.969, 0.969}\color{fgcolor}\begin{kframe}
\begin{alltt}
\hlkwd{suppressWarnings}\hlstd{(\{}
        \hlcom{# min kod h<U+00E4>r  }
\hlstd{\})}
\end{alltt}
\end{kframe}
\end{knitrout}

\end{swedish}%
Tänk på att välja material ni själva tycker är intressant!

\paragraph{Kommunala data}

Ni ska ladda ner kommunala data, där ni i slutändan har minst 4 variabler
på kommunnivå (d.s.v. för alla 290 kommuner) Ett exempel skulle kunna
vara antal arbetslösa i varje kommun. Spara er data i en eller flera
data.frames. Totalt ska dataseten ska ha \textbf{minst} \textbf{4
variabler} utöver kommunnamn. Ni väljer själv vilka variabler som
ska ingå och vilka områden data ska komma ifrån. Tanken är att i ska
göra enklare analyser och grafer som baseras på dessa variabler. Utöver
dessa 4 variabler så ska ni också ladda ner totalt antal invånare
i kommun som en variabel. 

När ni har valt ut era variabler ska ni ha en data.frame där \textbf{varje
rad motsvarar en kommun} och där det finns minst 5 kolumner med variabler.
Kommuner är alltså \textbf{observationer} i era analyser. Kolumnerna
motsvarar era variabler. Notera att många av de variabler som finns
på SCB:s databas är frekvenser, exempel: antal arbetslösa i varje
kommun. Tabell \ref{tab:stuktur data} visar ett exempel med hur data
ska vara strukturerat.

\begin{table}
\begin{tabular}{|c|c|c|c|c|c|}
\hline 
{\scriptsize{}Radnamn} &
{\scriptsize{}Variabel 1} &
{\scriptsize{}Variabel 2} &
{\scriptsize{}Variabel 3} &
{\scriptsize{}Variabel 4} &
{\scriptsize{}Totalt antal invånare}\tabularnewline
\hline 
\hline 
{\scriptsize{}Linköping} &
 &
 &
 &
 &
\tabularnewline
\hline 
{\scriptsize{}Norrköping} &
 &
 &
 &
 &
\tabularnewline
\hline 
{\scriptsize{}Mjölby} &
 &
 &
 &
 &
\tabularnewline
\hline 
{\scriptsize{}Motala} &
 &
 &
 &
 &
\tabularnewline
\hline 
{\scriptsize{}$\vdots$} &
 &
 &
 &
 &
\tabularnewline
\hline 
\end{tabular}\caption{\label{tab:stuktur data}}

\end{table}


\paragraph{Tidsseriedata}

Hitta ett dataset som innehåller en \textbf{tidserie}, det innebär
att det finns en variabel som har observerats över tiden. Kravet är
att data ska innehålla data på \textbf{månadsnivå} och innehålla data
från \textbf{minst 10 år} (120 månder). Här ska ni alltså hitta en
variabel som observerats under minst 120 tidpunkter, men fler går
bra. Data ska alltså innehålla två kolumner, en med variabeln som
vi är intresserade av och en med tidpunkterna. \href{https://raw.githubusercontent.com/STIMALiU/KursRprgm2/main/Labs/Project/F\%C3\%B6rslag\%20p\%C3\%A5\%20tidseriedata\%20som\%20\%C3\%A4r\%20p\%C3\%A5\%20m\%C3\%A5nadsniv\%C3\%A5.pdf}{Här}
finns en lista över några olika tidserier som används tidigare år.

\textbf{Obs!} Tidsperioden ska vara fix, d.v.s ex. jan 2005 - jan
2015. Detta innebär att ni måste ange ett fixt tidsintervall när ni
laddar ner data med \texttt{pxweb}. Om ni laddar ner data en månad
senare ska ni erhålla samma data med samma kod. Om ni laddar ner data
från SCB/pxweb så ska ni \textbf{inte} ange “\texttt{{*}}” på
tiden.

\subsection{Beskrivning av data}

\subsubsection{Dataanalys av kommundata}
\begin{itemize}
\item Skriv en kort inledning där ni beskriver era variabler. Om ni gör
nya transformationer av variablerna i del 2 måste de beskrivas också.
\item Vissa upp data för 5 kommuner och alla era variabler i en tabell.
Ta inte med fler kommuner. Ni väljer själv vilka kommuner ni visar
i tabellen.
\end{itemize}
Följande saker ska ni göra/ta med:
\begin{itemize}
\item Alla variabler som är relaterade till folkmängd på något sätt ska
normaliseras med hjälp av totalt antal invånare i varje kommun. Detta
eftersom det oftast är intressant att kolla på andelar istället för
absoluta antal. T.ex. andelen arbetslösa i en kommun istället för
antalet arbetslösa. I de fall då det är relevant att normalisera en
variabel, då ska ni använda den normaliserade variabeln i plottar
mm. Vissa variabler är inte relaterade till folkmängd, exempel “Antal
höns” eller “Medelålder”, sådana variabler behöver inte normaliseras. 
\begin{itemize}
\item Exempel: antalet arbetslösa/totalt antal invånare = andelen arbetslösa,
gör denna beräkning för varje kommun. För många variabler som har
små andeler så passar det att skapa variabler av typen “antal per
10 000/100 000 invånare”. Då räknar vi ut det som: \\
(antal/totalt antal invånare){*}10 000.
\end{itemize}
\end{itemize}
\begin{enumerate}
\item Producera minst en barplot, om ni bara har kontinuerliga variabler
kan ni använda \texttt{cut()}. Beskriv i text vad ni drar för slutsats.
Notera! Gör inte en plot där varje kommun har en egen stapel, alltså
ingen barplot med 290 staplar.
\item Producera minst ett histogram. Lägg till vertikala linjer för följande
punkter på x-axeln: medianen, första kvartilen och tredje kvartilen.
Beskriv i text vad ni drar för slutsats. Tips: \texttt{geom\_vline()}
och \href{http://www.sthda.com/english/wiki/ggplot2-add-straight-lines-to-a-plot-horizontal-vertical-and-regression-lines}{här}.
\item Producera minst en scatterplot mellan två variabler. Lägg till en
regressionslinje med \texttt{stat\_smooth(method=\textquotedbl lm\textquotedbl ,
se=FALSE)}. Beskriv i text vad ni drar för slutsats\footnote{För tips på tolkning: Se kap 13.2.5 i kursboken, speciellt figur 13-5.
Se även \href{https://chartio.com/learn/charts/what-is-a-scatter-plot/\#when-you-should-use-a-scatter-plot}{här}
för tips på hur scatter plots kan tolkas.}.
\item Beräkna korrelationer mellan de två variabler som ni använde i scatter
plot i steget ovan. Gör ett hypotestest där ni testar om korrelationen
mellan dessa två variabler är noll (=de är linjärt oberoende). Ni
ska alltså använda hypoteserna:
\[
\begin{array}{c}
H_{0}:cor\left(x_{1},x_{2}\right)=0\\
H_{a}:cor\left(x_{1},x_{2}\right)\ne0
\end{array}
\]
Tips: \texttt{cor.test()}. Presentera relevant information i en eller
flera tabeller. Beskriv kort hur ni tolkar resultatet. Ni ska alltså
presentera både den skattade korrelationen och information från testet
i rapporten. Ni får testa korrelationen mellan fler variabler om ni
vill. Presentera även ett konfidensintervall för den skattade korrelationen.
\item Skapa två kategoriska variabler utifrån era ursprungliga variabler.
Om ni redan har kategoriska variabler för era kommuner så kan ni använda
dessa. Dessa (nya) variabler ska användas i olika plottar. Se till
att beskriva era nya variabler.
\item Mer plottar:
\begin{enumerate}
\item Gör en scatterplot där färgen på observationerna ska bero på en av
era kategoriska variabler. Beskriv i text vad ni drar för slutsats.
\item Gör minst ett histogram/boxplot som är grupperat på en av era kategoriska
variabler. Beskriv i text vad ni drar för slutsats.
\item Gör ett grupperad barplot där ni använder båda era kategoriska variabler.
Beskriv i text vad ni drar för slutsats.
\item Gör minst en scatterplot/histogram/barplot/boxplot som är uppdelad
i minst två plottar med \texttt{facet\_grid()} eller \texttt{facet\_wrap()}
baserat på era kategoriska variabler. Beskriv i text vad ni drar för
slutsats.
\end{enumerate}
\end{enumerate}

\subsubsection{Dataanalys av tidseriedata }

Låt \texttt{Y}\footnote{\texttt{Y} är ett arbetsnamn, er variabel måst inte heta så i rapporten}
vara er variabel i tidsseriematerialet. Utför nu följande:
\begin{enumerate}
\item Gör en linjeplot mellan \texttt{Y} och er tidsvariabel. Skalan på
x-axeln ska vara en lämplig tidsskala. Detta gäller alla tidseriegrafer
som baseras på \texttt{Y}. Skriv en kort kommentar.
\item Beräkna medelvärden per månad och spara dessa i \texttt{month\_means}.
Presentera dessa i en tabell och som en lämplig graf. Skriv en kort
kommentar. Ni ska alltså beräkna ett medelvärde för alla värden för
januari och sen upprepa detta för alla månader. \textbf{Tips!} \texttt{aggregate()}
\item Gör grupperade boxplots för \texttt{Y}, där grupperingen baseras på
år. Ni ska alltså ha en boxplot för varje år, och de ska ligga sida
vid sida i samma plot. Skriv en kort kommentar.
\item Subtrahera månadsmedelvärden från \texttt{Y}, så ni tar bort säsongsvariationen
i data. Månadsmedelvärdet för januari ska subtraheras från alla januarivärden
i data, och likadant för de andra månaderna. Spara den nya tidserie
som \texttt{Z}. Addera medelvärdet för \textit{hela }tidserien\texttt{
Y} till \texttt{Z} för att ge \texttt{Z} rätt skala. Se nedan.\\
\begin{knitrout}
\definecolor{shadecolor}{rgb}{0.969, 0.969, 0.969}\color{fgcolor}\begin{kframe}
\begin{alltt}
\hlstd{Z}\hlkwb{<-}\hlstd{Z}\hlopt{+}\hlkwd{mean}\hlstd{(Y)}
\end{alltt}
\end{kframe}
\end{knitrout}
\item Gör en linjeplot mellan \texttt{Z} och tid i \texttt{ggplot2}. Lägg
också till \texttt{Y} i samma graf som jämförelse, men med annan färg.
Ange tydligt med en legend eller i text vilken linje som är \texttt{Y}
och \texttt{Z}.
\item Gör samma plot som i steg 1., men lägg till en regressionslinje med
\texttt{geom\_smooth(method=”lm”)}\footnote{En regressionslinje kan hjälpa oss att se om det finns någon tydlig
minskande eller ökande \emph{linjär} trend. Om linjen är nära att
vara horisontell så tolkar vi det som att det inte finns någon tydlig
linjär trend.}.
\item Verkar det finnas någon trend i data? Dvs ökar/minskar data med tiden,
eller är data konstant över tid. Finns det någon säsongsvariation
i data?\footnote{Exempel på säsongsvariation: December har ofta ett mycket högre värde
än öviga månader, sommarhalvåret har alltid lägre värden.} Dra er slutsats och skriv ned den i dokumentet.
\end{enumerate}
\pagebreak Lämna in rapporten både som en fullt reproducerbar \textbf{Rmd}-fil
och som \textbf{PDF/HTML} i LISAM. Tänk på följande:
\begin{itemize}
\item I denna del ska samtliga grafer vara skapade med \texttt{ggplot2}.
\item Tabeller ska vara “riktiga” tabeller (med ex. \texttt{kable()}),
inte utskrifter i R-kod. All statistik, information från statistiska
test, korrelationer etc ska presenteras i markdowntabeller eller med
inline-kod. Avrunda till ett lämpligt antal decimaler i tabellerna.
\item \textbf{Inga output från R console/varningar/meddelanden/felmeddelanden
ska visas i dokumentet.}
\end{itemize}

\end{document}
