% Options for packages loaded elsewhere
\PassOptionsToPackage{unicode}{hyperref}
\PassOptionsToPackage{hyphens}{url}
\documentclass[
  10pt,
  ignorenonframetext,
  handout]{beamer}
\newif\ifbibliography
\usepackage{pgfpages}
\setbeamertemplate{caption}[numbered]
\setbeamertemplate{caption label separator}{: }
\setbeamercolor{caption name}{fg=normal text.fg}
\beamertemplatenavigationsymbolsempty
% remove section numbering
\setbeamertemplate{part page}{
  \centering
  \begin{beamercolorbox}[sep=16pt,center]{part title}
    \usebeamerfont{part title}\insertpart\par
  \end{beamercolorbox}
}
\setbeamertemplate{section page}{
  \centering
  \begin{beamercolorbox}[sep=12pt,center]{section title}
    \usebeamerfont{section title}\insertsection\par
  \end{beamercolorbox}
}
\setbeamertemplate{subsection page}{
  \centering
  \begin{beamercolorbox}[sep=8pt,center]{subsection title}
    \usebeamerfont{subsection title}\insertsubsection\par
  \end{beamercolorbox}
}
% Prevent slide breaks in the middle of a paragraph
\widowpenalties 1 10000
\raggedbottom
\AtBeginPart{
  \frame{\partpage}
}
\AtBeginSection{
  \ifbibliography
  \else
    \frame{\sectionpage}
  \fi
}
\AtBeginSubsection{
  \frame{\subsectionpage}
}
\usepackage{iftex}
\ifPDFTeX
  \usepackage[T1]{fontenc}
  \usepackage[utf8]{inputenc}
  \usepackage{textcomp} % provide euro and other symbols
\else % if luatex or xetex
  \usepackage{unicode-math} % this also loads fontspec
  \defaultfontfeatures{Scale=MatchLowercase}
  \defaultfontfeatures[\rmfamily]{Ligatures=TeX,Scale=1}
\fi
\usepackage{lmodern}
\ifPDFTeX\else
  % xetex/luatex font selection
\fi
% Use upquote if available, for straight quotes in verbatim environments
\IfFileExists{upquote.sty}{\usepackage{upquote}}{}
\IfFileExists{microtype.sty}{% use microtype if available
  \usepackage[]{microtype}
  \UseMicrotypeSet[protrusion]{basicmath} % disable protrusion for tt fonts
}{}
\makeatletter
\@ifundefined{KOMAClassName}{% if non-KOMA class
  \IfFileExists{parskip.sty}{%
    \usepackage{parskip}
  }{% else
    \setlength{\parindent}{0pt}
    \setlength{\parskip}{6pt plus 2pt minus 1pt}}
}{% if KOMA class
  \KOMAoptions{parskip=half}}
\makeatother
\usepackage{color}
\usepackage{fancyvrb}
\newcommand{\VerbBar}{|}
\newcommand{\VERB}{\Verb[commandchars=\\\{\}]}
\DefineVerbatimEnvironment{Highlighting}{Verbatim}{commandchars=\\\{\}}
% Add ',fontsize=\small' for more characters per line
\usepackage{framed}
\definecolor{shadecolor}{RGB}{248,248,248}
\newenvironment{Shaded}{\begin{snugshade}}{\end{snugshade}}
\newcommand{\AlertTok}[1]{\textcolor[rgb]{0.94,0.16,0.16}{#1}}
\newcommand{\AnnotationTok}[1]{\textcolor[rgb]{0.56,0.35,0.01}{\textbf{\textit{#1}}}}
\newcommand{\AttributeTok}[1]{\textcolor[rgb]{0.13,0.29,0.53}{#1}}
\newcommand{\BaseNTok}[1]{\textcolor[rgb]{0.00,0.00,0.81}{#1}}
\newcommand{\BuiltInTok}[1]{#1}
\newcommand{\CharTok}[1]{\textcolor[rgb]{0.31,0.60,0.02}{#1}}
\newcommand{\CommentTok}[1]{\textcolor[rgb]{0.56,0.35,0.01}{\textit{#1}}}
\newcommand{\CommentVarTok}[1]{\textcolor[rgb]{0.56,0.35,0.01}{\textbf{\textit{#1}}}}
\newcommand{\ConstantTok}[1]{\textcolor[rgb]{0.56,0.35,0.01}{#1}}
\newcommand{\ControlFlowTok}[1]{\textcolor[rgb]{0.13,0.29,0.53}{\textbf{#1}}}
\newcommand{\DataTypeTok}[1]{\textcolor[rgb]{0.13,0.29,0.53}{#1}}
\newcommand{\DecValTok}[1]{\textcolor[rgb]{0.00,0.00,0.81}{#1}}
\newcommand{\DocumentationTok}[1]{\textcolor[rgb]{0.56,0.35,0.01}{\textbf{\textit{#1}}}}
\newcommand{\ErrorTok}[1]{\textcolor[rgb]{0.64,0.00,0.00}{\textbf{#1}}}
\newcommand{\ExtensionTok}[1]{#1}
\newcommand{\FloatTok}[1]{\textcolor[rgb]{0.00,0.00,0.81}{#1}}
\newcommand{\FunctionTok}[1]{\textcolor[rgb]{0.13,0.29,0.53}{\textbf{#1}}}
\newcommand{\ImportTok}[1]{#1}
\newcommand{\InformationTok}[1]{\textcolor[rgb]{0.56,0.35,0.01}{\textbf{\textit{#1}}}}
\newcommand{\KeywordTok}[1]{\textcolor[rgb]{0.13,0.29,0.53}{\textbf{#1}}}
\newcommand{\NormalTok}[1]{#1}
\newcommand{\OperatorTok}[1]{\textcolor[rgb]{0.81,0.36,0.00}{\textbf{#1}}}
\newcommand{\OtherTok}[1]{\textcolor[rgb]{0.56,0.35,0.01}{#1}}
\newcommand{\PreprocessorTok}[1]{\textcolor[rgb]{0.56,0.35,0.01}{\textit{#1}}}
\newcommand{\RegionMarkerTok}[1]{#1}
\newcommand{\SpecialCharTok}[1]{\textcolor[rgb]{0.81,0.36,0.00}{\textbf{#1}}}
\newcommand{\SpecialStringTok}[1]{\textcolor[rgb]{0.31,0.60,0.02}{#1}}
\newcommand{\StringTok}[1]{\textcolor[rgb]{0.31,0.60,0.02}{#1}}
\newcommand{\VariableTok}[1]{\textcolor[rgb]{0.00,0.00,0.00}{#1}}
\newcommand{\VerbatimStringTok}[1]{\textcolor[rgb]{0.31,0.60,0.02}{#1}}
\newcommand{\WarningTok}[1]{\textcolor[rgb]{0.56,0.35,0.01}{\textbf{\textit{#1}}}}
\usepackage{longtable,booktabs,array}
\usepackage{calc} % for calculating minipage widths
\usepackage{caption}
% Make caption package work with longtable
\makeatletter
\def\fnum@table{\tablename~\thetable}
\makeatother
\setlength{\emergencystretch}{3em} % prevent overfull lines
\providecommand{\tightlist}{%
  \setlength{\itemsep}{0pt}\setlength{\parskip}{0pt}}
\usetheme[progressbar=frametitle,block=fill]{metropolis} %numbering=none

%%% USEFUL PACKAGES
%\usepackage{showframe} % For debugging positioning
\usepackage{etex} % If too many packages
% Encoding and language
\usepackage[utf8]{inputenc}
\usepackage{babel}
\usepackage{amsmath, amssymb}
\usepackage{natbib}
%\usepackage{booktabs}
%\usepackage{algorithmic}
\usepackage{algorithm}
\usepackage{caption}
%\usepackage{animate} % Animations
\usepackage{bm} % Bold math
\usepackage{bbm}
%\usepackage{url}
%\usepackage{pifont}
%\usepackage{ulem} % Used for strikeouts \sout
%\usepackage{stackengine}
%\usepackage{enumitem}
%\setlist[description]{leftmargin=\parindent,labelindent=\parindent}
%\usepackage{colortbl} % Used for colored rows in tables


%%% GRAPHICS
\usepackage{graphicx}
\graphicspath{{./figs/}}


%%% COLORS
\setbeamercolor{background canvas}{bg=white}
\def\BlankFrame{
	\bgroup
	%\pdfpageheight 29.7cm
	\setbeamercolor{background canvas}{bg=}
	\begin{frame}[plain]
	\end{frame}
	%\makeatletter
	%\pdfpageheight \beamer@paperheight
	%\makeatother
	\egroup}

\usepackage{xcolor}
\definecolor{DarkGreen}{HTML}{00B200}
\definecolor{LightBlue}{HTML}{0090D9}
\definecolor{gold}{rgb}{.812,.710,.231}
% Text markup
%\setbeamercolor{alerted text}{fg=red}
\newcommand{\blue}[1]{\textcolor{blue}{#1}}
\newcommand{\red}[1]{\textcolor{red}{#1}}
\newcommand{\grey}[1]{\textcolor{gray}{#1}}
\newcommand{\orange}[1]{\textcolor{mLightBrown}{#1}}
\newcommand\myheading[1]{\textbf{#1}}
\newcommand\myemph[1]{\underline{\emph{#1}}}
\newcommand\textexample[1]{\textit{\textbf{#1}}}

%%% SPACING
\newcommand\vws[1][1]{\vspace{#1\baselineskip}} % vertical white space
%\newcommand\strt[1][1.5ex]{\rule[-.05\baselineskip]{0pt}{#1}} % strut
\newcommand\strt[2]{\rule[-#1ex]{0pt}{#2ex}} % strut
\newcommand\Hrule{\vspace{1ex} \hrule \vspace{1ex}} % Horisontal rule with some space after

%%% MISC
\newcommand\articleref[4]{\noindent\begin{minipage}[t]{0.04\textwidth}
		\vspace{0pt} 
		\pgfuseimage{beamericonarticle}
	\end{minipage}%
	\begin{minipage}[t]{0.96\textwidth}
		\vspace{0pt}
		#1. \textbf{#2.} \textit{#3}, #4.
	\end{minipage}}

%%% METROPOLIS THEME SPECIFIC
\makeatletter
\setlength{\metropolis@progressonsectionpage@linewidth}{1pt}
\makeatother
%\setbeamercolor{progress bar}{fg=red,bg=red!50}


%%% TEXTPOS
\usepackage[absolute,overlay]{textpos} % option showboxes is useful in draft mode
\setlength{\TPHorizModule}{\paperwidth}
\setlength{\TPVertModule}{\paperheight}
\textblockorigin{0pt}{10mm} % start everything at top-left, below gray 


%%% TIKZ/PGFPLOTS
\usepackage{tikz}
\usetikzlibrary{arrows,positioning,calc,shapes.geometric}
%\usetikzlibrary{arrows,calc,shapes.geometric,decorations.pathmorphing,backgrounds,positioning,fit,petri,decorations.pathreplacing}
%\usepackage{pgfplots}
%\pgfplotsset{compat = 1.3}


%%% BLOCKS AND BOXES
% Changing colors of blocks
%\setbeamercolor{block title alerted}{bg=UURed,fg=palette primary.fg}
%\setbeamercolor{block body alerted}{bg=UURed!15}
\setbeamercolor{block title alerted}{bg=mLightBrown,fg=palette primary.fg}
\setbeamercolor{block body alerted}{bg=mLightBrown!15}
%\setbeamercolor{block title example}{bg=UUGreen,fg=palette primary.fg}
%\setbeamercolor{block body example}{bg=UUGreen!10}
% \mybox is a rectangular box
\usepackage{boxedminipage}
\setlength\fboxrule{2pt}
\setlength\fboxsep{2\fboxsep}
\newcommand\mybox[3][\textwidth]{
  {\color{#2}
    \begin{boxedminipage}{#1}
      {\color{palette primary.bg} #3}
    \end{boxedminipage}}%
}   
\usepackage{tcolorbox}
\tcbset{arc=1mm,grow to left by=3mm,grow to right by=3mm,left=2mm}
%\newenvironment{redbox}{%
%	\begin{tcolorbox}[colback=UURed!15,colframe=UURed]}{%
%	\end{tcolorbox}}
%\newenvironment{greenbox}{%
%	\begin{tcolorbox}[colback=UUGreen!15,colframe=UUGreen]}{%
%	\end{tcolorbox}}
\newenvironment{redbox}{%
	\begin{tcolorbox}[colback=red!15,colframe=red]}{%
	\end{tcolorbox}}
\newenvironment{greenbox}{%
	\begin{tcolorbox}[colback=DarkGreen!15,colframe=DarkGreen]}{%
	\end{tcolorbox}}
\newenvironment{graybox}{%
	\begin{tcolorbox}[colback=mDarkTeal!5,colframe=mDarkTeal]}{%
	\end{tcolorbox}}
\newenvironment{orangebox}{%
\begin{tcolorbox}[colback=mLightBrown!15,colframe=mLightBrown]}{%
	\end{tcolorbox}}
\newenvironment{bwbox}{%
	\begin{tcolorbox}[colback=white,colframe=black]}{%
\end{tcolorbox}}
\newenvironment{bluebox}{%
	\begin{tcolorbox}[colback=LightBlue!15,colframe=LightBlue]}{%
\end{tcolorbox}}


%%%%%%%%% NEW MACROS

\newcommand\imp[1]{\alert{\textbf{#1}}}
\newcommand\bfit[1]{\textbf{\textit{#1}}}
\newcommand\good{\color{DarkGreen}{$\blacktriangle$}} % used in lists
\newcommand\bad{\color{red}{$\blacktriangledown$}} % used in lists

\usepackage{bookmark}
\IfFileExists{xurl.sty}{\usepackage{xurl}}{} % add URL line breaks if available
\urlstyle{same}
\hypersetup{
  pdftitle={R-programmering VT25},
  pdfauthor={Johan Alenlöv},
  hidelinks,
  pdfcreator={LaTeX via pandoc}}

\title{R-programmering VT25}
\subtitle{Föreläsning 8}
\author{Johan Alenlöv}
\date{}
\institute{Linköpings Universitet}

\begin{document}
\frame{\titlepage}

\section{Föreläsning 8}\label{fuxf6reluxe4sning-8}

\begin{frame}{Innehåll föreläsning 8}
\phantomsection\label{innehuxe5ll-fuxf6reluxe4sning-8}
\begin{itemize}
\tightlist
\item
  Texthantering

  \begin{itemize}
  \tightlist
  \item
    \texttt{stringr}
  \item
    regex
  \end{itemize}
\item
  Modern databearbetning

  \begin{itemize}
  \tightlist
  \item
    \texttt{tidyr}
  \item
    \texttt{dplyr}
  \end{itemize}
\end{itemize}
\end{frame}

\section{Texthantering i R}\label{texthantering-i-r}

\begin{frame}{Arbeta med strängar}
\phantomsection\label{arbeta-med-struxe4ngar}
\begin{itemize}
\tightlist
\item
  En sträng är en samling bokstäver
\item
  R har ett antal inbyggda funktioner för att hantera text

  \begin{itemize}
  \tightlist
  \item
    \texttt{paste()}
  \item
    \texttt{substr()}
  \item
    \texttt{nchar()}
  \end{itemize}
\item
  Använder hellre paketet \texttt{stringr}

  \begin{itemize}
  \tightlist
  \item
    Enklare
  \item
    Enhetligt
  \end{itemize}
\end{itemize}
\end{frame}

\begin{frame}{Arbeta med strängar - II}
\phantomsection\label{arbeta-med-struxe4ngar---ii}
\begin{itemize}
\tightlist
\item
  \texttt{readLines(con = , encoding = )} används för att läsa in en
  text.

  \begin{itemize}
  \tightlist
  \item
    \texttt{con} är ``connection'' t.ex. vart en fil ligger
  \item
    \texttt{encoding} är vilken text-kodning som används

    \begin{itemize}
    \tightlist
    \item
      \texttt{"latin1"} och \texttt{"utf8"} är vanligast.
    \end{itemize}
  \end{itemize}
\end{itemize}
\end{frame}

\section{Paketet stringr}\label{paketet-stringr}

\begin{frame}{Paketet stringr}
\phantomsection\label{paketet-stringr-1}
\begin{itemize}
\tightlist
\item
  Ett paket med funktioner för strängar

  \begin{itemize}
  \tightlist
  \item
    Optimerade och effektiva funktioner
  \item
    Funktioner börjar med \texttt{str\_}
  \end{itemize}
\item
  Två delar:

  \begin{itemize}
  \tightlist
  \item
    Standard funktioner
  \item
    Mönstermatchande funktioner
  \end{itemize}
\end{itemize}
\end{frame}

\begin{frame}[fragile]{Grundläggande strängfunktioner}
\phantomsection\label{grundluxe4ggande-struxe4ngfunktioner}
\begin{longtable}[]{@{}
  >{\raggedright\arraybackslash}p{(\linewidth - 4\tabcolsep) * \real{0.2500}}
  >{\raggedright\arraybackslash}p{(\linewidth - 4\tabcolsep) * \real{0.4167}}
  >{\raggedright\arraybackslash}p{(\linewidth - 4\tabcolsep) * \real{0.3333}}@{}}
\toprule\noalign{}
\begin{minipage}[b]{\linewidth}\raggedright
stringr
\end{minipage} & \begin{minipage}[b]{\linewidth}\raggedright
base
\end{minipage} & \begin{minipage}[b]{\linewidth}\raggedright
Användning
\end{minipage} \\
\midrule\noalign{}
\endhead
\texttt{str\_sub()} & \texttt{substr()} & substring, välja ut en del av
en sträng (regex) \\
\texttt{str\_c()} & \texttt{paste()}, \texttt{paste0()} & slår ihop
strängelement \\
\texttt{str\_split()} & \texttt{strsplit()} & dela upp en sträng i flera
element (regex) \\
\texttt{str\_length()} & \texttt{nchar()} & beräknar antalet tecken \\
\texttt{str\_trim()} & - & tar bort mellanslag (före/efter
textelement) \\
\texttt{str\_pad()} & - & lägger till mellanslag (före/efter
textelement) \\
\bottomrule\noalign{}
\end{longtable}
\end{frame}

\begin{frame}{Demo}
\phantomsection\label{demo}
\begin{block}{Demo: stringr}
\phantomsection\label{demo-stringr}
\end{block}
\end{frame}

\section{Mönstermatchning}\label{muxf6nstermatchning}

\begin{frame}{Regular expression (regex)}
\phantomsection\label{regular-expression-regex}
Från Wikipedia:

\begin{quote}
A regular expression (shortened as regex) is a sequence of characters
that specifies a search pattern in text. Usually such patterns are used
by string-searching algorithms for ``find'' or ``find and replace''
operations on strings, or for input validation. It is a technique
developed in theoretical computer science and formal language theory.
\end{quote}

\begin{itemize}
\tightlist
\item
  Notation för att beskriva strängar

  \begin{itemize}
  \tightlist
  \item
    Hitta en specifik del som uppfyller ett villkor
  \item
    Textmanipulation
  \end{itemize}
\item
  Byggs upp av

  \begin{itemize}
  \tightlist
  \item
    literals: Vanliga bokstäver och siffror
  \item
    metacharacters: Speciella regler
  \end{itemize}
\end{itemize}
\end{frame}

\begin{frame}[fragile]{Regular expression: Metacharachters}
\phantomsection\label{regular-expression-metacharachters}
\begin{longtable}[]{@{}ll@{}}
\toprule\noalign{}
Tecken & Betydelse \\
\midrule\noalign{}
\endhead
\texttt{.} & samtliga tecken (exkl. det ``tomma'' tecknet ``\,'') \\
\texttt{\^{}} & det ``tomma'' tecknet i början av en text \\
\texttt{\$} & det ``tomma'' tecknet i slutet text \\
\texttt{*} & föregående tecken 0 eller fler gånger \\
\texttt{+} & föregående tecken 1 eller fler gånger \\
\texttt{?} & föregående tecken är valfritt \\
\texttt{\{n,m\}} & föregående tecken \texttt{n} eller max \texttt{m}
gånger \\
\texttt{{[}...{]}} & teckenlista (character list) \\
\textbar{} & ELLER \\
\texttt{(...)} & Gruppering \\
\texttt{\textbackslash{}} & Används för att ``undvika''
metatecken/specialtecken. \\
\bottomrule\noalign{}
\end{longtable}

\textbf{Obs!} I R krävs: \texttt{\textbackslash{}\textbackslash{}}
\end{frame}

\begin{frame}[fragile]{Regular expression: teckenklass}
\phantomsection\label{regular-expression-teckenklass}
\begin{itemize}
\tightlist
\item
  Med \texttt{{[}\ {]}} skapas en lista över tänkbara tecken.
\item
  Används för att identifiera en mängd av tecken
\item
  Inom \texttt{{[}\ {]}} har bara följande meta-tecken en särskild
  betydelse
\end{itemize}

\begin{longtable}[]{@{}lll@{}}
\toprule\noalign{}
Tecken & Betydelse & Exempel \\
\midrule\noalign{}
\endhead
\texttt{-} & tecken & A-Z a-z 0-9 \\
\texttt{\^{}} & ICKE & \texttt{\textbackslash{}\^{}0-9} \\
\texttt{\textbackslash{}} & specialtecken &
\texttt{\textbackslash{}t\textbackslash{}n} \\
\bottomrule\noalign{}
\end{longtable}

\textbf{Obs!} I R krävs: \texttt{\textbackslash{}\textbackslash{}}
\end{frame}

\begin{frame}[fragile]{Regular expression: teckenklass - II}
\phantomsection\label{regular-expression-teckenklass---ii}
Vanliga fördefinerade klasser, (se \texttt{?regexp})

\begin{itemize}
\tightlist
\item
  \texttt{{[}:digit:{]}} Nummer
\item
  \texttt{{[}:lower:{]}} gemener
\item
  \texttt{{[}:upper:{]}} VERSALER
\item
  \texttt{{[}:punct:{]}} tecken, ej bokstäver och siffror
\item
  \texttt{{[}:space:{]}} mellanslag, tab, radbrytning, m.m.
\item
  i R behöver vi ange att det är en teckenklass
  \texttt{{[}{[}:space:{]}{]}}
\end{itemize}
\end{frame}

\begin{frame}{Lära sig regular expression}
\phantomsection\label{luxe4ra-sig-regular-expression}
\begin{itemize}
\tightlist
\item
  Testa dina expressions

  \begin{itemize}
  \tightlist
  \item
    \textbf{\url{https://regexr.com}}
  \item
    \textbf{\url{https://www.regexpal.com}}
  \end{itemize}
\item
  Roliga lekar med regex

  \begin{itemize}
  \tightlist
  \item
    Regex Golf \textbf{\url{https://alf.nu/RegexGolf}}
  \item
    Regex crossword \textbf{\url{https://regexcrossword.com}}
  \end{itemize}
\end{itemize}
\end{frame}

\begin{frame}[fragile]{Mönstermatchning i R}
\phantomsection\label{muxf6nstermatchning-i-r}
\begin{itemize}
\tightlist
\item
  \texttt{pattern} är ett regular expression i R
\end{itemize}

\begin{longtable}[]{@{}
  >{\raggedright\arraybackslash}p{(\linewidth - 4\tabcolsep) * \real{0.2500}}
  >{\raggedright\arraybackslash}p{(\linewidth - 4\tabcolsep) * \real{0.4167}}
  >{\raggedright\arraybackslash}p{(\linewidth - 4\tabcolsep) * \real{0.3333}}@{}}
\toprule\noalign{}
\begin{minipage}[b]{\linewidth}\raggedright
stringr
\end{minipage} & \begin{minipage}[b]{\linewidth}\raggedright
base
\end{minipage} & \begin{minipage}[b]{\linewidth}\raggedright
Användning
\end{minipage} \\
\midrule\noalign{}
\endhead
\texttt{str\_detect()} & \texttt{grepl()} & identifierar pattern,
returnerar en logisk vektor \\
\texttt{str\_locate()} & \texttt{gregexpr()} & identifierar pattern,
returnerar positionen i texten \\
\texttt{str\_replace()} & \texttt{gsub()} & identifierar pattern, och
ersätter detta med ny text \\
\texttt{str\_extract\_all} & - & Plocka ut alla strängar som uppfyller
\texttt{pattern} \\
\bottomrule\noalign{}
\end{longtable}
\end{frame}

\begin{frame}{Demo}
\phantomsection\label{demo-1}
\begin{block}{Demo: Mönstermatchning}
\phantomsection\label{demo-muxf6nstermatchning}
\end{block}
\end{frame}

\section{Modern databearbetning}\label{modern-databearbetning}

\begin{frame}{Varför databeartbetning}
\phantomsection\label{varfuxf6r-databeartbetning}
\begin{itemize}
\tightlist
\item
  Datamängder blir bara större och större
\item
  Smart hantering minskar arbetsbördan
\item
  Smart hantering för bearbetningen snabb
\item
  Analysfunktioner kräver särskilt format
\item
  Skriv kod för människor
\end{itemize}
\end{frame}

\begin{frame}[fragile]{piping}
\phantomsection\label{piping}
Piping görs med \texttt{\%\textgreater{}\%}

\begin{Shaded}
\begin{Highlighting}[]
\NormalTok{z }\OtherTok{\textless{}{-}}\NormalTok{ a }\SpecialCharTok{\%\textgreater{}\%}
  \FunctionTok{fun1}\NormalTok{(b) }\SpecialCharTok{\%\textgreater{}\%}
  \FunctionTok{fun3}\NormalTok{()}
\end{Highlighting}
\end{Shaded}

är samma som

\begin{Shaded}
\begin{Highlighting}[]
\NormalTok{x }\OtherTok{\textless{}{-}} \FunctionTok{fun1}\NormalTok{(a,b)}
\NormalTok{z }\OtherTok{\textless{}{-}} \FunctionTok{fun3}\NormalTok{(x)}
\end{Highlighting}
\end{Shaded}
\end{frame}

\begin{frame}[fragile]{tidyr: Tidy data}
\phantomsection\label{tidyr-tidy-data}
\begin{itemize}
\tightlist
\item
  Data är ofta ``messy''
\item
  Tidy data:

  \begin{itemize}
  \tightlist
  \item
    Varje kolumn en variabel
  \item
    Varje rad en observation
  \end{itemize}
\item
  \texttt{tidyr} är ett paket för att konvertera ``messy'' till ``tidy''
\item
  Effektivt både minnesmässigt och beräkningsmässigt
\item
  kommer bespara er mycket tid
\end{itemize}
\end{frame}

\begin{frame}[fragile]{dplyr}
\phantomsection\label{dplyr}
\begin{itemize}
\tightlist
\item
  Paket i R för att hantera \textbf{stora} datamängder.
\item
  En liten uppsättning funktioner (verb) för datahantering.
\item
  Väldigt optimerad kod för snabb och minneseffektiv hantering.
\item
  Går att koppla till databaser och Spark.
\item
  Lägger på klassen \texttt{tbl\_df} till \texttt{data.frame}
\end{itemize}
\end{frame}

\begin{frame}[fragile]{dplyr verb}
\phantomsection\label{dplyr-verb}
\begin{longtable}[]{@{}ll@{}}
\toprule\noalign{}
verb & beskrivning \\
\midrule\noalign{}
\endhead
\texttt{select()} & välj kolumn \\
\texttt{filter()} & filtrera rader \\
\texttt{arrange()} & arrangera rader \\
\texttt{mutate()} & skapa nya kolumner \\
\texttt{summarise()} & aggregera rader över grupp \\
\texttt{group\_by()} & gruppera för ``split-apply-combine''/aggregera \\
\texttt{join} & kombindera olika dataset \\
\texttt{bind\_rows} & kombindera dataset ``på höjden'' \\
\texttt{bind\_cols} & kombindera dataset ``på bredden'' \\
\bottomrule\noalign{}
\end{longtable}
\end{frame}

\begin{frame}[fragile]{dplyr joins}
\phantomsection\label{dplyr-joins}
\begin{itemize}
\tightlist
\item
  Slå ihop data är oftast centralt
\item
  Inom databser talar man om ``joins''
\end{itemize}

\begin{longtable}[]{@{}
  >{\raggedright\arraybackslash}p{(\linewidth - 2\tabcolsep) * \real{0.3750}}
  >{\raggedright\arraybackslash}p{(\linewidth - 2\tabcolsep) * \real{0.6250}}@{}}
\toprule\noalign{}
\begin{minipage}[b]{\linewidth}\raggedright
funktion
\end{minipage} & \begin{minipage}[b]{\linewidth}\raggedright
beskrivning
\end{minipage} \\
\midrule\noalign{}
\endhead
\texttt{left\_join()} & slå ihop efter variabel, behåll obs. i vänstra
data.frame \\
\texttt{right\_join()} & slå ihop efter variabel, behåll obs. i högra
data.frame \\
\texttt{full\_join()} & slå ihop efter variabel, behåll alla obs. \\
\texttt{anti\_join()} & slå ihop efter variabel, behåll obs. som inte
finns i båda \\
\bottomrule\noalign{}
\end{longtable}
\end{frame}

\begin{frame}{Demo}
\phantomsection\label{demo-2}
\begin{block}{Demo: Databehandling}
\phantomsection\label{demo-databehandling}
\end{block}
\end{frame}

\end{document}
