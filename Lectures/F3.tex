% Options for packages loaded elsewhere
\PassOptionsToPackage{unicode}{hyperref}
\PassOptionsToPackage{hyphens}{url}
%
\documentclass[
  11pt,
  ignorenonframetext,
]{beamer}
\usepackage{pgfpages}
\setbeamertemplate{caption}[numbered]
\setbeamertemplate{caption label separator}{: }
\setbeamercolor{caption name}{fg=normal text.fg}
\beamertemplatenavigationsymbolsempty
% Prevent slide breaks in the middle of a paragraph
\widowpenalties 1 10000
\raggedbottom
\setbeamertemplate{part page}{
  \centering
  \begin{beamercolorbox}[sep=16pt,center]{part title}
    \usebeamerfont{part title}\insertpart\par
  \end{beamercolorbox}
}
\setbeamertemplate{section page}{
  \centering
  \begin{beamercolorbox}[sep=12pt,center]{part title}
    \usebeamerfont{section title}\insertsection\par
  \end{beamercolorbox}
}
\setbeamertemplate{subsection page}{
  \centering
  \begin{beamercolorbox}[sep=8pt,center]{part title}
    \usebeamerfont{subsection title}\insertsubsection\par
  \end{beamercolorbox}
}
\AtBeginPart{
  \frame{\partpage}
}
\AtBeginSection{
  \ifbibliography
  \else
    \frame{\sectionpage}
  \fi
}
\AtBeginSubsection{
  \frame{\subsectionpage}
}
\usepackage{amsmath,amssymb}
\usepackage{lmodern}
\usepackage{iftex}
\ifPDFTeX
  \usepackage[T1]{fontenc}
  \usepackage[utf8]{inputenc}
  \usepackage{textcomp} % provide euro and other symbols
\else % if luatex or xetex
  \usepackage{unicode-math}
  \defaultfontfeatures{Scale=MatchLowercase}
  \defaultfontfeatures[\rmfamily]{Ligatures=TeX,Scale=1}
\fi
% Use upquote if available, for straight quotes in verbatim environments
\IfFileExists{upquote.sty}{\usepackage{upquote}}{}
\IfFileExists{microtype.sty}{% use microtype if available
  \usepackage[]{microtype}
  \UseMicrotypeSet[protrusion]{basicmath} % disable protrusion for tt fonts
}{}
\makeatletter
\@ifundefined{KOMAClassName}{% if non-KOMA class
  \IfFileExists{parskip.sty}{%
    \usepackage{parskip}
  }{% else
    \setlength{\parindent}{0pt}
    \setlength{\parskip}{6pt plus 2pt minus 1pt}}
}{% if KOMA class
  \KOMAoptions{parskip=half}}
\makeatother
\usepackage{xcolor}
\IfFileExists{xurl.sty}{\usepackage{xurl}}{} % add URL line breaks if available
\IfFileExists{bookmark.sty}{\usepackage{bookmark}}{\usepackage{hyperref}}
\hypersetup{
  pdftitle={R-programmering VT2023},
  pdfauthor={Josef Wilzén},
  hidelinks,
  pdfcreator={LaTeX via pandoc}}
\urlstyle{same} % disable monospaced font for URLs
\newif\ifbibliography
\usepackage{color}
\usepackage{fancyvrb}
\newcommand{\VerbBar}{|}
\newcommand{\VERB}{\Verb[commandchars=\\\{\}]}
\DefineVerbatimEnvironment{Highlighting}{Verbatim}{commandchars=\\\{\}}
% Add ',fontsize=\small' for more characters per line
\usepackage{framed}
\definecolor{shadecolor}{RGB}{248,248,248}
\newenvironment{Shaded}{\begin{snugshade}}{\end{snugshade}}
\newcommand{\AlertTok}[1]{\textcolor[rgb]{0.94,0.16,0.16}{#1}}
\newcommand{\AnnotationTok}[1]{\textcolor[rgb]{0.56,0.35,0.01}{\textbf{\textit{#1}}}}
\newcommand{\AttributeTok}[1]{\textcolor[rgb]{0.77,0.63,0.00}{#1}}
\newcommand{\BaseNTok}[1]{\textcolor[rgb]{0.00,0.00,0.81}{#1}}
\newcommand{\BuiltInTok}[1]{#1}
\newcommand{\CharTok}[1]{\textcolor[rgb]{0.31,0.60,0.02}{#1}}
\newcommand{\CommentTok}[1]{\textcolor[rgb]{0.56,0.35,0.01}{\textit{#1}}}
\newcommand{\CommentVarTok}[1]{\textcolor[rgb]{0.56,0.35,0.01}{\textbf{\textit{#1}}}}
\newcommand{\ConstantTok}[1]{\textcolor[rgb]{0.00,0.00,0.00}{#1}}
\newcommand{\ControlFlowTok}[1]{\textcolor[rgb]{0.13,0.29,0.53}{\textbf{#1}}}
\newcommand{\DataTypeTok}[1]{\textcolor[rgb]{0.13,0.29,0.53}{#1}}
\newcommand{\DecValTok}[1]{\textcolor[rgb]{0.00,0.00,0.81}{#1}}
\newcommand{\DocumentationTok}[1]{\textcolor[rgb]{0.56,0.35,0.01}{\textbf{\textit{#1}}}}
\newcommand{\ErrorTok}[1]{\textcolor[rgb]{0.64,0.00,0.00}{\textbf{#1}}}
\newcommand{\ExtensionTok}[1]{#1}
\newcommand{\FloatTok}[1]{\textcolor[rgb]{0.00,0.00,0.81}{#1}}
\newcommand{\FunctionTok}[1]{\textcolor[rgb]{0.00,0.00,0.00}{#1}}
\newcommand{\ImportTok}[1]{#1}
\newcommand{\InformationTok}[1]{\textcolor[rgb]{0.56,0.35,0.01}{\textbf{\textit{#1}}}}
\newcommand{\KeywordTok}[1]{\textcolor[rgb]{0.13,0.29,0.53}{\textbf{#1}}}
\newcommand{\NormalTok}[1]{#1}
\newcommand{\OperatorTok}[1]{\textcolor[rgb]{0.81,0.36,0.00}{\textbf{#1}}}
\newcommand{\OtherTok}[1]{\textcolor[rgb]{0.56,0.35,0.01}{#1}}
\newcommand{\PreprocessorTok}[1]{\textcolor[rgb]{0.56,0.35,0.01}{\textit{#1}}}
\newcommand{\RegionMarkerTok}[1]{#1}
\newcommand{\SpecialCharTok}[1]{\textcolor[rgb]{0.00,0.00,0.00}{#1}}
\newcommand{\SpecialStringTok}[1]{\textcolor[rgb]{0.31,0.60,0.02}{#1}}
\newcommand{\StringTok}[1]{\textcolor[rgb]{0.31,0.60,0.02}{#1}}
\newcommand{\VariableTok}[1]{\textcolor[rgb]{0.00,0.00,0.00}{#1}}
\newcommand{\VerbatimStringTok}[1]{\textcolor[rgb]{0.31,0.60,0.02}{#1}}
\newcommand{\WarningTok}[1]{\textcolor[rgb]{0.56,0.35,0.01}{\textbf{\textit{#1}}}}
\usepackage{longtable,booktabs,array}
\usepackage{calc} % for calculating minipage widths
\usepackage{caption}
% Make caption package work with longtable
\makeatletter
\def\fnum@table{\tablename~\thetable}
\makeatother
\setlength{\emergencystretch}{3em} % prevent overfull lines
\providecommand{\tightlist}{%
  \setlength{\itemsep}{0pt}\setlength{\parskip}{0pt}}
\setcounter{secnumdepth}{-\maxdimen} % remove section numbering
\usetheme[progressbar=frametitle,block=fill]{metropolis} %numbering=none

%%% USEFUL PACKAGES
%\usepackage{showframe} % For debugging positioning
\usepackage{etex} % If too many packages
% Encoding and language
\usepackage[utf8]{inputenc}
\usepackage{babel}
\usepackage{amsmath, amssymb}
\usepackage{natbib}
%\usepackage{booktabs}
%\usepackage{algorithmic}
\usepackage{algorithm}
\usepackage{caption}
%\usepackage{animate} % Animations
\usepackage{bm} % Bold math
\usepackage{bbm}
%\usepackage{url}
%\usepackage{pifont}
%\usepackage{ulem} % Used for strikeouts \sout
%\usepackage{stackengine}
%\usepackage{enumitem}
%\setlist[description]{leftmargin=\parindent,labelindent=\parindent}
%\usepackage{colortbl} % Used for colored rows in tables


%%% GRAPHICS
\usepackage{graphicx}
\graphicspath{{./figs/}}


%%% COLORS
\setbeamercolor{background canvas}{bg=white}
\def\BlankFrame{
	\bgroup
	%\pdfpageheight 29.7cm
	\setbeamercolor{background canvas}{bg=}
	\begin{frame}[plain]
	\end{frame}
	%\makeatletter
	%\pdfpageheight \beamer@paperheight
	%\makeatother
	\egroup}

\usepackage{xcolor}
\definecolor{DarkGreen}{HTML}{00B200}
\definecolor{LightBlue}{HTML}{0090D9}
\definecolor{gold}{rgb}{.812,.710,.231}
% Text markup
%\setbeamercolor{alerted text}{fg=red}
\newcommand{\blue}[1]{\textcolor{blue}{#1}}
\newcommand{\red}[1]{\textcolor{red}{#1}}
\newcommand{\grey}[1]{\textcolor{gray}{#1}}
\newcommand{\orange}[1]{\textcolor{mLightBrown}{#1}}
\newcommand\myheading[1]{\textbf{#1}}
\newcommand\myemph[1]{\underline{\emph{#1}}}
\newcommand\textexample[1]{\textit{\textbf{#1}}}

%%% SPACING
\newcommand\vws[1][1]{\vspace{#1\baselineskip}} % vertical white space
%\newcommand\strt[1][1.5ex]{\rule[-.05\baselineskip]{0pt}{#1}} % strut
\newcommand\strt[2]{\rule[-#1ex]{0pt}{#2ex}} % strut
\newcommand\Hrule{\vspace{1ex} \hrule \vspace{1ex}} % Horisontal rule with some space after

%%% MISC
\newcommand\articleref[4]{\noindent\begin{minipage}[t]{0.04\textwidth}
		\vspace{0pt} 
		\pgfuseimage{beamericonarticle}
	\end{minipage}%
	\begin{minipage}[t]{0.96\textwidth}
		\vspace{0pt}
		#1. \textbf{#2.} \textit{#3}, #4.
	\end{minipage}}

%%% METROPOLIS THEME SPECIFIC
\makeatletter
\setlength{\metropolis@progressonsectionpage@linewidth}{1pt}
\makeatother
%\setbeamercolor{progress bar}{fg=red,bg=red!50}


%%% TEXTPOS
\usepackage[absolute,overlay]{textpos} % option showboxes is useful in draft mode
\setlength{\TPHorizModule}{\paperwidth}
\setlength{\TPVertModule}{\paperheight}
\textblockorigin{0pt}{10mm} % start everything at top-left, below gray 


%%% TIKZ/PGFPLOTS
\usepackage{tikz}
\usetikzlibrary{arrows,positioning,calc,shapes.geometric}
%\usetikzlibrary{arrows,calc,shapes.geometric,decorations.pathmorphing,backgrounds,positioning,fit,petri,decorations.pathreplacing}
%\usepackage{pgfplots}
%\pgfplotsset{compat = 1.3}


%%% BLOCKS AND BOXES
% Changing colors of blocks
%\setbeamercolor{block title alerted}{bg=UURed,fg=palette primary.fg}
%\setbeamercolor{block body alerted}{bg=UURed!15}
\setbeamercolor{block title alerted}{bg=mLightBrown,fg=palette primary.fg}
\setbeamercolor{block body alerted}{bg=mLightBrown!15}
%\setbeamercolor{block title example}{bg=UUGreen,fg=palette primary.fg}
%\setbeamercolor{block body example}{bg=UUGreen!10}
% \mybox is a rectangular box
\usepackage{boxedminipage}
\setlength\fboxrule{2pt}
\setlength\fboxsep{2\fboxsep}
\newcommand\mybox[3][\textwidth]{
  {\color{#2}
    \begin{boxedminipage}{#1}
      {\color{palette primary.bg} #3}
    \end{boxedminipage}}%
}   
\usepackage{tcolorbox}
\tcbset{arc=1mm,grow to left by=3mm,grow to right by=3mm,left=2mm}
%\newenvironment{redbox}{%
%	\begin{tcolorbox}[colback=UURed!15,colframe=UURed]}{%
%	\end{tcolorbox}}
%\newenvironment{greenbox}{%
%	\begin{tcolorbox}[colback=UUGreen!15,colframe=UUGreen]}{%
%	\end{tcolorbox}}
\newenvironment{redbox}{%
	\begin{tcolorbox}[colback=red!15,colframe=red]}{%
	\end{tcolorbox}}
\newenvironment{greenbox}{%
	\begin{tcolorbox}[colback=DarkGreen!15,colframe=DarkGreen]}{%
	\end{tcolorbox}}
\newenvironment{graybox}{%
	\begin{tcolorbox}[colback=mDarkTeal!5,colframe=mDarkTeal]}{%
	\end{tcolorbox}}
\newenvironment{orangebox}{%
\begin{tcolorbox}[colback=mLightBrown!15,colframe=mLightBrown]}{%
	\end{tcolorbox}}
\newenvironment{bwbox}{%
	\begin{tcolorbox}[colback=white,colframe=black]}{%
\end{tcolorbox}}
\newenvironment{bluebox}{%
	\begin{tcolorbox}[colback=LightBlue!15,colframe=LightBlue]}{%
\end{tcolorbox}}


%%%%%%%%% NEW MACROS

\newcommand\imp[1]{\alert{\textbf{#1}}}
\newcommand\bfit[1]{\textbf{\textit{#1}}}
\newcommand\good{\color{DarkGreen}{$\blacktriangle$}} % used in lists
\newcommand\bad{\color{red}{$\blacktriangledown$}} % used in lists

\ifLuaTeX
  \usepackage{selnolig}  % disable illegal ligatures
\fi

\title{R-programmering VT2023}
\subtitle{Föreläsning 3}
\author{Josef Wilzén}
\date{2023-02-06}
\institute{Linköpings Universitet}

\begin{document}
\frame{\titlepage}

%%%%%%%%% slide %%%%%%%%%

\begin{frame}{Föreläsning 3:}
\protect\hypertarget{fuxf6reluxe4sning-3}{}
\textbf{Seminarium} på torsdag $\rightarrow$ behöver era frågor/feedback
\textbf{Studieteknik}  $\rightarrow$ funkar det bra?

\begin{itemize}
\tightlist
\item
  Sammanfattning Föreläsning 2
\item
  Villkorssatser
\item
  Loopar
\item
  Funktionsmeddelanden
\item
  Debugging
\end{itemize}
\end{frame}

\hypertarget{sammanfattning-fuxf6reluxe4sning-2}{%
\section{Sammanfattning föreläsning
2}\label{sammanfattning-fuxf6reluxe4sning-2}}

%%%%%%%%% slide %%%%%%%%%

\begin{frame}{Matriser}
\protect\hypertarget{matriser}{}
\begin{itemize}
\tightlist
\item
  Två dimensionell vektor, alla element av samma typ
\item
  Välj index med \texttt{[ "rad" , "kolumn" ]}

  \begin{itemize}
  \tightlist
  \item
    Saknas \texttt{rad} eller \texttt{kolumn} väljs hela.
  \item
    Om man väljer ut bara är en rad eller kolumn görs resultatet om till
    en vektor.

    \begin{itemize}
    \tightlist
    \item
      För att behålla matrisstrukturen använd \texttt{drop = FALSE}
    \end{itemize}
  \end{itemize}
\item
  \texttt{length} ger antal element
\item
  \texttt{dim} ger dimensionerna
\end{itemize}
\end{frame}

%%%%%%%%% slide %%%%%%%%%

\begin{frame}{data.frame}
\protect\hypertarget{data.frame}{}
\begin{itemize}
\tightlist
\item
  Dataset, där varje variabel (kolumn) kan ha olika typer.
\item
  Varje variabel har ett namn.

  \begin{itemize}
  \tightlist
  \item
    Komma åt variabler via namnet eller ordningen.
  \end{itemize}
\item
  Varje rad innehåller ett värde per kolumn.
\item
  Lätt att lägga till och ta bort kolumner.
\end{itemize}
\end{frame}

%%%%%%%%% slide %%%%%%%%%

\begin{frame}{Listor}
\protect\hypertarget{listor}{}
\begin{itemize}
\tightlist
\item
  Tänk vektor där varje element kan vara vad som helst.
\item
  Möjligt att namnge element.
\item
  \texttt{[ ]} tar fram en del av listan
\item
  \texttt{[[ ]]} tar fram elementet
\end{itemize}
\end{frame}

\hypertarget{programkontroll}{%
\section{Programkontroll}\label{programkontroll}}

%%%%%%%%% slide %%%%%%%%%

\begin{frame}{Programkontroll}
\protect\hypertarget{programkontroll-1}{}
\begin{itemize}
\tightlist
\item
  Kontrollera körningen av program eller funktioner
\item
  Olika typer av kontroller vi kan göra:

  \begin{itemize}
  \tightlist
  \item
    Köra en annan del av koden
  \item
    Vilkorsstyra kod
  \item
    Kör kod upprepade antal gånger
  \item
    Köra kod tills ett villkor är uppfyllt
  \item
    Avbryta ett program i förtid
  \end{itemize}
\end{itemize}
\end{frame}

\hypertarget{villkorssatser}{%
\section{Villkorssatser}\label{villkorssatser}}

%%%%%%%%% slide %%%%%%%%%

\begin{frame}[fragile]{Vilkorssatser (if-else)}
\protect\hypertarget{vilkorssatser-if-else}{}
\begin{itemize}
\tightlist
\item
  Välja att utföra något baserat på logiskt villkor
\end{itemize}

\begin{Shaded}
\begin{Highlighting}[]
\ControlFlowTok{if}\NormalTok{ ( Villkor ) \{}
\NormalTok{  Kod om Villkor }\SpecialCharTok{==} \ConstantTok{TRUE}
\NormalTok{\} }\ControlFlowTok{else}\NormalTok{ \{}
\NormalTok{  Kod om Villkor }\SpecialCharTok{==} \ConstantTok{FALSE}
\NormalTok{\}}
\end{Highlighting}
\end{Shaded}

\begin{itemize}
\tightlist
\item
  Villkor ska vara någon logik

  \begin{itemize}
  \tightlist
  \item
    Tal större än noll är vanligtvis sanna.
  \end{itemize}
\item
  Om villkor är en vektor kommer den bara kolla på \imp{första}
  elementet
\end{itemize}
\end{frame}

%%%%%%%%% slide %%%%%%%%%

\begin{frame}[fragile]{If - else if - else}
\protect\hypertarget{if---else-if---else}{}
\begin{itemize}
\tightlist
\item
  Det är inte alltid man bara vill ha två olika utfall
\end{itemize}

\begin{Shaded}
\begin{Highlighting}[]
\ControlFlowTok{if}\NormalTok{ ( Villkor1 ) \{}
\NormalTok{  Kod om Villkor1 }\SpecialCharTok{==} \ConstantTok{TRUE}
\NormalTok{\} }\ControlFlowTok{else} \ControlFlowTok{if}\NormalTok{ ( Villkor2 ) \{}
\NormalTok{  Kod om Villkor2 }\SpecialCharTok{==} \ConstantTok{TRUE}
\NormalTok{\} }\ControlFlowTok{else} \ControlFlowTok{if}\NormalTok{ ( Villkor3 ) \{}
\NormalTok{  Kod om Villkor3 }\SpecialCharTok{==} \ConstantTok{TRUE}
\NormalTok{\} }\ControlFlowTok{else}\NormalTok{ \{}
\NormalTok{  Kod om Villkor1 och Villkor2 och Villkor3 }\SpecialCharTok{==} \ConstantTok{FALSE}
\NormalTok{\}}
\end{Highlighting}
\end{Shaded}

\begin{itemize}
\tightlist
\item
  Kommer välja den \imp{första} som är sann
\end{itemize}
\end{frame}

%%%%%%%%% slide %%%%%%%%%

\begin{frame}[fragile]{Villkorssatser - Exempel}
\protect\hypertarget{villkorssatser---exempel}{}
\begin{Shaded}
\begin{Highlighting}[]
\NormalTok{prgm }\OtherTok{\textless{}{-}} \StringTok{"R"}
\ControlFlowTok{if}\NormalTok{ ( prgm }\SpecialCharTok{==} \StringTok{"R"}\NormalTok{) \{ }\FunctionTok{cat}\NormalTok{(}\StringTok{"Kul med"}\NormalTok{,prgm)\}}
\end{Highlighting}
\end{Shaded}

\pause

\begin{verbatim}
## Kul med R
\end{verbatim}

\pause

\begin{Shaded}
\begin{Highlighting}[]
\NormalTok{prgm }\OtherTok{\textless{}{-}} \StringTok{"Excel"}
\ControlFlowTok{if}\NormalTok{ ( prgm }\SpecialCharTok{==} \StringTok{"R"}\NormalTok{) \{ }\FunctionTok{cat}\NormalTok{(}\StringTok{"Kul med"}\NormalTok{,prgm)\}}
\end{Highlighting}
\end{Shaded}
\end{frame}

%%%%%%%%% slide %%%%%%%%%

\begin{frame}[fragile]{Villkorssatser - Exempel II}
\protect\hypertarget{villkorssatser---exempel-ii}{}
\begin{Shaded}
\begin{Highlighting}[]
\NormalTok{prgm }\OtherTok{\textless{}{-}} \StringTok{"Excel"}
\ControlFlowTok{if}\NormalTok{ ( prgm }\SpecialCharTok{==} \StringTok{"R"}\NormalTok{) \{ }
  \FunctionTok{cat}\NormalTok{(}\StringTok{"Kul med"}\NormalTok{,prgm)}
\NormalTok{\} }\ControlFlowTok{else} \ControlFlowTok{if}\NormalTok{ ( prgm }\SpecialCharTok{==} \StringTok{"Python"}\NormalTok{ ) \{}
  \FunctionTok{print}\NormalTok{(}\StringTok{"Okej"}\NormalTok{)}
\NormalTok{\} }\ControlFlowTok{else}\NormalTok{ \{}
  \FunctionTok{print}\NormalTok{(}\StringTok{"Hmm..."}\NormalTok{)}
\NormalTok{\}}
\end{Highlighting}
\end{Shaded}

\pause

\begin{verbatim}
## [1] "Hmm..."
\end{verbatim}
\end{frame}

\hypertarget{loopar-for-loop}{%
\section{Loopar (for-loop)}\label{loopar-for-loop}}

%%%%%%%%% slide %%%%%%%%%

\begin{frame}[fragile]{For-loop}
\protect\hypertarget{for-loop}{}
\begin{itemize}
\tightlist
\item
  Upprepningar av kod
\item
  I R används \texttt{for} för loopar över vektor/lista
\end{itemize}

\begin{Shaded}
\begin{Highlighting}[]
\ControlFlowTok{for}\NormalTok{ ( elem }\ControlFlowTok{in}\NormalTok{ vektor ) \{}
  \CommentTok{\# Kod som anropas en gång per element}
  \CommentTok{\# elem är ett element i vektorn}
\NormalTok{\}}
\end{Highlighting}
\end{Shaded}

\begin{itemize}
\tightlist
\item
  Vilken vektor eller lista kan användas
\item
  Viktigt koncept: Loopens index (\texttt{elem}) är det ENDA som ändras
  i loopen.
\end{itemize}
\end{frame}

%%%%%%%%% slide %%%%%%%%%

\begin{frame}[fragile]{For-loop - Exempel}
\protect\hypertarget{for-loop---exempel}{}
\begin{Shaded}
\begin{Highlighting}[]
\NormalTok{vektor }\OtherTok{\textless{}{-}} \DecValTok{3}\SpecialCharTok{:}\DecValTok{5}
\ControlFlowTok{for}\NormalTok{ ( element }\ControlFlowTok{in}\NormalTok{ vektor ) \{}
  \FunctionTok{print}\NormalTok{( element}\SpecialCharTok{*}\DecValTok{2}\NormalTok{ )}
\NormalTok{\}}
\end{Highlighting}
\end{Shaded}

\pause

\begin{verbatim}
## [1] 6
## [1] 8
## [1] 10
\end{verbatim}
\end{frame}

%%%%%%%%% slide %%%%%%%%%

\begin{frame}[fragile]{For-loop - Exempel II}
\protect\hypertarget{for-loop---exempel-ii}{}
\begin{Shaded}
\begin{Highlighting}[]
\NormalTok{vektor }\OtherTok{\textless{}{-}} \FunctionTok{c}\NormalTok{(}\StringTok{"a"}\NormalTok{,}\StringTok{"b"}\NormalTok{,}\StringTok{"c"}\NormalTok{,}\StringTok{"d"}\NormalTok{)}
\ControlFlowTok{for}\NormalTok{ ( element }\ControlFlowTok{in}\NormalTok{ vektor ) \{}
  \FunctionTok{print}\NormalTok{( element )}
\NormalTok{\}}
\end{Highlighting}
\end{Shaded}

\pause

\begin{verbatim}
## [1] "a"
## [1] "b"
## [1] "c"
## [1] "d"
\end{verbatim}
\end{frame}

%%%%%%%%% slide %%%%%%%%%

\begin{frame}[fragile]{For-loop - Exempel III}
\protect\hypertarget{for-loop---exempel-iii}{}
\begin{itemize}
\tightlist
\item
  Kan loopa på flera olika sätt
\end{itemize}

\begin{Shaded}
\begin{Highlighting}[]
\ControlFlowTok{for}\NormalTok{ ( element }\ControlFlowTok{in}\NormalTok{ vektor ) \{}
  \FunctionTok{print}\NormalTok{( element )}
\NormalTok{\}}
\end{Highlighting}
\end{Shaded}

\begin{Shaded}
\begin{Highlighting}[]
\ControlFlowTok{for}\NormalTok{ ( index }\ControlFlowTok{in} \FunctionTok{seq\_along}\NormalTok{(vektor) ) \{}
  \FunctionTok{print}\NormalTok{( vektor[index] )}
\NormalTok{\}}
\end{Highlighting}
\end{Shaded}

\begin{Shaded}
\begin{Highlighting}[]
\ControlFlowTok{for}\NormalTok{ ( index }\ControlFlowTok{in} \DecValTok{1}\SpecialCharTok{:}\FunctionTok{length}\NormalTok{(vektor) ) \{}
  \FunctionTok{print}\NormalTok{( vektor[index] )}
\NormalTok{\}}
\end{Highlighting}
\end{Shaded}
\end{frame}

\hypertarget{nuxe4stlade-loopar}{%
\section{Nästlade loopar}\label{nuxe4stlade-loopar}}

%%%%%%%%% slide %%%%%%%%%

\begin{frame}{Nästlade loopar}
\protect\hypertarget{nuxe4stlade-loopar-1}{}
\begin{itemize}
\tightlist
\item
  Om vi vill loopa i flera nivåer
\item
  Delas ofta upp i yttre och inre loop
\item
  Tänk som en klocka

  \begin{itemize}
  \tightlist
  \item
    Timmar: Yttersta loopen
  \item
    Minuter: Inre loop, gör 60 iterationer / timme
  \item
    Sekunder: Innsersta loopen, gör 60 iteration / minut
  \end{itemize}
\end{itemize}
\end{frame}

%%%%%%%%% slide %%%%%%%%%

\begin{frame}[fragile]{Nästlade loopar - Exmepel}
\protect\hypertarget{nuxe4stlade-loopar---exmepel}{}
\begin{Shaded}
\begin{Highlighting}[]
\NormalTok{A }\OtherTok{\textless{}{-}} \FunctionTok{matrix}\NormalTok{(}\DecValTok{1}\SpecialCharTok{:}\DecValTok{6}\NormalTok{, }\AttributeTok{nrow =} \DecValTok{2}\NormalTok{)}
\ControlFlowTok{for}\NormalTok{ ( i }\ControlFlowTok{in} \DecValTok{1}\SpecialCharTok{:}\DecValTok{2}\NormalTok{ ) \{}
  \ControlFlowTok{for}\NormalTok{ ( j }\ControlFlowTok{in} \DecValTok{1}\SpecialCharTok{:}\DecValTok{3}\NormalTok{ ) \{}
\NormalTok{    text }\OtherTok{\textless{}{-}} \FunctionTok{paste}\NormalTok{(}\StringTok{"rad:"}\NormalTok{,i,}\StringTok{" kolumn:"}\NormalTok{,j,}
                  \StringTok{"värde:"}\NormalTok{,A[i,j])}
    \FunctionTok{print}\NormalTok{(text)}
\NormalTok{  \}}
\NormalTok{\}}
\end{Highlighting}
\end{Shaded}

\pause

\begin{verbatim}
## [1] "rad: 1  kolumn: 1 värde: 1"
## [1] "rad: 1  kolumn: 2 värde: 3"
## [1] "rad: 1  kolumn: 3 värde: 5"
## [1] "rad: 2  kolumn: 1 värde: 2"
## [1] "rad: 2  kolumn: 2 värde: 4"
## [1] "rad: 2  kolumn: 3 värde: 6"
\end{verbatim}
\end{frame}

\hypertarget{while-loop}{%
\section{while-loop}\label{while-loop}}

%%%%%%%%% slide %%%%%%%%%

\begin{frame}[fragile]{while-loop}
\protect\hypertarget{while-loop-1}{}
\begin{itemize}
\tightlist
\item
  Om vi inte vet antalet iterationer på förhand
\end{itemize}

\begin{Shaded}
\begin{Highlighting}[]
\ControlFlowTok{while}\NormalTok{ ( Villkor ) \{}
  \CommentTok{\# Kod som anropas så länge Villkor == TRUE}
\NormalTok{\}}
\end{Highlighting}
\end{Shaded}

\begin{itemize}
\tightlist
\item
  \imp{Varning!} Kan fortsätta hur länge som helst
\item
  \imp{Obs!} \text{Villkor} måste kunna utvärderas innan loopen startar
\item
  \texttt{repeat} repeterar kod till \texttt{break}
\end{itemize}
\end{frame}

%%%%%%%%% slide %%%%%%%%%

\begin{frame}[fragile]{while-loop - Exempel}
\protect\hypertarget{while-loop---exempel}{}
\begin{Shaded}
\begin{Highlighting}[]
\NormalTok{i }\OtherTok{\textless{}{-}} \DecValTok{1} \CommentTok{\# Obs!}
\ControlFlowTok{while}\NormalTok{ ( i }\SpecialCharTok{\textless{}} \DecValTok{5}\NormalTok{ ) \{}
  \FunctionTok{print}\NormalTok{(i)}
\NormalTok{  i }\OtherTok{\textless{}{-}}\NormalTok{ i }\SpecialCharTok{+} \DecValTok{1}
\NormalTok{\}}
\end{Highlighting}
\end{Shaded}

\pause

\begin{verbatim}
## [1] 1
## [1] 2
## [1] 3
## [1] 4
\end{verbatim}
\end{frame}

\hypertarget{kontrollstrukturer-fuxf6r-loopar}{%
\section{Kontrollstrukturer för
loopar}\label{kontrollstrukturer-fuxf6r-loopar}}

%%%%%%%%% slide %%%%%%%%%

\begin{frame}{Kontrollstrukturer för loopar}
\protect\hypertarget{kontrollstrukturer-fuxf6r-loopar-1}{}
\begin{itemize}
\tightlist
\item
  Vill ofta kontrollera hur looparna arbetar
\item
  Finns följande kontrollstrukturer:
\end{itemize}

\begin{longtable}[]{@{}ll@{}}
\toprule
\textbf{Kontrollstruktur} & \textbf{Effet} \\
\midrule
\endhead
\texttt{next()} & Börja med nästa iteration direkt \\
\texttt{break()} & Avbryt den aktuella / innersta loopen \\
\texttt{stop()} & Avbryter allt och genererar ett felmeddelande \\
\bottomrule
\end{longtable}
\end{frame}

%%%%%%%%% slide %%%%%%%%%

\begin{frame}[fragile]{next() - Exempel}
\protect\hypertarget{next---exempel}{}
\begin{Shaded}
\begin{Highlighting}[]
\NormalTok{i }\OtherTok{\textless{}{-}} \DecValTok{0}
\ControlFlowTok{while}\NormalTok{ ( i }\SpecialCharTok{\textless{}} \DecValTok{11}\NormalTok{ ) \{}
\NormalTok{  i }\OtherTok{\textless{}{-}}\NormalTok{ i }\SpecialCharTok{+} \DecValTok{1}
  \ControlFlowTok{if}\NormalTok{ ( i }\SpecialCharTok{\%\%} \DecValTok{2} \SpecialCharTok{==} \DecValTok{0}\NormalTok{) \{ }\ControlFlowTok{next}\NormalTok{() \}}
  \FunctionTok{print}\NormalTok{(i)}
\NormalTok{\}}
\end{Highlighting}
\end{Shaded}

\pause

\begin{verbatim}
## [1] 1
## [1] 3
## [1] 5
## [1] 7
## [1] 9
## [1] 11
\end{verbatim}
\end{frame}

%%%%%%%%% slide %%%%%%%%%

\begin{frame}[fragile]{break() - exempel}
\protect\hypertarget{break---exempel}{}
\begin{Shaded}
\begin{Highlighting}[]
\ControlFlowTok{for}\NormalTok{ ( i }\ControlFlowTok{in} \DecValTok{1}\SpecialCharTok{:}\DecValTok{3}\NormalTok{ ) \{}
  \ControlFlowTok{for}\NormalTok{ ( letter }\ControlFlowTok{in} \FunctionTok{c}\NormalTok{(}\StringTok{"a"}\NormalTok{,}\StringTok{"b"}\NormalTok{,}\StringTok{"c"}\NormalTok{) ) \{}
    \ControlFlowTok{if}\NormalTok{ ( letter }\SpecialCharTok{==} \StringTok{"b"}\NormalTok{ ) \{ }\ControlFlowTok{break}\NormalTok{() \}}
    \FunctionTok{print}\NormalTok{(letter)}
\NormalTok{  \}}
\NormalTok{\}}
\end{Highlighting}
\end{Shaded}

\pause

\begin{verbatim}
## [1] "a"
## [1] "a"
## [1] "a"
\end{verbatim}
\end{frame}

%%%%%%%%% slide %%%%%%%%%

\begin{frame}[fragile]{stop() - Exempel}
\protect\hypertarget{stop---exempel}{}
\begin{Shaded}
\begin{Highlighting}[]
\ControlFlowTok{for}\NormalTok{ ( i }\ControlFlowTok{in} \DecValTok{1}\SpecialCharTok{:}\DecValTok{3}\NormalTok{ ) \{}
  \ControlFlowTok{for}\NormalTok{ ( letter }\ControlFlowTok{in} \FunctionTok{c}\NormalTok{(}\StringTok{"a"}\NormalTok{,}\StringTok{"b"}\NormalTok{,}\StringTok{"c"}\NormalTok{) ) \{}
    \ControlFlowTok{if}\NormalTok{ ( letter }\SpecialCharTok{==} \StringTok{"b"}\NormalTok{ ) \{ }\FunctionTok{stop}\NormalTok{(}\StringTok{"Det blev fel!"}\NormalTok{) \}}
    \FunctionTok{print}\NormalTok{(letter)}
\NormalTok{  \}}
\NormalTok{\}}
\end{Highlighting}
\end{Shaded}

\pause

\begin{verbatim}
## [1] "a"
\end{verbatim}

\begin{verbatim}
## Error: Det blev fel!
\end{verbatim}
\end{frame}

\hypertarget{varningsmeddelanden-och-debugg}{%
\section{Varningsmeddelanden och
debugg}\label{varningsmeddelanden-och-debugg}}

%%%%%%%%% slide %%%%%%%%%

\begin{frame}{Varningsmeddelanden}
\protect\hypertarget{varningsmeddelanden}{}
\begin{itemize}
\tightlist
\item
  \texttt{stop()} avbryter funktioner/loopar och meddelar ett fel
\item
  \texttt{warning()} skapar en varning, som inte avbryter
\item
  Varningar sparas och skrivs ut sist
\item
  \texttt{warnings()} skriver ut tidigare varningar
\end{itemize}
\end{frame}

%%%%%%%%% slide %%%%%%%%%

\begin{frame}[fragile]{Varningsmeddelanden - Exempel}
\protect\hypertarget{varningsmeddelanden---exempel}{}
\begin{Shaded}
\begin{Highlighting}[]
\ControlFlowTok{for}\NormalTok{ ( chr }\ControlFlowTok{in} \FunctionTok{c}\NormalTok{(}\StringTok{"a"}\NormalTok{,}\StringTok{"b"}\NormalTok{) ) \{}
  \FunctionTok{print}\NormalTok{(chr)}
  \FunctionTok{warning}\NormalTok{( }\FunctionTok{paste}\NormalTok{(}\StringTok{"Farligt värde"}\NormalTok{,chr) )}
\NormalTok{\}}
\end{Highlighting}
\end{Shaded}

\pause

\begin{Shaded}
\begin{Highlighting}[]
\StringTok{"a"}
\StringTok{"b"}
\NormalTok{Warning messages}\SpecialCharTok{:}
\DecValTok{1}\SpecialCharTok{:}\NormalTok{ Farligt värde a }
\DecValTok{2}\SpecialCharTok{:}\NormalTok{ Farligt värde b }
\end{Highlighting}
\end{Shaded}
\end{frame}

%%%%%%%%% slide %%%%%%%%%

\begin{frame}{Debugging}
\protect\hypertarget{debugging}{}
\begin{itemize}
\tightlist
\item
  Det uppstår ofta fel vid programmering
\item
  Debugging handlar om att hitta orsaken
\item
  Olika typer av fel:

  \begin{itemize}
  \tightlist
  \item
    Syntaktiska fel: Felaktig syntax i koden
  \item
    Semantiska fel: Olämplig användning av objekt/funktioner
  \item
    Logiska fel: Programmet löser inte det tänkta problemet
  \end{itemize}
\end{itemize}
\end{frame}

%%%%%%%%% slide %%%%%%%%%

\begin{frame}{Debugging II}
\protect\hypertarget{debugging-ii}{}
\begin{itemize}
\tightlist
\item
  Använd \texttt{cat()}, \texttt{print()} eller \texttt{message()} för
  att skriva ut värden under körning.

  \begin{itemize}
  \tightlist
  \item
    Använd \texttt{paste()} för att kombinera text och variabler till en
    sträng.
  \end{itemize}
\item
  \texttt{browser()} Hoppar in i funktionen

  \begin{itemize}
  \tightlist
  \item
    \texttt{n}: kör nästa rad
  \item
    \texttt{c}: kör allt i funktion / loop
  \item
    \texttt{Q}: avsluta
  \end{itemize}
\item
  \texttt{debugg()} Hoppa in i funktionen från början
\end{itemize}
\end{frame}

\end{document}
