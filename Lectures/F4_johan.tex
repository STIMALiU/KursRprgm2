% Options for packages loaded elsewhere
\PassOptionsToPackage{unicode}{hyperref}
\PassOptionsToPackage{hyphens}{url}
\documentclass[
  11pt,
  ignorenonframetext,
]{beamer}
\newif\ifbibliography
\usepackage{pgfpages}
\setbeamertemplate{caption}[numbered]
\setbeamertemplate{caption label separator}{: }
\setbeamercolor{caption name}{fg=normal text.fg}
\beamertemplatenavigationsymbolsempty
% remove section numbering
\setbeamertemplate{part page}{
  \centering
  \begin{beamercolorbox}[sep=16pt,center]{part title}
    \usebeamerfont{part title}\insertpart\par
  \end{beamercolorbox}
}
\setbeamertemplate{section page}{
  \centering
  \begin{beamercolorbox}[sep=12pt,center]{section title}
    \usebeamerfont{section title}\insertsection\par
  \end{beamercolorbox}
}
\setbeamertemplate{subsection page}{
  \centering
  \begin{beamercolorbox}[sep=8pt,center]{subsection title}
    \usebeamerfont{subsection title}\insertsubsection\par
  \end{beamercolorbox}
}
% Prevent slide breaks in the middle of a paragraph
\widowpenalties 1 10000
\raggedbottom
\AtBeginPart{
  \frame{\partpage}
}
\AtBeginSection{
  \ifbibliography
  \else
    \frame{\sectionpage}
  \fi
}
\AtBeginSubsection{
  \frame{\subsectionpage}
}
\usepackage{iftex}
\ifPDFTeX
  \usepackage[T1]{fontenc}
  \usepackage[utf8]{inputenc}
  \usepackage{textcomp} % provide euro and other symbols
\else % if luatex or xetex
  \usepackage{unicode-math} % this also loads fontspec
  \defaultfontfeatures{Scale=MatchLowercase}
  \defaultfontfeatures[\rmfamily]{Ligatures=TeX,Scale=1}
\fi
\usepackage{lmodern}
\ifPDFTeX\else
  % xetex/luatex font selection
\fi
% Use upquote if available, for straight quotes in verbatim environments
\IfFileExists{upquote.sty}{\usepackage{upquote}}{}
\IfFileExists{microtype.sty}{% use microtype if available
  \usepackage[]{microtype}
  \UseMicrotypeSet[protrusion]{basicmath} % disable protrusion for tt fonts
}{}
\makeatletter
\@ifundefined{KOMAClassName}{% if non-KOMA class
  \IfFileExists{parskip.sty}{%
    \usepackage{parskip}
  }{% else
    \setlength{\parindent}{0pt}
    \setlength{\parskip}{6pt plus 2pt minus 1pt}}
}{% if KOMA class
  \KOMAoptions{parskip=half}}
\makeatother
\usepackage{color}
\usepackage{fancyvrb}
\newcommand{\VerbBar}{|}
\newcommand{\VERB}{\Verb[commandchars=\\\{\}]}
\DefineVerbatimEnvironment{Highlighting}{Verbatim}{commandchars=\\\{\}}
% Add ',fontsize=\small' for more characters per line
\usepackage{framed}
\definecolor{shadecolor}{RGB}{248,248,248}
\newenvironment{Shaded}{\begin{snugshade}}{\end{snugshade}}
\newcommand{\AlertTok}[1]{\textcolor[rgb]{0.94,0.16,0.16}{#1}}
\newcommand{\AnnotationTok}[1]{\textcolor[rgb]{0.56,0.35,0.01}{\textbf{\textit{#1}}}}
\newcommand{\AttributeTok}[1]{\textcolor[rgb]{0.13,0.29,0.53}{#1}}
\newcommand{\BaseNTok}[1]{\textcolor[rgb]{0.00,0.00,0.81}{#1}}
\newcommand{\BuiltInTok}[1]{#1}
\newcommand{\CharTok}[1]{\textcolor[rgb]{0.31,0.60,0.02}{#1}}
\newcommand{\CommentTok}[1]{\textcolor[rgb]{0.56,0.35,0.01}{\textit{#1}}}
\newcommand{\CommentVarTok}[1]{\textcolor[rgb]{0.56,0.35,0.01}{\textbf{\textit{#1}}}}
\newcommand{\ConstantTok}[1]{\textcolor[rgb]{0.56,0.35,0.01}{#1}}
\newcommand{\ControlFlowTok}[1]{\textcolor[rgb]{0.13,0.29,0.53}{\textbf{#1}}}
\newcommand{\DataTypeTok}[1]{\textcolor[rgb]{0.13,0.29,0.53}{#1}}
\newcommand{\DecValTok}[1]{\textcolor[rgb]{0.00,0.00,0.81}{#1}}
\newcommand{\DocumentationTok}[1]{\textcolor[rgb]{0.56,0.35,0.01}{\textbf{\textit{#1}}}}
\newcommand{\ErrorTok}[1]{\textcolor[rgb]{0.64,0.00,0.00}{\textbf{#1}}}
\newcommand{\ExtensionTok}[1]{#1}
\newcommand{\FloatTok}[1]{\textcolor[rgb]{0.00,0.00,0.81}{#1}}
\newcommand{\FunctionTok}[1]{\textcolor[rgb]{0.13,0.29,0.53}{\textbf{#1}}}
\newcommand{\ImportTok}[1]{#1}
\newcommand{\InformationTok}[1]{\textcolor[rgb]{0.56,0.35,0.01}{\textbf{\textit{#1}}}}
\newcommand{\KeywordTok}[1]{\textcolor[rgb]{0.13,0.29,0.53}{\textbf{#1}}}
\newcommand{\NormalTok}[1]{#1}
\newcommand{\OperatorTok}[1]{\textcolor[rgb]{0.81,0.36,0.00}{\textbf{#1}}}
\newcommand{\OtherTok}[1]{\textcolor[rgb]{0.56,0.35,0.01}{#1}}
\newcommand{\PreprocessorTok}[1]{\textcolor[rgb]{0.56,0.35,0.01}{\textit{#1}}}
\newcommand{\RegionMarkerTok}[1]{#1}
\newcommand{\SpecialCharTok}[1]{\textcolor[rgb]{0.81,0.36,0.00}{\textbf{#1}}}
\newcommand{\SpecialStringTok}[1]{\textcolor[rgb]{0.31,0.60,0.02}{#1}}
\newcommand{\StringTok}[1]{\textcolor[rgb]{0.31,0.60,0.02}{#1}}
\newcommand{\VariableTok}[1]{\textcolor[rgb]{0.00,0.00,0.00}{#1}}
\newcommand{\VerbatimStringTok}[1]{\textcolor[rgb]{0.31,0.60,0.02}{#1}}
\newcommand{\WarningTok}[1]{\textcolor[rgb]{0.56,0.35,0.01}{\textbf{\textit{#1}}}}
\usepackage{longtable,booktabs,array}
\usepackage{calc} % for calculating minipage widths
\usepackage{caption}
% Make caption package work with longtable
\makeatletter
\def\fnum@table{\tablename~\thetable}
\makeatother
\setlength{\emergencystretch}{3em} % prevent overfull lines
\providecommand{\tightlist}{%
  \setlength{\itemsep}{0pt}\setlength{\parskip}{0pt}}
\usetheme[progressbar=frametitle,block=fill]{metropolis} %numbering=none

%%% USEFUL PACKAGES
%\usepackage{showframe} % For debugging positioning
\usepackage{etex} % If too many packages
% Encoding and language
\usepackage[utf8]{inputenc}
\usepackage{babel}
\usepackage{amsmath, amssymb}
\usepackage{natbib}
%\usepackage{booktabs}
%\usepackage{algorithmic}
\usepackage{algorithm}
\usepackage{caption}
%\usepackage{animate} % Animations
\usepackage{bm} % Bold math
\usepackage{bbm}
%\usepackage{url}
%\usepackage{pifont}
%\usepackage{ulem} % Used for strikeouts \sout
%\usepackage{stackengine}
%\usepackage{enumitem}
%\setlist[description]{leftmargin=\parindent,labelindent=\parindent}
%\usepackage{colortbl} % Used for colored rows in tables


%%% GRAPHICS
\usepackage{graphicx}
\graphicspath{{./figs/}}


%%% COLORS
\setbeamercolor{background canvas}{bg=white}
\def\BlankFrame{
	\bgroup
	%\pdfpageheight 29.7cm
	\setbeamercolor{background canvas}{bg=}
	\begin{frame}[plain]
	\end{frame}
	%\makeatletter
	%\pdfpageheight \beamer@paperheight
	%\makeatother
	\egroup}

\usepackage{xcolor}
\definecolor{DarkGreen}{HTML}{00B200}
\definecolor{LightBlue}{HTML}{0090D9}
\definecolor{gold}{rgb}{.812,.710,.231}
% Text markup
%\setbeamercolor{alerted text}{fg=red}
\newcommand{\blue}[1]{\textcolor{blue}{#1}}
\newcommand{\red}[1]{\textcolor{red}{#1}}
\newcommand{\grey}[1]{\textcolor{gray}{#1}}
\newcommand{\orange}[1]{\textcolor{mLightBrown}{#1}}
\newcommand\myheading[1]{\textbf{#1}}
\newcommand\myemph[1]{\underline{\emph{#1}}}
\newcommand\textexample[1]{\textit{\textbf{#1}}}

%%% SPACING
\newcommand\vws[1][1]{\vspace{#1\baselineskip}} % vertical white space
%\newcommand\strt[1][1.5ex]{\rule[-.05\baselineskip]{0pt}{#1}} % strut
\newcommand\strt[2]{\rule[-#1ex]{0pt}{#2ex}} % strut
\newcommand\Hrule{\vspace{1ex} \hrule \vspace{1ex}} % Horisontal rule with some space after

%%% MISC
\newcommand\articleref[4]{\noindent\begin{minipage}[t]{0.04\textwidth}
		\vspace{0pt} 
		\pgfuseimage{beamericonarticle}
	\end{minipage}%
	\begin{minipage}[t]{0.96\textwidth}
		\vspace{0pt}
		#1. \textbf{#2.} \textit{#3}, #4.
	\end{minipage}}

%%% METROPOLIS THEME SPECIFIC
\makeatletter
\setlength{\metropolis@progressonsectionpage@linewidth}{1pt}
\makeatother
%\setbeamercolor{progress bar}{fg=red,bg=red!50}


%%% TEXTPOS
\usepackage[absolute,overlay]{textpos} % option showboxes is useful in draft mode
\setlength{\TPHorizModule}{\paperwidth}
\setlength{\TPVertModule}{\paperheight}
\textblockorigin{0pt}{10mm} % start everything at top-left, below gray 


%%% TIKZ/PGFPLOTS
\usepackage{tikz}
\usetikzlibrary{arrows,positioning,calc,shapes.geometric}
%\usetikzlibrary{arrows,calc,shapes.geometric,decorations.pathmorphing,backgrounds,positioning,fit,petri,decorations.pathreplacing}
%\usepackage{pgfplots}
%\pgfplotsset{compat = 1.3}


%%% BLOCKS AND BOXES
% Changing colors of blocks
%\setbeamercolor{block title alerted}{bg=UURed,fg=palette primary.fg}
%\setbeamercolor{block body alerted}{bg=UURed!15}
\setbeamercolor{block title alerted}{bg=mLightBrown,fg=palette primary.fg}
\setbeamercolor{block body alerted}{bg=mLightBrown!15}
%\setbeamercolor{block title example}{bg=UUGreen,fg=palette primary.fg}
%\setbeamercolor{block body example}{bg=UUGreen!10}
% \mybox is a rectangular box
\usepackage{boxedminipage}
\setlength\fboxrule{2pt}
\setlength\fboxsep{2\fboxsep}
\newcommand\mybox[3][\textwidth]{
  {\color{#2}
    \begin{boxedminipage}{#1}
      {\color{palette primary.bg} #3}
    \end{boxedminipage}}%
}   
\usepackage{tcolorbox}
\tcbset{arc=1mm,grow to left by=3mm,grow to right by=3mm,left=2mm}
%\newenvironment{redbox}{%
%	\begin{tcolorbox}[colback=UURed!15,colframe=UURed]}{%
%	\end{tcolorbox}}
%\newenvironment{greenbox}{%
%	\begin{tcolorbox}[colback=UUGreen!15,colframe=UUGreen]}{%
%	\end{tcolorbox}}
\newenvironment{redbox}{%
	\begin{tcolorbox}[colback=red!15,colframe=red]}{%
	\end{tcolorbox}}
\newenvironment{greenbox}{%
	\begin{tcolorbox}[colback=DarkGreen!15,colframe=DarkGreen]}{%
	\end{tcolorbox}}
\newenvironment{graybox}{%
	\begin{tcolorbox}[colback=mDarkTeal!5,colframe=mDarkTeal]}{%
	\end{tcolorbox}}
\newenvironment{orangebox}{%
\begin{tcolorbox}[colback=mLightBrown!15,colframe=mLightBrown]}{%
	\end{tcolorbox}}
\newenvironment{bwbox}{%
	\begin{tcolorbox}[colback=white,colframe=black]}{%
\end{tcolorbox}}
\newenvironment{bluebox}{%
	\begin{tcolorbox}[colback=LightBlue!15,colframe=LightBlue]}{%
\end{tcolorbox}}


%%%%%%%%% NEW MACROS

\newcommand\imp[1]{\alert{\textbf{#1}}}
\newcommand\bfit[1]{\textbf{\textit{#1}}}
\newcommand\good{\color{DarkGreen}{$\blacktriangle$}} % used in lists
\newcommand\bad{\color{red}{$\blacktriangledown$}} % used in lists

\usepackage{bookmark}
\IfFileExists{xurl.sty}{\usepackage{xurl}}{} % add URL line breaks if available
\urlstyle{same}
\hypersetup{
  pdftitle={R-programmering VT2025},
  pdfauthor={Johan Alenlöv},
  hidelinks,
  pdfcreator={LaTeX via pandoc}}

\title{R-programmering VT2025}
\subtitle{Föreläsning 4}
\author{Johan Alenlöv}
\date{}
\institute{Linköpings Universitet}

\begin{document}
\frame{\titlepage}

\begin{frame}{Föreläsning 4:}
\phantomsection\label{fuxf6reluxe4sning-4}
\begin{itemize}
\tightlist
\item
  Mer om funktioner
\item
  Globala och lokala miljöer i R
\item
  Bra och effektiv kod
\item
  R-paket
\item
  Dokumentation av kod
\item
  *apply-funktioner
\end{itemize}
\end{frame}

\section{Mer om funktioner}\label{mer-om-funktioner}

\begin{frame}{Funktioner - repetition}
\phantomsection\label{funktioner---repetition}
Vi påminner oss om att en funktion i R består av

\begin{itemize}
\tightlist
\item
  ett funktionsnamn
\item
  en funktionsdefinition \texttt{function()}
\item
  argument (0 eller flera)
\item
  måsvingar
\item
  programkod
\item
  \texttt{return()}
\end{itemize}
\end{frame}

\begin{frame}[fragile]{Funktioner - Allmänt I}
\phantomsection\label{funktioner---allmuxe4nt-i}
\begin{itemize}
\tightlist
\item
  Funktioner är objekt.
\end{itemize}

\begin{Shaded}
\begin{Highlighting}[]
\NormalTok{f }\OtherTok{\textless{}{-}} \ControlFlowTok{function}\NormalTok{(x,y) \{}
\NormalTok{  z }\OtherTok{\textless{}{-}}\NormalTok{ x}\SpecialCharTok{\^{}}\DecValTok{2} \SpecialCharTok{{-}}\NormalTok{ y}\SpecialCharTok{\^{}}\DecValTok{2}
  \FunctionTok{return}\NormalTok{(z)}
\NormalTok{\}}
\end{Highlighting}
\end{Shaded}

\begin{Shaded}
\begin{Highlighting}[]
\FunctionTok{typeof}\NormalTok{(f)}
\end{Highlighting}
\end{Shaded}

\begin{verbatim}
## [1] "closure"
\end{verbatim}

\begin{Shaded}
\begin{Highlighting}[]
\FunctionTok{class}\NormalTok{(f)}
\end{Highlighting}
\end{Shaded}

\begin{verbatim}
## [1] "function"
\end{verbatim}
\end{frame}

\begin{frame}{Funktioner - Allmänt II}
\phantomsection\label{funktioner---allmuxe4nt-ii}
\begin{itemize}
\tightlist
\item
  Funktioner har tre delar:

  \begin{itemize}
  \tightlist
  \item
    argument
  \item
    funktionskropp
  \item
    miljö
  \end{itemize}
\end{itemize}
\end{frame}

\begin{frame}[fragile]{Funktioner - Argument}
\phantomsection\label{funktioner---argument}
\begin{itemize}
\tightlist
\item
  Argument är insignalerna
\end{itemize}

\begin{Shaded}
\begin{Highlighting}[]
\FunctionTok{formals}\NormalTok{(f)}
\end{Highlighting}
\end{Shaded}

\begin{verbatim}
## $x
## 
## 
## $y
\end{verbatim}
\end{frame}

\begin{frame}[fragile]{Funktioner - Funktionskropp}
\phantomsection\label{funktioner---funktionskropp}
\begin{itemize}
\tightlist
\item
  Kroppen är koden som körs
\end{itemize}

\begin{Shaded}
\begin{Highlighting}[]
\FunctionTok{body}\NormalTok{(f)}
\end{Highlighting}
\end{Shaded}

\begin{verbatim}
## {
##     z <- x^2 - y^2
##     return(z)
## }
\end{verbatim}
\end{frame}

\begin{frame}[fragile]{Funktioner - Miljö}
\phantomsection\label{funktioner---miljuxf6}
\begin{itemize}
\tightlist
\item
  Miljön är vart funktioner finns
\end{itemize}

\begin{Shaded}
\begin{Highlighting}[]
\FunctionTok{environment}\NormalTok{(f)}
\end{Highlighting}
\end{Shaded}

\begin{verbatim}
## <environment: R_GlobalEnv>
\end{verbatim}
\end{frame}

\begin{frame}[fragile]{Funktioner - Allmänt III}
\phantomsection\label{funktioner---allmuxe4nt-iii}
\begin{itemize}
\tightlist
\item
  En funktion kan ha en funktion som argument
\end{itemize}

\begin{Shaded}
\begin{Highlighting}[]
\NormalTok{complex\_function }\OtherTok{\textless{}{-}} \ControlFlowTok{function}\NormalTok{(x) \{}
  \FunctionTok{return}\NormalTok{(x }\SpecialCharTok{*} \FunctionTok{exp}\NormalTok{(}\SpecialCharTok{{-}}\NormalTok{x))}
\NormalTok{\}}
\FunctionTok{integrate}\NormalTok{(complex\_function, }\AttributeTok{lower =} \DecValTok{0}\NormalTok{, }\AttributeTok{upper =} \DecValTok{1}\NormalTok{)}
\end{Highlighting}
\end{Shaded}

\begin{verbatim}
## 0.2642411 with absolute error < 2.9e-15
\end{verbatim}

\begin{itemize}
\tightlist
\item
  Funktioner kan returnera funktioner
\end{itemize}
\end{frame}

\begin{frame}[fragile]{Funktioner - Allmänt IV}
\phantomsection\label{funktioner---allmuxe4nt-iv}
\begin{itemize}
\tightlist
\item
  Vi behöver inte namnge argumenten
\end{itemize}

\begin{Shaded}
\begin{Highlighting}[]
\FunctionTok{f}\NormalTok{(}\DecValTok{1}\NormalTok{, }\DecValTok{5}\NormalTok{)}
\end{Highlighting}
\end{Shaded}

\begin{verbatim}
## [1] -24
\end{verbatim}

\begin{Shaded}
\begin{Highlighting}[]
\FunctionTok{f}\NormalTok{(}\AttributeTok{x =} \DecValTok{1}\NormalTok{, }\AttributeTok{y =} \DecValTok{5}\NormalTok{)}
\end{Highlighting}
\end{Shaded}

\begin{verbatim}
## [1] -24
\end{verbatim}

\begin{itemize}
\tightlist
\item
  Ordningen spelar ingen roll
\end{itemize}

\begin{Shaded}
\begin{Highlighting}[]
\FunctionTok{f}\NormalTok{(}\AttributeTok{y =} \DecValTok{5}\NormalTok{, }\AttributeTok{x =} \DecValTok{1}\NormalTok{)}
\end{Highlighting}
\end{Shaded}

\begin{verbatim}
## [1] -24
\end{verbatim}
\end{frame}

\begin{frame}[fragile]{Funktioner - Allmänt V}
\phantomsection\label{funktioner---allmuxe4nt-v}
\begin{itemize}
\tightlist
\item
  \texttt{return()} och måsvingar behövs inte för små funktioner
\end{itemize}

\begin{Shaded}
\begin{Highlighting}[]
\NormalTok{f }\OtherTok{\textless{}{-}} \ControlFlowTok{function}\NormalTok{(x) }\DecValTok{3}\SpecialCharTok{*}\NormalTok{x }\SpecialCharTok{{-}} \DecValTok{5}
\FunctionTok{f}\NormalTok{(}\DecValTok{1}\NormalTok{)}
\end{Highlighting}
\end{Shaded}

\begin{verbatim}
## [1] -2
\end{verbatim}
\end{frame}

\begin{frame}[fragile]{Funktioner - Allmänt VI}
\phantomsection\label{funktioner---allmuxe4nt-vi}
\begin{itemize}
\tightlist
\item
  Vi kan sätta defaultvärden
\end{itemize}

\begin{Shaded}
\begin{Highlighting}[]
\NormalTok{f }\OtherTok{\textless{}{-}} \ControlFlowTok{function}\NormalTok{(}\AttributeTok{x =} \DecValTok{10}\NormalTok{) }\DecValTok{3}\SpecialCharTok{*}\NormalTok{x }\SpecialCharTok{{-}} \DecValTok{5}
\FunctionTok{f}\NormalTok{(}\DecValTok{1}\NormalTok{)}
\end{Highlighting}
\end{Shaded}

\begin{verbatim}
## [1] -2
\end{verbatim}

\begin{Shaded}
\begin{Highlighting}[]
\FunctionTok{f}\NormalTok{()}
\end{Highlighting}
\end{Shaded}

\begin{verbatim}
## [1] 25
\end{verbatim}
\end{frame}

\section{Globala och lokala miljöer i
R}\label{globala-och-lokala-miljuxf6er-i-r}

\begin{frame}{Miljöer i R}
\phantomsection\label{miljuxf6er-i-r}
\begin{itemize}
\tightlist
\item
  Objekt kan definerass/skapas i

  \begin{itemize}
  \tightlist
  \item
    Den globala miljön
  \item
    Lokala miljön
  \item
    Namespaces
  \end{itemize}
\end{itemize}
\end{frame}

\begin{frame}{Hur vet R vad vi menar}
\phantomsection\label{hur-vet-r-vad-vi-menar}
\begin{itemize}
\tightlist
\item
  R:s söklista:

  \begin{enumerate}
  \tightlist
  \item
    Lokala miljöer
  \item
    Globala miljön
  \item
    Vidare i den ordning \texttt{namespaces} är laddade
  \end{enumerate}
\end{itemize}
\end{frame}

\begin{frame}[fragile]{Fria vaariabler}
\phantomsection\label{fria-vaariabler}
\begin{Shaded}
\begin{Highlighting}[]
\NormalTok{f }\OtherTok{\textless{}{-}} \ControlFlowTok{function}\NormalTok{(x) x }\SpecialCharTok{+}\NormalTok{ y}
\end{Highlighting}
\end{Shaded}

Objektet/värdet för ``fria'' variabler undersöks först i den miljön
funktionen var definerad/skapad.

Efter det söker R i samma ordning som förra sliden.
\end{frame}

\begin{frame}[fragile]{Fria variabler - Exempel}
\phantomsection\label{fria-variabler---exempel}
\begin{Shaded}
\begin{Highlighting}[]
\NormalTok{f }\OtherTok{\textless{}{-}} \ControlFlowTok{function}\NormalTok{(x) x }\SpecialCharTok{+}\NormalTok{ y}
\FunctionTok{f}\NormalTok{(}\DecValTok{3}\NormalTok{)}
\end{Highlighting}
\end{Shaded}

\begin{verbatim}
## [1] "Error in f(3) : object 'y' not found"
\end{verbatim}

\begin{Shaded}
\begin{Highlighting}[]
\NormalTok{y }\OtherTok{\textless{}{-}} \DecValTok{2}
\FunctionTok{f}\NormalTok{(}\DecValTok{3}\NormalTok{)}
\end{Highlighting}
\end{Shaded}

\begin{verbatim}
## [1] 5
\end{verbatim}
\end{frame}

\section{Bra och effektiv kod}\label{bra-och-effektiv-kod}

\begin{frame}{God (vetenskaplig) programmering}
\phantomsection\label{god-vetenskaplig-programmering}
\begin{itemize}
\tightlist
\item
  Programmering är en viktig del av analys
\item
  Forskare/statistiker spenderar stor del av sin tid med att skriva kod
  -Det är enkelt att det blir fel
\end{itemize}
\end{frame}

\begin{frame}{8 steg för god programmering}
\phantomsection\label{steg-fuxf6r-god-programmering}
\begin{enumerate}
\tightlist
\item
  Skriv kod för människor, inte datorer
\item
  Låt datorn gör arbetet
\item
  Ta små steg
\item
  Upprepa aldrig dig själv (eller andra)
\item
  Planera för misstag
\item
  Optimera kod först när den fungerar
\item
  Dokumentera
\item
  Samarbeta
\end{enumerate}
\end{frame}

\begin{frame}{1. Skriv kod för människor, inte datorer}
\phantomsection\label{skriv-kod-fuxf6r-muxe4nniskor-inte-datorer}
\begin{itemize}
\tightlist
\item
  En läsare ska inte behöva hålla allt i minnet.
\item
  Ge funktioner och variabler meningsfulla namn.
\item
  Använd en konsekvent kodstil och formatering
\end{itemize}
\end{frame}

\begin{frame}{2. Låt datorn gör arbetet}
\phantomsection\label{luxe5t-datorn-guxf6r-arbetet}
\begin{itemize}
\tightlist
\item
  Låt datorn upprepa uppgifter.
\item
  Skriv inte samma kod flera gånger.
\item
  Automatisera arbetsflödet.
\end{itemize}
\end{frame}

\begin{frame}{3. Ta små steg}
\phantomsection\label{ta-smuxe5-steg}
\begin{itemize}
\tightlist
\item
  Gör små ändringar, bygg upp funktionen i små steg.

  \begin{itemize}
  \tightlist
  \item
    Testa ofta!
  \end{itemize}
\item
  Versionshantera din kod (överkurs).

  \begin{itemize}
  \tightlist
  \item
    Git finns inbyggt stöd för i R-studio.
  \item
    Finns enkel versionshantering i de vaniga molntjänsterna.
  \end{itemize}
\end{itemize}
\end{frame}

\begin{frame}{4. Upprepa aldrig dig själv (eller andra)}
\phantomsection\label{upprepa-aldrig-dig-sjuxe4lv-eller-andra}
\begin{itemize}
\tightlist
\item
  All data ska bara finnas på ett ställe.
\item
  Undivk att klippa och klistra din kod, skriv en funktion istället.
\item
  Återvinn kod istället för att skriva ny.
\end{itemize}
\end{frame}

\begin{frame}{5. Planera för misstag}
\phantomsection\label{planera-fuxf6r-misstag}
\begin{itemize}
\tightlist
\item
  Skapa kontroller av input (och output).

  \begin{itemize}
  \tightlist
  \item
    Hitta fel så tidigt som möjligt och avbryt.
  \end{itemize}
\item
  Använd ett testpaket (överkurs)

  \begin{itemize}
  \tightlist
  \item
    i R finns \texttt{testthat}
  \end{itemize}
\item
  Gör om buggar till testfall.
\item
  Använd debuggers!

  \begin{itemize}
  \tightlist
  \item
    \texttt{browser()}
  \item
    \texttt{debug()}
  \end{itemize}
\end{itemize}
\end{frame}

\begin{frame}{6. Optimera kod först när den fungerar}
\phantomsection\label{optimera-kod-fuxf6rst-nuxe4r-den-fungerar}
\begin{itemize}
\tightlist
\item
  Se till att koden fungerar och löser uppgiften först.
\item
  Först efter det kan du fundera på om den behöver vara:

  \begin{itemize}
  \tightlist
  \item
    snabbare
  \item
    mer minneseffektiv
  \item
    mer användarvänlig
  \end{itemize}
\item
  Använd profileringsverktyg (överkurs).

  \begin{itemize}
  \tightlist
  \item
    i R finns \texttt{Rprof()}
  \item
    i R-studio kan man klicka på ``Profile''.
  \end{itemize}
\item
  Använd ett högnivåspråk, som R.
\end{itemize}
\end{frame}

\begin{frame}{7. Dokumentera}
\phantomsection\label{dokumentera}
\begin{itemize}
\tightlist
\item
  Dokumentera syftet med koden, inte vad koden gör.
\item
  Gör koden lätt att förstå.
\item
  Kombinera kod och dokumentation.

  \begin{itemize}
  \tightlist
  \item
    Lägg in många kommentarer med \#
  \item
    Använd \texttt{knitr} (miniprojektet)
  \item
    Generera dokumentation med \texttt{roxygen2}
  \end{itemize}
\end{itemize}
\end{frame}

\begin{frame}{8. Samarbeta}
\phantomsection\label{samarbeta}
\begin{itemize}
\tightlist
\item
  Låt andra titta på din kod. (inte på inlämningsuppgifterna)
\item
  Använd parprogrammering för att:

  \begin{itemize}
  \tightlist
  \item
    hjälpa kollegor in i projekt
  \item
    hantera komplexa programmeringsproblem
  \end{itemize}
\end{itemize}
\end{frame}

\section{R-paket}\label{r-paket}

\begin{frame}{Vad är R-paket?}
\phantomsection\label{vad-uxe4r-r-paket}
\begin{itemize}
\tightlist
\item
  R:s största styrka!
\item
  En samlig funktioner.
\item
  Många utveklare
\item
  Två huvudsakliga ``arkiv'' av paket:

  \begin{itemize}
  \tightlist
  \item
    CRAN
  \item
    GitHub
  \end{itemize}
\item
  Det är enkelt att bidra med egna paket!
\item
  Finns många riktigt bra paket.
\item
  Finns också mycket skräp.
\item
  Glöm inte att citera med \texttt{citation()}
\end{itemize}
\end{frame}

\begin{frame}{R-paket - I}
\phantomsection\label{r-paket---i}
\begin{itemize}
\tightlist
\item
  Läsa in paket görs med \texttt{library()}
\item
  Anropa funktioner utan att ladda paket görs med \texttt{::}

  \begin{itemize}
  \tightlist
  \item
    \texttt{base::mean()}
  \end{itemize}
\item
  Installera paket

  \begin{itemize}
  \tightlist
  \item
    CRAN: \texttt{install.packages()}
  \item
    GitHub: \texttt{devtools::install\_github()}
  \end{itemize}
\end{itemize}
\end{frame}

\begin{frame}{R-paket - II}
\phantomsection\label{r-paket---ii}
\begin{itemize}
\tightlist
\item
  En del paket följer med R
\item
  \texttt{ls("package:MASS")} visar vilka funktioner som finns i paketet
  MASS.
\item
  I Rstudio finns en flik för pakethantering.
\item
  Lista alla inläsa paket: \texttt{sessionInfo()}
\end{itemize}
\end{frame}

\begin{frame}{Att hitta rätt paket}
\phantomsection\label{att-hitta-ruxe4tt-paket}
\begin{itemize}
\tightlist
\item
  Alla paket håller inte samma kvalitet.
\item
  Följande tips för att se om det är ett bra paket:

  \begin{enumerate}
  \tightlist
  \item
    Kommer paketet med R eller från R Core Team?
  \item
    När kom senaste uppdateringen?
  \item
    Är paketet en utvecklingsversion?
  \item
    Sök på nätet och se om andra använder paketet och till vad.
  \item
    Mejla och fråga utvecklaren.
  \item
    Kontrollräkna centrala funktioner.
  \end{enumerate}
\item
  För att komma igång med nya paket: vignetter
\item
  På cran task view finns många paket ordnade efter ämne
\end{itemize}
\end{frame}

\begin{frame}[fragile]{Installera paket}
\phantomsection\label{installera-paket}
CRAN:

\begin{Shaded}
\begin{Highlighting}[]
\FunctionTok{install.package}\NormalTok{(}\StringTok{"lubridate"}\NormalTok{)}
\FunctionTok{install.package}\NormalTok{(}\StringTok{"devtools"}\NormalTok{)}
\end{Highlighting}
\end{Shaded}

GitHub:

\begin{Shaded}
\begin{Highlighting}[]
\FunctionTok{library}\NormalTok{(devtools)}
\FunctionTok{install\_github}\NormalTok{(}\StringTok{"ropengov/pxweb"}\NormalTok{)}
\CommentTok{\# alternativt}
\NormalTok{devtools}\SpecialCharTok{::}\FunctionTok{install\_github}\NormalTok{(}\StringTok{"ropengov/pxweb"}\NormalTok{)}
\end{Highlighting}
\end{Shaded}
\end{frame}

\begin{frame}[fragile]{Läsa in paket}
\phantomsection\label{luxe4sa-in-paket}
\begin{Shaded}
\begin{Highlighting}[]
\CommentTok{\# läsa in}
\FunctionTok{library}\NormalTok{(lubridate)}
\CommentTok{\# ta bort från aktuell session}
\FunctionTok{detach}\NormalTok{(}\StringTok{"package:lubridate"}\NormalTok{, }\AttributeTok{unload =} \ConstantTok{TRUE}\NormalTok{)}
\end{Highlighting}
\end{Shaded}
\end{frame}

\begin{frame}{Läsa in paket}
\phantomsection\label{luxe4sa-in-paket-1}
\begin{itemize}
\tightlist
\item
  I SU-salarna finns många paket redan installerade.
\item
  Kör följande i en terminal:

  \begin{itemize}
  \tightlist
  \item
    \texttt{module add courses/732G33}
  \end{itemize}
\item
  Då får ni tillgång till R, R-Studio samt många paket.
\end{itemize}
\end{frame}

\section{Dokumentation med ROxygen}\label{dokumentation-med-roxygen}

\begin{frame}{Dokumentation med ROxygen}
\phantomsection\label{dokumentation-med-roxygen-1}
\begin{itemize}
\tightlist
\item
  \texttt{roxygen2} är standard för dokumentation
\item
  Samma som JavaDoc
\item
  Skapar automatiskt \texttt{.Rd} i paket
\item
  Använder \#'
\end{itemize}
\end{frame}

\begin{frame}[fragile]{Dokumentation med ROxygen}
\phantomsection\label{dokumentation-med-roxygen-2}
\begin{longtable}[]{@{}ll@{}}
\toprule\noalign{}
ROxygendel & Innehåll \\
\midrule\noalign{}
\endhead
\texttt{@title} & Anger titel för dokumentet \\
\texttt{@description} & En beskrivning vad funktionen gör \\
\texttt{@details} & Detaljer om funktionen \\
\texttt{@param} & Argument till funktionen \\
\texttt{@return} & Vad funktionen returnerar \\
\texttt{@references} & Eventuella referenser av intresse \\
\texttt{@seealso} & Andra funktioner som kan vara aktuella \\
\texttt{@examples} & Exempel på hur funktionen kan användas \\
\bottomrule\noalign{}
\end{longtable}
\end{frame}

\begin{frame}[fragile]{Dokumentation med ROxygen - Exempel}
\phantomsection\label{dokumentation-med-roxygen---exempel}
\begin{Shaded}
\begin{Highlighting}[]
\CommentTok{\#\textquotesingle{} @title f}
\CommentTok{\#\textquotesingle{} @description}
\CommentTok{\#\textquotesingle{} En funktion som kvadrerar argumenten }
\CommentTok{\#\textquotesingle{} i x och y och summerar dem. }
\CommentTok{\#\textquotesingle{} @param x}
\CommentTok{\#\textquotesingle{} Den numeriska variabel x som ska kvadreras}
\CommentTok{\#\textquotesingle{} @param y}
\CommentTok{\#\textquotesingle{} Den numeriska variabel y som ska kvadreras}
\CommentTok{\#\textquotesingle{}}
\CommentTok{\#\textquotesingle{} @return}
\CommentTok{\#\textquotesingle{} Funktionen returnerar en numerisk vektor}
\CommentTok{\#\textquotesingle{}}
\NormalTok{f }\OtherTok{\textless{}{-}} \ControlFlowTok{function}\NormalTok{(x, y) x}\SpecialCharTok{\^{}}\DecValTok{2} \SpecialCharTok{+}\NormalTok{ y}\SpecialCharTok{\^{}}\DecValTok{2}
\end{Highlighting}
\end{Shaded}
\end{frame}

\begin{frame}{*apply-funktioner}
\phantomsection\label{apply-funktioner}
\begin{itemize}
\tightlist
\item
  ``Högnivåfunktioner''.
\item
  Ett (snabbare) alternativ till loopar.
\item
  Internt i R: loop i C-kod.
\item
  Funktioner:

  \begin{itemize}
  \tightlist
  \item
    \texttt{lapply()} : loopar över element i en lista.
  \item
    \texttt{tapply()} : loopar över ett index (ex. \texttt{aggregate()})
  \item
    \texttt{apply()} : loopar över marginaler (ex. \texttt{colSums()})
  \item
    Finns fler
  \end{itemize}
\end{itemize}
\end{frame}

\begin{frame}[fragile]{Exempel på lapply()}
\phantomsection\label{exempel-puxe5-lapply}
\begin{itemize}
\tightlist
\item
  \texttt{lapply()} har tre argument:

  \begin{itemize}
  \tightlist
  \item
    \texttt{x} listan vi vill loopa över.
  \item
    \texttt{FUN} funktionen att applicera.
  \item
    \texttt{...} argument till funktionen
  \end{itemize}
\end{itemize}

\begin{Shaded}
\begin{Highlighting}[]
\NormalTok{myList }\OtherTok{\textless{}{-}} \FunctionTok{list}\NormalTok{(}\AttributeTok{x=}\DecValTok{1}\SpecialCharTok{:}\DecValTok{10}\NormalTok{, }\AttributeTok{y =} \FunctionTok{c}\NormalTok{(}\ConstantTok{NA}\NormalTok{,}\DecValTok{12}\SpecialCharTok{:}\DecValTok{20}\NormalTok{))}
\FunctionTok{str}\NormalTok{(}\FunctionTok{lapply}\NormalTok{(}\AttributeTok{X=}\NormalTok{myList, }\AttributeTok{FUN=}\NormalTok{mean, }\AttributeTok{na.rm=}\ConstantTok{TRUE}\NormalTok{))}
\end{Highlighting}
\end{Shaded}

\begin{verbatim}
## List of 2
##  $ x: num 5.5
##  $ y: num 16
\end{verbatim}

\begin{Shaded}
\begin{Highlighting}[]
\CommentTok{\# Detta är ett test å ä ö}
\end{Highlighting}
\end{Shaded}
\end{frame}

\end{document}
