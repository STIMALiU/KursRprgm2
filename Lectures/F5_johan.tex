% Options for packages loaded elsewhere
\PassOptionsToPackage{unicode}{hyperref}
\PassOptionsToPackage{hyphens}{url}
%
\documentclass[
  11pt,
  ignorenonframetext,
  handout]{beamer}
\usepackage{pgfpages}
\setbeamertemplate{caption}[numbered]
\setbeamertemplate{caption label separator}{: }
\setbeamercolor{caption name}{fg=normal text.fg}
\beamertemplatenavigationsymbolsempty
% Prevent slide breaks in the middle of a paragraph
\widowpenalties 1 10000
\raggedbottom
\setbeamertemplate{part page}{
  \centering
  \begin{beamercolorbox}[sep=16pt,center]{part title}
    \usebeamerfont{part title}\insertpart\par
  \end{beamercolorbox}
}
\setbeamertemplate{section page}{
  \centering
  \begin{beamercolorbox}[sep=12pt,center]{part title}
    \usebeamerfont{section title}\insertsection\par
  \end{beamercolorbox}
}
\setbeamertemplate{subsection page}{
  \centering
  \begin{beamercolorbox}[sep=8pt,center]{part title}
    \usebeamerfont{subsection title}\insertsubsection\par
  \end{beamercolorbox}
}
\AtBeginPart{
  \frame{\partpage}
}
\AtBeginSection{
  \ifbibliography
  \else
    \frame{\sectionpage}
  \fi
}
\AtBeginSubsection{
  \frame{\subsectionpage}
}
\usepackage{amsmath,amssymb}
\usepackage{iftex}
\ifPDFTeX
  \usepackage[T1]{fontenc}
  \usepackage[utf8]{inputenc}
  \usepackage{textcomp} % provide euro and other symbols
\else % if luatex or xetex
  \usepackage{unicode-math} % this also loads fontspec
  \defaultfontfeatures{Scale=MatchLowercase}
  \defaultfontfeatures[\rmfamily]{Ligatures=TeX,Scale=1}
\fi
\usepackage{lmodern}
\ifPDFTeX\else
  % xetex/luatex font selection
\fi
% Use upquote if available, for straight quotes in verbatim environments
\IfFileExists{upquote.sty}{\usepackage{upquote}}{}
\IfFileExists{microtype.sty}{% use microtype if available
  \usepackage[]{microtype}
  \UseMicrotypeSet[protrusion]{basicmath} % disable protrusion for tt fonts
}{}
\makeatletter
\@ifundefined{KOMAClassName}{% if non-KOMA class
  \IfFileExists{parskip.sty}{%
    \usepackage{parskip}
  }{% else
    \setlength{\parindent}{0pt}
    \setlength{\parskip}{6pt plus 2pt minus 1pt}}
}{% if KOMA class
  \KOMAoptions{parskip=half}}
\makeatother
\usepackage{xcolor}
\newif\ifbibliography
\usepackage{color}
\usepackage{fancyvrb}
\newcommand{\VerbBar}{|}
\newcommand{\VERB}{\Verb[commandchars=\\\{\}]}
\DefineVerbatimEnvironment{Highlighting}{Verbatim}{commandchars=\\\{\}}
% Add ',fontsize=\small' for more characters per line
\usepackage{framed}
\definecolor{shadecolor}{RGB}{248,248,248}
\newenvironment{Shaded}{\begin{snugshade}}{\end{snugshade}}
\newcommand{\AlertTok}[1]{\textcolor[rgb]{0.94,0.16,0.16}{#1}}
\newcommand{\AnnotationTok}[1]{\textcolor[rgb]{0.56,0.35,0.01}{\textbf{\textit{#1}}}}
\newcommand{\AttributeTok}[1]{\textcolor[rgb]{0.13,0.29,0.53}{#1}}
\newcommand{\BaseNTok}[1]{\textcolor[rgb]{0.00,0.00,0.81}{#1}}
\newcommand{\BuiltInTok}[1]{#1}
\newcommand{\CharTok}[1]{\textcolor[rgb]{0.31,0.60,0.02}{#1}}
\newcommand{\CommentTok}[1]{\textcolor[rgb]{0.56,0.35,0.01}{\textit{#1}}}
\newcommand{\CommentVarTok}[1]{\textcolor[rgb]{0.56,0.35,0.01}{\textbf{\textit{#1}}}}
\newcommand{\ConstantTok}[1]{\textcolor[rgb]{0.56,0.35,0.01}{#1}}
\newcommand{\ControlFlowTok}[1]{\textcolor[rgb]{0.13,0.29,0.53}{\textbf{#1}}}
\newcommand{\DataTypeTok}[1]{\textcolor[rgb]{0.13,0.29,0.53}{#1}}
\newcommand{\DecValTok}[1]{\textcolor[rgb]{0.00,0.00,0.81}{#1}}
\newcommand{\DocumentationTok}[1]{\textcolor[rgb]{0.56,0.35,0.01}{\textbf{\textit{#1}}}}
\newcommand{\ErrorTok}[1]{\textcolor[rgb]{0.64,0.00,0.00}{\textbf{#1}}}
\newcommand{\ExtensionTok}[1]{#1}
\newcommand{\FloatTok}[1]{\textcolor[rgb]{0.00,0.00,0.81}{#1}}
\newcommand{\FunctionTok}[1]{\textcolor[rgb]{0.13,0.29,0.53}{\textbf{#1}}}
\newcommand{\ImportTok}[1]{#1}
\newcommand{\InformationTok}[1]{\textcolor[rgb]{0.56,0.35,0.01}{\textbf{\textit{#1}}}}
\newcommand{\KeywordTok}[1]{\textcolor[rgb]{0.13,0.29,0.53}{\textbf{#1}}}
\newcommand{\NormalTok}[1]{#1}
\newcommand{\OperatorTok}[1]{\textcolor[rgb]{0.81,0.36,0.00}{\textbf{#1}}}
\newcommand{\OtherTok}[1]{\textcolor[rgb]{0.56,0.35,0.01}{#1}}
\newcommand{\PreprocessorTok}[1]{\textcolor[rgb]{0.56,0.35,0.01}{\textit{#1}}}
\newcommand{\RegionMarkerTok}[1]{#1}
\newcommand{\SpecialCharTok}[1]{\textcolor[rgb]{0.81,0.36,0.00}{\textbf{#1}}}
\newcommand{\SpecialStringTok}[1]{\textcolor[rgb]{0.31,0.60,0.02}{#1}}
\newcommand{\StringTok}[1]{\textcolor[rgb]{0.31,0.60,0.02}{#1}}
\newcommand{\VariableTok}[1]{\textcolor[rgb]{0.00,0.00,0.00}{#1}}
\newcommand{\VerbatimStringTok}[1]{\textcolor[rgb]{0.31,0.60,0.02}{#1}}
\newcommand{\WarningTok}[1]{\textcolor[rgb]{0.56,0.35,0.01}{\textbf{\textit{#1}}}}
\usepackage{longtable,booktabs,array}
\usepackage{calc} % for calculating minipage widths
\usepackage{caption}
% Make caption package work with longtable
\makeatletter
\def\fnum@table{\tablename~\thetable}
\makeatother
\usepackage{graphicx}
\makeatletter
\def\maxwidth{\ifdim\Gin@nat@width>\linewidth\linewidth\else\Gin@nat@width\fi}
\def\maxheight{\ifdim\Gin@nat@height>\textheight\textheight\else\Gin@nat@height\fi}
\makeatother
% Scale images if necessary, so that they will not overflow the page
% margins by default, and it is still possible to overwrite the defaults
% using explicit options in \includegraphics[width, height, ...]{}
\setkeys{Gin}{width=\maxwidth,height=\maxheight,keepaspectratio}
% Set default figure placement to htbp
\makeatletter
\def\fps@figure{htbp}
\makeatother
\setlength{\emergencystretch}{3em} % prevent overfull lines
\providecommand{\tightlist}{%
  \setlength{\itemsep}{0pt}\setlength{\parskip}{0pt}}
\setcounter{secnumdepth}{-\maxdimen} % remove section numbering
\usetheme[progressbar=frametitle,block=fill]{metropolis} %numbering=none

%%% USEFUL PACKAGES
%\usepackage{showframe} % For debugging positioning
\usepackage{etex} % If too many packages
% Encoding and language
\usepackage[utf8]{inputenc}
\usepackage{babel}
\usepackage{amsmath, amssymb}
\usepackage{natbib}
%\usepackage{booktabs}
%\usepackage{algorithmic}
\usepackage{algorithm}
\usepackage{caption}
%\usepackage{animate} % Animations
\usepackage{bm} % Bold math
\usepackage{bbm}
%\usepackage{url}
%\usepackage{pifont}
%\usepackage{ulem} % Used for strikeouts \sout
%\usepackage{stackengine}
%\usepackage{enumitem}
%\setlist[description]{leftmargin=\parindent,labelindent=\parindent}
%\usepackage{colortbl} % Used for colored rows in tables


%%% GRAPHICS
\usepackage{graphicx}
\graphicspath{{./figs/}}


%%% COLORS
\setbeamercolor{background canvas}{bg=white}
\def\BlankFrame{
	\bgroup
	%\pdfpageheight 29.7cm
	\setbeamercolor{background canvas}{bg=}
	\begin{frame}[plain]
	\end{frame}
	%\makeatletter
	%\pdfpageheight \beamer@paperheight
	%\makeatother
	\egroup}

\usepackage{xcolor}
\definecolor{DarkGreen}{HTML}{00B200}
\definecolor{LightBlue}{HTML}{0090D9}
\definecolor{gold}{rgb}{.812,.710,.231}
% Text markup
%\setbeamercolor{alerted text}{fg=red}
\newcommand{\blue}[1]{\textcolor{blue}{#1}}
\newcommand{\red}[1]{\textcolor{red}{#1}}
\newcommand{\grey}[1]{\textcolor{gray}{#1}}
\newcommand{\orange}[1]{\textcolor{mLightBrown}{#1}}
\newcommand\myheading[1]{\textbf{#1}}
\newcommand\myemph[1]{\underline{\emph{#1}}}
\newcommand\textexample[1]{\textit{\textbf{#1}}}

%%% SPACING
\newcommand\vws[1][1]{\vspace{#1\baselineskip}} % vertical white space
%\newcommand\strt[1][1.5ex]{\rule[-.05\baselineskip]{0pt}{#1}} % strut
\newcommand\strt[2]{\rule[-#1ex]{0pt}{#2ex}} % strut
\newcommand\Hrule{\vspace{1ex} \hrule \vspace{1ex}} % Horisontal rule with some space after

%%% MISC
\newcommand\articleref[4]{\noindent\begin{minipage}[t]{0.04\textwidth}
		\vspace{0pt} 
		\pgfuseimage{beamericonarticle}
	\end{minipage}%
	\begin{minipage}[t]{0.96\textwidth}
		\vspace{0pt}
		#1. \textbf{#2.} \textit{#3}, #4.
	\end{minipage}}

%%% METROPOLIS THEME SPECIFIC
\makeatletter
\setlength{\metropolis@progressonsectionpage@linewidth}{1pt}
\makeatother
%\setbeamercolor{progress bar}{fg=red,bg=red!50}


%%% TEXTPOS
\usepackage[absolute,overlay]{textpos} % option showboxes is useful in draft mode
\setlength{\TPHorizModule}{\paperwidth}
\setlength{\TPVertModule}{\paperheight}
\textblockorigin{0pt}{10mm} % start everything at top-left, below gray 


%%% TIKZ/PGFPLOTS
\usepackage{tikz}
\usetikzlibrary{arrows,positioning,calc,shapes.geometric}
%\usetikzlibrary{arrows,calc,shapes.geometric,decorations.pathmorphing,backgrounds,positioning,fit,petri,decorations.pathreplacing}
%\usepackage{pgfplots}
%\pgfplotsset{compat = 1.3}


%%% BLOCKS AND BOXES
% Changing colors of blocks
%\setbeamercolor{block title alerted}{bg=UURed,fg=palette primary.fg}
%\setbeamercolor{block body alerted}{bg=UURed!15}
\setbeamercolor{block title alerted}{bg=mLightBrown,fg=palette primary.fg}
\setbeamercolor{block body alerted}{bg=mLightBrown!15}
%\setbeamercolor{block title example}{bg=UUGreen,fg=palette primary.fg}
%\setbeamercolor{block body example}{bg=UUGreen!10}
% \mybox is a rectangular box
\usepackage{boxedminipage}
\setlength\fboxrule{2pt}
\setlength\fboxsep{2\fboxsep}
\newcommand\mybox[3][\textwidth]{
  {\color{#2}
    \begin{boxedminipage}{#1}
      {\color{palette primary.bg} #3}
    \end{boxedminipage}}%
}   
\usepackage{tcolorbox}
\tcbset{arc=1mm,grow to left by=3mm,grow to right by=3mm,left=2mm}
%\newenvironment{redbox}{%
%	\begin{tcolorbox}[colback=UURed!15,colframe=UURed]}{%
%	\end{tcolorbox}}
%\newenvironment{greenbox}{%
%	\begin{tcolorbox}[colback=UUGreen!15,colframe=UUGreen]}{%
%	\end{tcolorbox}}
\newenvironment{redbox}{%
	\begin{tcolorbox}[colback=red!15,colframe=red]}{%
	\end{tcolorbox}}
\newenvironment{greenbox}{%
	\begin{tcolorbox}[colback=DarkGreen!15,colframe=DarkGreen]}{%
	\end{tcolorbox}}
\newenvironment{graybox}{%
	\begin{tcolorbox}[colback=mDarkTeal!5,colframe=mDarkTeal]}{%
	\end{tcolorbox}}
\newenvironment{orangebox}{%
\begin{tcolorbox}[colback=mLightBrown!15,colframe=mLightBrown]}{%
	\end{tcolorbox}}
\newenvironment{bwbox}{%
	\begin{tcolorbox}[colback=white,colframe=black]}{%
\end{tcolorbox}}
\newenvironment{bluebox}{%
	\begin{tcolorbox}[colback=LightBlue!15,colframe=LightBlue]}{%
\end{tcolorbox}}


%%%%%%%%% NEW MACROS

\newcommand\imp[1]{\alert{\textbf{#1}}}
\newcommand\bfit[1]{\textbf{\textit{#1}}}
\newcommand\good{\color{DarkGreen}{$\blacktriangle$}} % used in lists
\newcommand\bad{\color{red}{$\blacktriangledown$}} % used in lists

\ifLuaTeX
  \usepackage{selnolig}  % disable illegal ligatures
\fi
\usepackage{bookmark}
\IfFileExists{xurl.sty}{\usepackage{xurl}}{} % add URL line breaks if available
\urlstyle{same}
\hypersetup{
  pdftitle={R-programmering VT2024},
  pdfauthor={Johan Alenlöv},
  hidelinks,
  pdfcreator={LaTeX via pandoc}}

\title{R-programmering VT2024}
\subtitle{Föreläsning 5}
\author{Johan Alenlöv}
\date{}
\institute{Linköpings Universitet}

\begin{document}
\frame{\titlepage}

\section{Föreläsning 5}\label{fuxf6reluxe4sning-5}

\begin{frame}{Innehåll föreläsning 5}
\phantomsection\label{innehuxe5ll-fuxf6reluxe4sning-5}
\begin{itemize}
\tightlist
\item
  Information del 2
\item
  Grafik
\item
  Slumptal och simulering
\item
  knitr och markdown
\item
  Externa data och pxweb
\end{itemize}
\end{frame}

\section{Information del 2}\label{information-del-2}

\begin{frame}{R-programmering - del 2}
\phantomsection\label{r-programmering---del-2}
\begin{itemize}
\tightlist
\item
  Del 1: Grunderna i programmering:

  \begin{itemize}
  \tightlist
  \item
    Variabler och tilldelning
  \item
    Datastrukturer
  \item
    Kontrollstrukturer
  \item
    Funktioner
  \item
    Debugging
  \item
    Dokumentation
  \end{itemize}
\item
  Del 2: Tillämpningar
\end{itemize}
\end{frame}

\begin{frame}{R-programmering - del 2}
\phantomsection\label{r-programmering---del-2-1}
\begin{itemize}
\tightlist
\item
  Del 1: Grunderna i programmering:
\item
  Del 2: Tillämpningar

  \begin{itemize}
  \tightlist
  \item
    Grafik
  \item
    Slumptal
  \item
    Statistik och analys
  \item
    Externa data
  \item
    Datum
  \item
    Texthantering och regular expression
  \item
    Linjär algebra
  \item
    knitr, markdown, literate programming
  \end{itemize}
\end{itemize}
\end{frame}

\begin{frame}{R-programmering - del 2}
\phantomsection\label{r-programmering---del-2-2}
\begin{itemize}
\tightlist
\item
  Labbarna görs nu i par
\item
  Parprogrammering:

  \begin{itemize}
  \item
    \imp{Turas om att skriva koden}
  \item
    Den som inte kodar är engagerad i koden och problemet
  \item
    Byt vem som kodar var 10:e minut
  \item
    Viktigt att kommentra koden: ROxygen och inline
  \item
    \imp{Båda} är delaktiga i programmeringen
  \end{itemize}
\end{itemize}
\end{frame}

\begin{frame}{Miniprojektet}
\phantomsection\label{miniprojektet}
Info finns på kurshemsidan.

\begin{itemize}
\tightlist
\item
  Hitta data på webben (eller lägg upp eget)

  \begin{itemize}
  \tightlist
  \item
    Kommunala (tvärsnitt) data
  \item
    Tidsseriedata
  \end{itemize}
\item
  Presentera med grafik, knitr och markdown
\item
  Rapporten ska vara reproducerbar
\item
  Lämna in som \imp{PDF} och \imp{Rmd}
\item
  Miniprojektet har egen inlämning
\end{itemize}
\end{frame}

\section{Basgrafik}\label{basgrafik}

\begin{frame}{Grafik i R}
\phantomsection\label{grafik-i-r}
\begin{itemize}
\tightlist
\item
  Grafik är en styrka med R
\item
  Massa olika paket: \texttt{ggplot2}, \texttt{lattice} m.m.
\item
  Hög nivå (funktioner, \texttt{plot})
\item
  Låg nivå (bygga upp en plot steg för steg)
\item
  Använd grafik för att:

  \begin{itemize}
  \tightlist
  \item
    Sammanfatta tabeller visuellt
  \item
    Jämföra olika dataset
  \item
    Rita ut funktioner
  \end{itemize}
\end{itemize}
\end{frame}

\begin{frame}{Enkel grafik: plot()}
\phantomsection\label{enkel-grafik-plot}
\begin{itemize}
\tightlist
\item
  \texttt{plot()} - kan plotta många olika objekt

  \begin{itemize}
  \tightlist
  \item
    \texttt{plot(x,y)} - ger en scatterplot
  \item
    \texttt{plot(x)} - om \texttt{x} en \texttt{data.frame} så skapas en
    matrix-plot.
  \item
    Vanliga argument:

    \begin{itemize}
    \tightlist
    \item
      \texttt{type = } Hur plotten ska se ut
    \item
      \texttt{main = } Titel på plotten
    \item
      \texttt{xlab = } Text på x-axeln
    \item
      \texttt{ylab = } Text på y-axeln
    \item
      \texttt{xlim = } Gränserna på x-axeln
    \item
      \texttt{ylim = } Gränserna på y-axeln
    \item
      \texttt{col = } Färgerna som ska användas
    \end{itemize}
  \end{itemize}
\end{itemize}
\end{frame}

\begin{frame}[fragile]{plot() - exempel}
\phantomsection\label{plot---exempel}
\begin{Shaded}
\begin{Highlighting}[]
\FunctionTok{data}\NormalTok{(iris)}
\FunctionTok{plot}\NormalTok{(}\AttributeTok{x =}\NormalTok{ iris}\SpecialCharTok{$}\NormalTok{Sepal.Length, }
     \AttributeTok{y =}\NormalTok{ iris}\SpecialCharTok{$}\NormalTok{Petal.Length, }\AttributeTok{main =} \StringTok{"Iris"}\NormalTok{)}
\end{Highlighting}
\end{Shaded}

\includegraphics{F5_johan_files/figure-beamer/unnamed-chunk-1-1.pdf}
\end{frame}

\begin{frame}{Diagramtyper}
\phantomsection\label{diagramtyper}
\begin{itemize}
\tightlist
\item
  \texttt{hist(x = , breaks = )} ger ett histogram,

  \begin{itemize}
  \tightlist
  \item
    \texttt{x} är en numerisk vektor.
  \item
    \texttt{breaks} antalet intervall som data delas in i.
  \end{itemize}
\item
  \texttt{boxplot(x = )} ger boxplots
\item
  \texttt{barplot(height = )} ger stapeldiagram
\item
  \texttt{pie(x = )} geren piechart
\item
  Använd hjälpen för att se exempel och fler argument
\end{itemize}
\end{frame}

\begin{frame}[fragile]{hist() - exempel}
\phantomsection\label{hist---exempel}
\begin{Shaded}
\begin{Highlighting}[]
\FunctionTok{hist}\NormalTok{(iris}\SpecialCharTok{$}\NormalTok{Sepal.Length)}
\end{Highlighting}
\end{Shaded}

\includegraphics{F5_johan_files/figure-beamer/unnamed-chunk-2-1.pdf}
\end{frame}

\begin{frame}[fragile]{bokxplot() - exempel}
\phantomsection\label{bokxplot---exempel}
\begin{Shaded}
\begin{Highlighting}[]
\FunctionTok{boxplot}\NormalTok{(iris}\SpecialCharTok{$}\NormalTok{Sepal.Length)}
\end{Highlighting}
\end{Shaded}

\includegraphics{F5_johan_files/figure-beamer/unnamed-chunk-3-1.pdf}
\end{frame}

\begin{frame}[fragile]{barplot() - exempel}
\phantomsection\label{barplot---exempel}
\begin{Shaded}
\begin{Highlighting}[]
\NormalTok{tab }\OtherTok{\textless{}{-}} \FunctionTok{table}\NormalTok{(iris}\SpecialCharTok{$}\NormalTok{Species)}
\FunctionTok{barplot}\NormalTok{(tab)}
\end{Highlighting}
\end{Shaded}

\includegraphics{F5_johan_files/figure-beamer/unnamed-chunk-4-1.pdf}
\end{frame}

\begin{frame}[fragile]{pie() - exempel}
\phantomsection\label{pie---exempel}
\begin{Shaded}
\begin{Highlighting}[]
\NormalTok{tab }\OtherTok{\textless{}{-}} \FunctionTok{table}\NormalTok{(iris}\SpecialCharTok{$}\NormalTok{Species)}
\FunctionTok{pie}\NormalTok{(tab)}
\end{Highlighting}
\end{Shaded}

\includegraphics{F5_johan_files/figure-beamer/unnamed-chunk-5-1.pdf}
\end{frame}

\begin{frame}{Lågnivågrafik}
\phantomsection\label{luxe5gnivuxe5grafik}
\begin{itemize}
\tightlist
\item
  Kan användas för att bygga upp en graf från grunden
\item
  Kör lager på lager:

  \begin{itemize}
  \tightlist
  \item
    \texttt{points(x, y)} lägger till punkter
  \item
    \texttt{lines(x, y)} lägger till linjer
  \item
    \texttt{abline(a, b, h, v)} lägger till räta linjer
  \item
    \texttt{legend(x, y, legend)} lägger till en förklaringsruta
  \item
    \texttt{par()} fler grafiska alternativ
  \end{itemize}
\item
  Går att bygga upp många olika sorters plottar och grafik
\item
  Mycket smidigare att använda \texttt{ggplot2}
\end{itemize}
\end{frame}

\begin{frame}{Demo}
\phantomsection\label{demo}
\begin{block}{Demo: Grafik}
\phantomsection\label{demo-grafik}
\end{block}
\end{frame}

\section{Slumptal och simulering}\label{slumptal-och-simulering}

\begin{frame}[fragile]{Slumptal och simulering - I}
\phantomsection\label{slumptal-och-simulering---i}
\begin{itemize}
\tightlist
\item
  R har en stor uppsättning funktioner för fördelningar
\end{itemize}

\begin{longtable}[]{@{}lll@{}}
\toprule\noalign{}
prefix & Beskrivning & Exempel \\
\midrule\noalign{}
\endhead
\texttt{r} & simulera från fördelningen & \texttt{runif()} \\
\texttt{d} & täthetsfunktionen (pdf) & \texttt{dunif()} \\
\texttt{p} & kulmulativ fördelninsgfunktion (cdf) & \texttt{punif()} \\
\texttt{q} & inversa kulmulativa fördelningsfunktionen &
\texttt{qunif()} \\
\bottomrule\noalign{}
\end{longtable}

\begin{itemize}
\tightlist
\item
  Se \texttt{?Distributions} för fler fördelningar
\end{itemize}
\end{frame}

\begin{frame}{Slumptal och simulering - II}
\phantomsection\label{slumptal-och-simulering---ii}
\begin{itemize}
\tightlist
\item
  Observera att: \imp{Det finns ingen riktig slump i datorer}
\item
  Det finns slumptalsgeneratorer

  \begin{itemize}
  \tightlist
  \item
    Algoritmer där output ser slumpmässigt ut
  \end{itemize}
\item
  Kan styra slumpen genom att bestämma startvärdet, ``slumpfröet''

  \begin{itemize}
  \tightlist
  \item
    i R använder vi \texttt{set.seed( )}
  \end{itemize}
\item
  För att dra ett slumpmässigt urval använder vi \texttt{sample( )}
\end{itemize}
\end{frame}

\begin{frame}[fragile]{Slumptal och simulering - Exempel}
\phantomsection\label{slumptal-och-simulering---exempel}
\begin{Shaded}
\begin{Highlighting}[]
\FunctionTok{runif}\NormalTok{(}\AttributeTok{n =} \DecValTok{3}\NormalTok{, }\AttributeTok{min =} \SpecialCharTok{{-}}\DecValTok{1}\NormalTok{, }\AttributeTok{max =} \DecValTok{1}\NormalTok{)}
\end{Highlighting}
\end{Shaded}

\pause

\begin{verbatim}
## [1]  0.1166546 -0.1855843  0.7623537
\end{verbatim}

\pause

\begin{Shaded}
\begin{Highlighting}[]
\FunctionTok{set.seed}\NormalTok{(}\DecValTok{20220221}\NormalTok{)}
\FunctionTok{runif}\NormalTok{(}\AttributeTok{n =} \DecValTok{3}\NormalTok{, }\AttributeTok{min =} \SpecialCharTok{{-}}\DecValTok{1}\NormalTok{, }\AttributeTok{max =} \DecValTok{1}\NormalTok{)}
\end{Highlighting}
\end{Shaded}

\pause

\begin{verbatim}
## [1] -0.2819856 -0.7837770 -0.4418922
\end{verbatim}

\pause

\begin{Shaded}
\begin{Highlighting}[]
\FunctionTok{set.seed}\NormalTok{(}\DecValTok{20220221}\NormalTok{)}
\FunctionTok{runif}\NormalTok{(}\AttributeTok{n =} \DecValTok{3}\NormalTok{, }\AttributeTok{min =} \SpecialCharTok{{-}}\DecValTok{1}\NormalTok{, }\AttributeTok{max =} \DecValTok{1}\NormalTok{)}
\end{Highlighting}
\end{Shaded}

\pause

\begin{verbatim}
## [1] -0.2819856 -0.7837770 -0.4418922
\end{verbatim}
\end{frame}

\begin{frame}[fragile]{Slumptal och simulering - Exempel}
\phantomsection\label{slumptal-och-simulering---exempel-1}
\begin{Shaded}
\begin{Highlighting}[]
\NormalTok{text }\OtherTok{\textless{}{-}} \FunctionTok{c}\NormalTok{(}\StringTok{"Johan"}\NormalTok{, }\StringTok{"Josef"}\NormalTok{, }\StringTok{"Rasmus"}\NormalTok{)}
\FunctionTok{set.seed}\NormalTok{(}\DecValTok{20220221}\NormalTok{)}
\FunctionTok{sample}\NormalTok{(}\AttributeTok{x =}\NormalTok{ text, }\AttributeTok{size =} \DecValTok{5}\NormalTok{, }\AttributeTok{replace =} \ConstantTok{TRUE}\NormalTok{)}
\end{Highlighting}
\end{Shaded}

\pause

\begin{verbatim}
## [1] "Josef"  "Johan"  "Johan"  "Josef"  "Rasmus"
\end{verbatim}

\pause

\begin{Shaded}
\begin{Highlighting}[]
\FunctionTok{sample}\NormalTok{(}\AttributeTok{x =}\NormalTok{ text, }\AttributeTok{size =} \DecValTok{5}\NormalTok{, }\AttributeTok{replace =} \ConstantTok{TRUE}\NormalTok{)}
\end{Highlighting}
\end{Shaded}

\pause

\begin{verbatim}
## [1] "Rasmus" "Josef"  "Josef"  "Josef"  "Josef"
\end{verbatim}
\end{frame}

\begin{frame}{Demo}
\phantomsection\label{demo-1}
\begin{block}{Demo: Slumptal}
\phantomsection\label{demo-slumptal}
\end{block}
\end{frame}

\section{R-Markdown, knitr och
Notebooks}\label{r-markdown-knitr-och-notebooks}

\begin{frame}{R-Markdown, knitr och Notebooks}
\phantomsection\label{r-markdown-knitr-och-notebooks-1}
\begin{itemize}
\tightlist
\item
  Kombinera text, kod och grafik i en fil
\item
  Förenkla för era laborationer i denna och andra kurser
\item
  Inbyggd del av R-studio
\item
  Två delar:

  \begin{itemize}
  \tightlist
  \item
    R-Markdown (för text)
  \item
    knitR (för R-kod)
  \end{itemize}
\item
  Kan producera, PDF, Word och/eller HTML filer

  \begin{itemize}
  \tightlist
  \item
    Alla slides och kurshemsidan är skapad med detta.
  \item
    Laborationerna använder knitR med lyx istället för R-Markdown
  \end{itemize}
\end{itemize}
\end{frame}

\begin{frame}{R-Markdown}
\phantomsection\label{r-markdown}
\begin{itemize}
\tightlist
\item
  Markupspråk

  \begin{itemize}
  \tightlist
  \item
    Markup används på Teams, Forum, Discord, Slack m.m.
  \end{itemize}
\item
  Väldigt enkelt att skriva
\item
  Kan hantera matematik och matematiska formler (via LaTeX)
\item
  Integrerat med R-Studio

  \begin{itemize}
  \tightlist
  \item
    Kan behöva installera LaTeX för att skapa PDFer

    \begin{itemize}
    \tightlist
    \item
      MikTex för Windows
    \item
      MacTex för OS X
    \item
      TexLive för Linux
    \end{itemize}
  \end{itemize}
\end{itemize}
\end{frame}

\begin{frame}{knitr}
\phantomsection\label{knitr}
\begin{itemize}
\tightlist
\item
  Kör R-kod och ersätter texten med resultatet
\item
  Sätter ihop text och kod
\item
  Ger dynamiska rapporter
\item
  Skapa tabeller med \texttt{kable( )}
\item
  Kan hantera R och Python (och annat)
\end{itemize}
\end{frame}

\begin{frame}{R-Studio Notebooks}
\phantomsection\label{r-studio-notebooks}
\begin{itemize}
\tightlist
\item
  Rmd-filer (samma som R-Markdown)
\item
  Kan köras interaktivt direkt i R-Studio
\item
  Inspirerat av Jupyter Notebooks
\item
  Skillnad mellan R-Markdown och Notebooks:

  \begin{itemize}
  \tightlist
  \item
    Samma kod
  \item
    i R-Markdown körs all kod när du genererar dokumentet
  \item
    i Notebook körs en rad i taget
  \item
    i Notebooks kan du köra kod direkt och se resultatet
  \end{itemize}
\end{itemize}
\end{frame}

\begin{frame}{Demo}
\phantomsection\label{demo-2}
\begin{block}{Demo: Markdown}
\phantomsection\label{demo-markdown}
\end{block}
\end{frame}

\section{Externa data och pxweb}\label{externa-data-och-pxweb}

\begin{frame}{Externa data}
\phantomsection\label{externa-data}
\begin{itemize}
\tightlist
\item
  Mer och mer data finns på webben
\item
  Vill kunna hantera data programtiskt

  \begin{itemize}
  \tightlist
  \item
    Vill kunna säga åt koden vilken data som ska användas
  \item
    Samma data varje gång även om källmaterialet uppdateras
  \item
    Inte vara beroende av en specifik nerladdad fil
  \end{itemize}
\item
  Kan vara lite klurigt i början
\item
  Central del av reproducerbarheten i rapporter
\end{itemize}
\end{frame}

\begin{frame}{Ladda ner och läsa in från webben}
\phantomsection\label{ladda-ner-och-luxe4sa-in-fruxe5n-webben}
\begin{itemize}
\tightlist
\item
  Vill vi bara ladda ner: \texttt{downloader}
\item
  Vill vi ladda in direkt i R: \texttt{repmis}
\end{itemize}

\texttt{repmis} hanterar:

\begin{itemize}
\tightlist
\item
  \texttt{.Rdata}
\item
  \texttt{.csv}
\item
  \texttt{.txt}
\end{itemize}
\end{frame}

\begin{frame}{Vanliga källor för data}
\phantomsection\label{vanliga-kuxe4llor-fuxf6r-data}
\begin{itemize}
\tightlist
\item
  Dropbox
\item
  Google Docs
\item
  Github
\end{itemize}
\end{frame}

\begin{frame}{pxweb api}
\phantomsection\label{pxweb-api}
\begin{itemize}
\tightlist
\item
  Ett api är en dörr till ett system
\item
  \texttt{pxweb} är ett paket för att gå in genom dörren
\item
  Många myndigheter använder \texttt{pxweb} api

  \begin{itemize}
  \tightlist
  \item
    Till exempel SCB
  \end{itemize}
\item
  Kan användas för:

  \begin{itemize}
  \tightlist
  \item
    Navigera i datalager
  \item
    Ladda ner förutbestämd data med kod
  \end{itemize}
\end{itemize}
\end{frame}

\begin{frame}[fragile]{pxweb}
\phantomsection\label{pxweb}
\begin{Shaded}
\begin{Highlighting}[]
\FunctionTok{install.packages}\NormalTok{(}\StringTok{"pxweb"}\NormalTok{)}
\FunctionTok{library}\NormalTok{(pxweb)}

\NormalTok{min\_data }\OtherTok{\textless{}{-}} \FunctionTok{pxweb\_interactive}\NormalTok{()}
\end{Highlighting}
\end{Shaded}
\end{frame}

\begin{frame}{Demo}
\phantomsection\label{demo-3}
\begin{block}{Demo: PXWEB}
\phantomsection\label{demo-pxweb}
\end{block}
\end{frame}

\end{document}
