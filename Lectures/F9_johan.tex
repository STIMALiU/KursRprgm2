% Options for packages loaded elsewhere
\PassOptionsToPackage{unicode}{hyperref}
\PassOptionsToPackage{hyphens}{url}
%
\documentclass[
  11pt,
  ignorenonframetext,
  handout]{beamer}
\usepackage{pgfpages}
\setbeamertemplate{caption}[numbered]
\setbeamertemplate{caption label separator}{: }
\setbeamercolor{caption name}{fg=normal text.fg}
\beamertemplatenavigationsymbolsempty
% Prevent slide breaks in the middle of a paragraph
\widowpenalties 1 10000
\raggedbottom
\setbeamertemplate{part page}{
  \centering
  \begin{beamercolorbox}[sep=16pt,center]{part title}
    \usebeamerfont{part title}\insertpart\par
  \end{beamercolorbox}
}
\setbeamertemplate{section page}{
  \centering
  \begin{beamercolorbox}[sep=12pt,center]{part title}
    \usebeamerfont{section title}\insertsection\par
  \end{beamercolorbox}
}
\setbeamertemplate{subsection page}{
  \centering
  \begin{beamercolorbox}[sep=8pt,center]{part title}
    \usebeamerfont{subsection title}\insertsubsection\par
  \end{beamercolorbox}
}
\AtBeginPart{
  \frame{\partpage}
}
\AtBeginSection{
  \ifbibliography
  \else
    \frame{\sectionpage}
  \fi
}
\AtBeginSubsection{
  \frame{\subsectionpage}
}
\usepackage{amsmath,amssymb}
\usepackage{lmodern}
\usepackage{iftex}
\ifPDFTeX
  \usepackage[T1]{fontenc}
  \usepackage[utf8]{inputenc}
  \usepackage{textcomp} % provide euro and other symbols
\else % if luatex or xetex
  \usepackage{unicode-math}
  \defaultfontfeatures{Scale=MatchLowercase}
  \defaultfontfeatures[\rmfamily]{Ligatures=TeX,Scale=1}
\fi
% Use upquote if available, for straight quotes in verbatim environments
\IfFileExists{upquote.sty}{\usepackage{upquote}}{}
\IfFileExists{microtype.sty}{% use microtype if available
  \usepackage[]{microtype}
  \UseMicrotypeSet[protrusion]{basicmath} % disable protrusion for tt fonts
}{}
\makeatletter
\@ifundefined{KOMAClassName}{% if non-KOMA class
  \IfFileExists{parskip.sty}{%
    \usepackage{parskip}
  }{% else
    \setlength{\parindent}{0pt}
    \setlength{\parskip}{6pt plus 2pt minus 1pt}}
}{% if KOMA class
  \KOMAoptions{parskip=half}}
\makeatother
\usepackage{xcolor}
\IfFileExists{xurl.sty}{\usepackage{xurl}}{} % add URL line breaks if available
\IfFileExists{bookmark.sty}{\usepackage{bookmark}}{\usepackage{hyperref}}
\hypersetup{
  pdftitle={R-programmering VT2022},
  pdfauthor={Johan Alenlöv},
  hidelinks,
  pdfcreator={LaTeX via pandoc}}
\urlstyle{same} % disable monospaced font for URLs
\newif\ifbibliography
\setlength{\emergencystretch}{3em} % prevent overfull lines
\providecommand{\tightlist}{%
  \setlength{\itemsep}{0pt}\setlength{\parskip}{0pt}}
\setcounter{secnumdepth}{-\maxdimen} % remove section numbering
\usetheme[progressbar=frametitle,block=fill]{metropolis} %numbering=none

%%% USEFUL PACKAGES
%\usepackage{showframe} % For debugging positioning
\usepackage{etex} % If too many packages
% Encoding and language
\usepackage[utf8]{inputenc}
\usepackage{babel}
\usepackage{amsmath, amssymb}
\usepackage{natbib}
%\usepackage{booktabs}
%\usepackage{algorithmic}
\usepackage{algorithm}
\usepackage{caption}
%\usepackage{animate} % Animations
\usepackage{bm} % Bold math
\usepackage{bbm}
%\usepackage{url}
%\usepackage{pifont}
%\usepackage{ulem} % Used for strikeouts \sout
%\usepackage{stackengine}
%\usepackage{enumitem}
%\setlist[description]{leftmargin=\parindent,labelindent=\parindent}
%\usepackage{colortbl} % Used for colored rows in tables


%%% GRAPHICS
\usepackage{graphicx}
\graphicspath{{./figs/}}


%%% COLORS
\setbeamercolor{background canvas}{bg=white}
\def\BlankFrame{
	\bgroup
	%\pdfpageheight 29.7cm
	\setbeamercolor{background canvas}{bg=}
	\begin{frame}[plain]
	\end{frame}
	%\makeatletter
	%\pdfpageheight \beamer@paperheight
	%\makeatother
	\egroup}

\usepackage{xcolor}
\definecolor{DarkGreen}{HTML}{00B200}
\definecolor{LightBlue}{HTML}{0090D9}
\definecolor{gold}{rgb}{.812,.710,.231}
% Text markup
%\setbeamercolor{alerted text}{fg=red}
\newcommand{\blue}[1]{\textcolor{blue}{#1}}
\newcommand{\red}[1]{\textcolor{red}{#1}}
\newcommand{\grey}[1]{\textcolor{gray}{#1}}
\newcommand{\orange}[1]{\textcolor{mLightBrown}{#1}}
\newcommand\myheading[1]{\textbf{#1}}
\newcommand\myemph[1]{\underline{\emph{#1}}}
\newcommand\textexample[1]{\textit{\textbf{#1}}}

%%% SPACING
\newcommand\vws[1][1]{\vspace{#1\baselineskip}} % vertical white space
%\newcommand\strt[1][1.5ex]{\rule[-.05\baselineskip]{0pt}{#1}} % strut
\newcommand\strt[2]{\rule[-#1ex]{0pt}{#2ex}} % strut
\newcommand\Hrule{\vspace{1ex} \hrule \vspace{1ex}} % Horisontal rule with some space after

%%% MISC
\newcommand\articleref[4]{\noindent\begin{minipage}[t]{0.04\textwidth}
		\vspace{0pt} 
		\pgfuseimage{beamericonarticle}
	\end{minipage}%
	\begin{minipage}[t]{0.96\textwidth}
		\vspace{0pt}
		#1. \textbf{#2.} \textit{#3}, #4.
	\end{minipage}}

%%% METROPOLIS THEME SPECIFIC
\makeatletter
\setlength{\metropolis@progressonsectionpage@linewidth}{1pt}
\makeatother
%\setbeamercolor{progress bar}{fg=red,bg=red!50}


%%% TEXTPOS
\usepackage[absolute,overlay]{textpos} % option showboxes is useful in draft mode
\setlength{\TPHorizModule}{\paperwidth}
\setlength{\TPVertModule}{\paperheight}
\textblockorigin{0pt}{10mm} % start everything at top-left, below gray 


%%% TIKZ/PGFPLOTS
\usepackage{tikz}
\usetikzlibrary{arrows,positioning,calc,shapes.geometric}
%\usetikzlibrary{arrows,calc,shapes.geometric,decorations.pathmorphing,backgrounds,positioning,fit,petri,decorations.pathreplacing}
%\usepackage{pgfplots}
%\pgfplotsset{compat = 1.3}


%%% BLOCKS AND BOXES
% Changing colors of blocks
%\setbeamercolor{block title alerted}{bg=UURed,fg=palette primary.fg}
%\setbeamercolor{block body alerted}{bg=UURed!15}
\setbeamercolor{block title alerted}{bg=mLightBrown,fg=palette primary.fg}
\setbeamercolor{block body alerted}{bg=mLightBrown!15}
%\setbeamercolor{block title example}{bg=UUGreen,fg=palette primary.fg}
%\setbeamercolor{block body example}{bg=UUGreen!10}
% \mybox is a rectangular box
\usepackage{boxedminipage}
\setlength\fboxrule{2pt}
\setlength\fboxsep{2\fboxsep}
\newcommand\mybox[3][\textwidth]{
  {\color{#2}
    \begin{boxedminipage}{#1}
      {\color{palette primary.bg} #3}
    \end{boxedminipage}}%
}   
\usepackage{tcolorbox}
\tcbset{arc=1mm,grow to left by=3mm,grow to right by=3mm,left=2mm}
%\newenvironment{redbox}{%
%	\begin{tcolorbox}[colback=UURed!15,colframe=UURed]}{%
%	\end{tcolorbox}}
%\newenvironment{greenbox}{%
%	\begin{tcolorbox}[colback=UUGreen!15,colframe=UUGreen]}{%
%	\end{tcolorbox}}
\newenvironment{redbox}{%
	\begin{tcolorbox}[colback=red!15,colframe=red]}{%
	\end{tcolorbox}}
\newenvironment{greenbox}{%
	\begin{tcolorbox}[colback=DarkGreen!15,colframe=DarkGreen]}{%
	\end{tcolorbox}}
\newenvironment{graybox}{%
	\begin{tcolorbox}[colback=mDarkTeal!5,colframe=mDarkTeal]}{%
	\end{tcolorbox}}
\newenvironment{orangebox}{%
\begin{tcolorbox}[colback=mLightBrown!15,colframe=mLightBrown]}{%
	\end{tcolorbox}}
\newenvironment{bwbox}{%
	\begin{tcolorbox}[colback=white,colframe=black]}{%
\end{tcolorbox}}
\newenvironment{bluebox}{%
	\begin{tcolorbox}[colback=LightBlue!15,colframe=LightBlue]}{%
\end{tcolorbox}}


%%%%%%%%% NEW MACROS

\newcommand\imp[1]{\alert{\textbf{#1}}}
\newcommand\bfit[1]{\textbf{\textit{#1}}}
\newcommand\good{\color{DarkGreen}{$\blacktriangle$}} % used in lists
\newcommand\bad{\color{red}{$\blacktriangledown$}} % used in lists

\ifLuaTeX
  \usepackage{selnolig}  % disable illegal ligatures
\fi

\title{R-programmering VT2022}
\subtitle{Föreläsning 9}
\author{Johan Alenlöv}
\date{2022-03-21}
\institute{Linköpings Universitet}

\begin{document}
\frame{\titlepage}

\hypertarget{fuxf6reluxe4sning-9}{%
\section{Föreläsning 9}\label{fuxf6reluxe4sning-9}}

\begin{frame}{Innehåll föreläsning 9}
\protect\hypertarget{innehuxe5ll-fuxf6reluxe4sning-9}{}
\begin{itemize}
\tightlist
\item
  Kompletteringar
\item
  Tentainfo
\item
  R i andra kurser
\item
  Gammal tenta
\end{itemize}
\end{frame}

\begin{frame}[fragile]{Kompletteringar}
\protect\hypertarget{kompletteringar}{}
\begin{itemize}
\tightlist
\item
  Deadline för kompletteringar finns på kurshemsidan

  \begin{itemize}
  \tightlist
  \item
    Nästa komplettering 2022-04-15
  \end{itemize}
\item
  Underkända labbar markeras med \texttt{komplettering} och
  kompletteringar lämnas in på samma lab.
\item
  Alla labbar måste vara godkända efter sista kompletteringstillfället.

  \begin{itemize}
  \tightlist
  \item
    Annars måste man göra om \textbf{alla} labbar nästa kursomgång.
  \item
    Vänta inte till sista kompletteringen.
  \end{itemize}
\end{itemize}
\end{frame}

\begin{frame}{Kursutvärdering}
\protect\hypertarget{kursutvuxe4rdering}{}
\begin{itemize}
\tightlist
\item
  Efter tentan kommer det skickas ut en kursutvärdering.
\item
  Svara på den, kom gärna med kreativ kritik om vilka delar som kan bli
  förbättrade.
\item
  Skriv även vad som varit och fungerat bra.
\end{itemize}
\end{frame}

\begin{frame}{Tentainfo}
\protect\hypertarget{tentainfo}{}
\begin{itemize}
\tightlist
\item
  Tentamen är \textbf{31/3 kl 8:00 -- 12:00}.
\item
  Skrivplats är en SU-sal.
\item
  Endast anmälda studenter får skriva tentan!
\item
  Tentan kommer att likna de gamla tentorna som finns på kurshemsidan.
\item
  Hjälpmedel kommer vara ett antal ``cheatsheets'', de kommer finnas som
  pdf, ni ska \textbf{INTE} ta med några papper.
\item
  Inlämning är en R-fil per uppgift.
\item
  \textbf{testsession} 22/3 8-10 i SU-17/18.
\end{itemize}
\end{frame}

\begin{frame}{R i andra kurser}
\protect\hypertarget{r-i-andra-kurser}{}
Årskurs 1 och 2:

\begin{itemize}
\tightlist
\item
  Grundläggande tidsserieanalys
\item
  Regressions- och variansanalys
\item
  Databaser: design och programmering
\end{itemize}

Årskurs 3:

\begin{itemize}
\tightlist
\item
  Bayesiansk statistik
\item
  Statistisk analys av komplexa data
\item
  Data-Mining
\end{itemize}
\end{frame}

\begin{frame}{R i andra kurser}
\protect\hypertarget{r-i-andra-kurser-1}{}
Kan användas i:

\begin{itemize}
\tightlist
\item
  Projektarbete i statistik
\item
  Kandidatuppsats
\end{itemize}

Masterprogrammet:

\begin{itemize}
\tightlist
\item
  Advanced Programming in R
\end{itemize}
\end{frame}

\begin{frame}{R i framtiden}
\protect\hypertarget{r-i-framtiden}{}
\begin{itemize}
\tightlist
\item
  Programmering är en färskvara
\item
  Viktigt att programmera regelbundet

  \begin{itemize}
  \tightlist
  \item
    Finns många programmeringspussel på nätet

    \begin{itemize}
    \tightlist
    \item
      Project Euler
    \item
      Advent of Code
    \end{itemize}
  \end{itemize}
\item
  Se ``Efter kursen'' på kurshemsidan
\end{itemize}
\end{frame}

\begin{frame}{Gammal tenta}
\protect\hypertarget{gammal-tenta}{}
Tentan ``732G33\_732G83\_exam\_2020-03-27.pdf''

Vad har vi lärt oss i kursen?

\begin{itemize}
\tightlist
\item
  Variabler, tilldelning, typer
\item
  Datastrukturer
\item
  Kontrollstrukturer
\item
  Funktioner, miljöer, objekt
\item
  Tillämpningar:

  \begin{itemize}
  \tightlist
  \item
    Grafik
  \item
    Statistik
  \item
    Texthantering
  \item
    Datum
  \item
    \ldots.
  \end{itemize}
\end{itemize}
\end{frame}

\end{document}
