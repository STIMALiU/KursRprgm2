% Options for packages loaded elsewhere
\PassOptionsToPackage{unicode}{hyperref}
\PassOptionsToPackage{hyphens}{url}
%
\documentclass[
  11pt,
  ignorenonframetext,
  handout]{beamer}
\usepackage{pgfpages}
\setbeamertemplate{caption}[numbered]
\setbeamertemplate{caption label separator}{: }
\setbeamercolor{caption name}{fg=normal text.fg}
\beamertemplatenavigationsymbolsempty
% Prevent slide breaks in the middle of a paragraph
\widowpenalties 1 10000
\raggedbottom
\setbeamertemplate{part page}{
  \centering
  \begin{beamercolorbox}[sep=16pt,center]{part title}
    \usebeamerfont{part title}\insertpart\par
  \end{beamercolorbox}
}
\setbeamertemplate{section page}{
  \centering
  \begin{beamercolorbox}[sep=12pt,center]{part title}
    \usebeamerfont{section title}\insertsection\par
  \end{beamercolorbox}
}
\setbeamertemplate{subsection page}{
  \centering
  \begin{beamercolorbox}[sep=8pt,center]{part title}
    \usebeamerfont{subsection title}\insertsubsection\par
  \end{beamercolorbox}
}
\AtBeginPart{
  \frame{\partpage}
}
\AtBeginSection{
  \ifbibliography
  \else
    \frame{\sectionpage}
  \fi
}
\AtBeginSubsection{
  \frame{\subsectionpage}
}
\usepackage{amsmath,amssymb}
\usepackage{iftex}
\ifPDFTeX
  \usepackage[T1]{fontenc}
  \usepackage[utf8]{inputenc}
  \usepackage{textcomp} % provide euro and other symbols
\else % if luatex or xetex
  \usepackage{unicode-math} % this also loads fontspec
  \defaultfontfeatures{Scale=MatchLowercase}
  \defaultfontfeatures[\rmfamily]{Ligatures=TeX,Scale=1}
\fi
\usepackage{lmodern}
\ifPDFTeX\else
  % xetex/luatex font selection
\fi
% Use upquote if available, for straight quotes in verbatim environments
\IfFileExists{upquote.sty}{\usepackage{upquote}}{}
\IfFileExists{microtype.sty}{% use microtype if available
  \usepackage[]{microtype}
  \UseMicrotypeSet[protrusion]{basicmath} % disable protrusion for tt fonts
}{}
\makeatletter
\@ifundefined{KOMAClassName}{% if non-KOMA class
  \IfFileExists{parskip.sty}{%
    \usepackage{parskip}
  }{% else
    \setlength{\parindent}{0pt}
    \setlength{\parskip}{6pt plus 2pt minus 1pt}}
}{% if KOMA class
  \KOMAoptions{parskip=half}}
\makeatother
\usepackage{xcolor}
\newif\ifbibliography
\usepackage{color}
\usepackage{fancyvrb}
\newcommand{\VerbBar}{|}
\newcommand{\VERB}{\Verb[commandchars=\\\{\}]}
\DefineVerbatimEnvironment{Highlighting}{Verbatim}{commandchars=\\\{\}}
% Add ',fontsize=\small' for more characters per line
\usepackage{framed}
\definecolor{shadecolor}{RGB}{248,248,248}
\newenvironment{Shaded}{\begin{snugshade}}{\end{snugshade}}
\newcommand{\AlertTok}[1]{\textcolor[rgb]{0.94,0.16,0.16}{#1}}
\newcommand{\AnnotationTok}[1]{\textcolor[rgb]{0.56,0.35,0.01}{\textbf{\textit{#1}}}}
\newcommand{\AttributeTok}[1]{\textcolor[rgb]{0.13,0.29,0.53}{#1}}
\newcommand{\BaseNTok}[1]{\textcolor[rgb]{0.00,0.00,0.81}{#1}}
\newcommand{\BuiltInTok}[1]{#1}
\newcommand{\CharTok}[1]{\textcolor[rgb]{0.31,0.60,0.02}{#1}}
\newcommand{\CommentTok}[1]{\textcolor[rgb]{0.56,0.35,0.01}{\textit{#1}}}
\newcommand{\CommentVarTok}[1]{\textcolor[rgb]{0.56,0.35,0.01}{\textbf{\textit{#1}}}}
\newcommand{\ConstantTok}[1]{\textcolor[rgb]{0.56,0.35,0.01}{#1}}
\newcommand{\ControlFlowTok}[1]{\textcolor[rgb]{0.13,0.29,0.53}{\textbf{#1}}}
\newcommand{\DataTypeTok}[1]{\textcolor[rgb]{0.13,0.29,0.53}{#1}}
\newcommand{\DecValTok}[1]{\textcolor[rgb]{0.00,0.00,0.81}{#1}}
\newcommand{\DocumentationTok}[1]{\textcolor[rgb]{0.56,0.35,0.01}{\textbf{\textit{#1}}}}
\newcommand{\ErrorTok}[1]{\textcolor[rgb]{0.64,0.00,0.00}{\textbf{#1}}}
\newcommand{\ExtensionTok}[1]{#1}
\newcommand{\FloatTok}[1]{\textcolor[rgb]{0.00,0.00,0.81}{#1}}
\newcommand{\FunctionTok}[1]{\textcolor[rgb]{0.13,0.29,0.53}{\textbf{#1}}}
\newcommand{\ImportTok}[1]{#1}
\newcommand{\InformationTok}[1]{\textcolor[rgb]{0.56,0.35,0.01}{\textbf{\textit{#1}}}}
\newcommand{\KeywordTok}[1]{\textcolor[rgb]{0.13,0.29,0.53}{\textbf{#1}}}
\newcommand{\NormalTok}[1]{#1}
\newcommand{\OperatorTok}[1]{\textcolor[rgb]{0.81,0.36,0.00}{\textbf{#1}}}
\newcommand{\OtherTok}[1]{\textcolor[rgb]{0.56,0.35,0.01}{#1}}
\newcommand{\PreprocessorTok}[1]{\textcolor[rgb]{0.56,0.35,0.01}{\textit{#1}}}
\newcommand{\RegionMarkerTok}[1]{#1}
\newcommand{\SpecialCharTok}[1]{\textcolor[rgb]{0.81,0.36,0.00}{\textbf{#1}}}
\newcommand{\SpecialStringTok}[1]{\textcolor[rgb]{0.31,0.60,0.02}{#1}}
\newcommand{\StringTok}[1]{\textcolor[rgb]{0.31,0.60,0.02}{#1}}
\newcommand{\VariableTok}[1]{\textcolor[rgb]{0.00,0.00,0.00}{#1}}
\newcommand{\VerbatimStringTok}[1]{\textcolor[rgb]{0.31,0.60,0.02}{#1}}
\newcommand{\WarningTok}[1]{\textcolor[rgb]{0.56,0.35,0.01}{\textbf{\textit{#1}}}}
\usepackage{longtable,booktabs,array}
\usepackage{calc} % for calculating minipage widths
\usepackage{caption}
% Make caption package work with longtable
\makeatletter
\def\fnum@table{\tablename~\thetable}
\makeatother
\setlength{\emergencystretch}{3em} % prevent overfull lines
\providecommand{\tightlist}{%
  \setlength{\itemsep}{0pt}\setlength{\parskip}{0pt}}
\setcounter{secnumdepth}{-\maxdimen} % remove section numbering
\usetheme[progressbar=frametitle,block=fill]{metropolis} %numbering=none

%%% USEFUL PACKAGES
%\usepackage{showframe} % For debugging positioning
\usepackage{etex} % If too many packages
% Encoding and language
\usepackage[utf8]{inputenc}
\usepackage{babel}
\usepackage{amsmath, amssymb}
\usepackage{natbib}
%\usepackage{booktabs}
%\usepackage{algorithmic}
\usepackage{algorithm}
\usepackage{caption}
%\usepackage{animate} % Animations
\usepackage{bm} % Bold math
\usepackage{bbm}
%\usepackage{url}
%\usepackage{pifont}
%\usepackage{ulem} % Used for strikeouts \sout
%\usepackage{stackengine}
%\usepackage{enumitem}
%\setlist[description]{leftmargin=\parindent,labelindent=\parindent}
%\usepackage{colortbl} % Used for colored rows in tables


%%% GRAPHICS
\usepackage{graphicx}
\graphicspath{{./figs/}}


%%% COLORS
\setbeamercolor{background canvas}{bg=white}
\def\BlankFrame{
	\bgroup
	%\pdfpageheight 29.7cm
	\setbeamercolor{background canvas}{bg=}
	\begin{frame}[plain]
	\end{frame}
	%\makeatletter
	%\pdfpageheight \beamer@paperheight
	%\makeatother
	\egroup}

\usepackage{xcolor}
\definecolor{DarkGreen}{HTML}{00B200}
\definecolor{LightBlue}{HTML}{0090D9}
\definecolor{gold}{rgb}{.812,.710,.231}
% Text markup
%\setbeamercolor{alerted text}{fg=red}
\newcommand{\blue}[1]{\textcolor{blue}{#1}}
\newcommand{\red}[1]{\textcolor{red}{#1}}
\newcommand{\grey}[1]{\textcolor{gray}{#1}}
\newcommand{\orange}[1]{\textcolor{mLightBrown}{#1}}
\newcommand\myheading[1]{\textbf{#1}}
\newcommand\myemph[1]{\underline{\emph{#1}}}
\newcommand\textexample[1]{\textit{\textbf{#1}}}

%%% SPACING
\newcommand\vws[1][1]{\vspace{#1\baselineskip}} % vertical white space
%\newcommand\strt[1][1.5ex]{\rule[-.05\baselineskip]{0pt}{#1}} % strut
\newcommand\strt[2]{\rule[-#1ex]{0pt}{#2ex}} % strut
\newcommand\Hrule{\vspace{1ex} \hrule \vspace{1ex}} % Horisontal rule with some space after

%%% MISC
\newcommand\articleref[4]{\noindent\begin{minipage}[t]{0.04\textwidth}
		\vspace{0pt} 
		\pgfuseimage{beamericonarticle}
	\end{minipage}%
	\begin{minipage}[t]{0.96\textwidth}
		\vspace{0pt}
		#1. \textbf{#2.} \textit{#3}, #4.
	\end{minipage}}

%%% METROPOLIS THEME SPECIFIC
\makeatletter
\setlength{\metropolis@progressonsectionpage@linewidth}{1pt}
\makeatother
%\setbeamercolor{progress bar}{fg=red,bg=red!50}


%%% TEXTPOS
\usepackage[absolute,overlay]{textpos} % option showboxes is useful in draft mode
\setlength{\TPHorizModule}{\paperwidth}
\setlength{\TPVertModule}{\paperheight}
\textblockorigin{0pt}{10mm} % start everything at top-left, below gray 


%%% TIKZ/PGFPLOTS
\usepackage{tikz}
\usetikzlibrary{arrows,positioning,calc,shapes.geometric}
%\usetikzlibrary{arrows,calc,shapes.geometric,decorations.pathmorphing,backgrounds,positioning,fit,petri,decorations.pathreplacing}
%\usepackage{pgfplots}
%\pgfplotsset{compat = 1.3}


%%% BLOCKS AND BOXES
% Changing colors of blocks
%\setbeamercolor{block title alerted}{bg=UURed,fg=palette primary.fg}
%\setbeamercolor{block body alerted}{bg=UURed!15}
\setbeamercolor{block title alerted}{bg=mLightBrown,fg=palette primary.fg}
\setbeamercolor{block body alerted}{bg=mLightBrown!15}
%\setbeamercolor{block title example}{bg=UUGreen,fg=palette primary.fg}
%\setbeamercolor{block body example}{bg=UUGreen!10}
% \mybox is a rectangular box
\usepackage{boxedminipage}
\setlength\fboxrule{2pt}
\setlength\fboxsep{2\fboxsep}
\newcommand\mybox[3][\textwidth]{
  {\color{#2}
    \begin{boxedminipage}{#1}
      {\color{palette primary.bg} #3}
    \end{boxedminipage}}%
}   
\usepackage{tcolorbox}
\tcbset{arc=1mm,grow to left by=3mm,grow to right by=3mm,left=2mm}
%\newenvironment{redbox}{%
%	\begin{tcolorbox}[colback=UURed!15,colframe=UURed]}{%
%	\end{tcolorbox}}
%\newenvironment{greenbox}{%
%	\begin{tcolorbox}[colback=UUGreen!15,colframe=UUGreen]}{%
%	\end{tcolorbox}}
\newenvironment{redbox}{%
	\begin{tcolorbox}[colback=red!15,colframe=red]}{%
	\end{tcolorbox}}
\newenvironment{greenbox}{%
	\begin{tcolorbox}[colback=DarkGreen!15,colframe=DarkGreen]}{%
	\end{tcolorbox}}
\newenvironment{graybox}{%
	\begin{tcolorbox}[colback=mDarkTeal!5,colframe=mDarkTeal]}{%
	\end{tcolorbox}}
\newenvironment{orangebox}{%
\begin{tcolorbox}[colback=mLightBrown!15,colframe=mLightBrown]}{%
	\end{tcolorbox}}
\newenvironment{bwbox}{%
	\begin{tcolorbox}[colback=white,colframe=black]}{%
\end{tcolorbox}}
\newenvironment{bluebox}{%
	\begin{tcolorbox}[colback=LightBlue!15,colframe=LightBlue]}{%
\end{tcolorbox}}


%%%%%%%%% NEW MACROS

\newcommand\imp[1]{\alert{\textbf{#1}}}
\newcommand\bfit[1]{\textbf{\textit{#1}}}
\newcommand\good{\color{DarkGreen}{$\blacktriangle$}} % used in lists
\newcommand\bad{\color{red}{$\blacktriangledown$}} % used in lists

\ifLuaTeX
  \usepackage{selnolig}  % disable illegal ligatures
\fi
\usepackage{bookmark}
\IfFileExists{xurl.sty}{\usepackage{xurl}}{} % add URL line breaks if available
\urlstyle{same}
\hypersetup{
  pdftitle={R-programmering VT2024},
  pdfauthor={Josef Wilzén},
  hidelinks,
  pdfcreator={LaTeX via pandoc}}

\title{R-programmering VT2024}
\subtitle{Föreläsning 6}
\author{Josef Wilzén}
\date{}
\institute{Linköpings Universitet}

\begin{document}
\frame{\titlepage}

\begin{frame}{Innehåll föreläsning 6}
\phantomsection\label{innehuxe5ll-fuxf6reluxe4sning-6}
\begin{itemize}
\tightlist
\item
  Information
\item
  Programmeringsparadigm

  \begin{itemize}
  \tightlist
  \item
    Funktionell programmering
  \item
    Objektorientering
  \end{itemize}
\item
  Datum och tid med \texttt{lubridate}
\item
  Linjär algebra i R
\end{itemize}
\end{frame}

\section{Information}\label{information}

\begin{frame}{Information}
\phantomsection\label{information-1}
\begin{itemize}
\tightlist
\item Anmäl er till grupper för inlämningar på del 2. Redovisning endast tillåten om man är med i en grupp.
\item
  Kom ihåg att anmäla er till datortentan Anmälan är öppen: 2025-05-12 - 2025-06-01

  \begin{itemize}
  \tightlist
  \item
    Endast anmälda får skriva tentan
  \end{itemize}
\item
  Datortentan kommer att vara på plats

  \begin{itemize}
  \tightlist
  \item
    Tenta i SU-sal
  \item
    Skriva R-kod (funktioner likt inlämningar)

    \begin{itemize}
    \tightlist
    \item
      Lämnas in som .R fil
    \end{itemize}
  \item
    Hjälpmedel finns på Lisam
  \end{itemize}
\end{itemize}
\end{frame}

\section{Programmeringsparadigm}\label{programmeringsparadigm}

\begin{frame}{Programmeringsparadigm}
\phantomsection\label{programmeringsparadigm-1}
\begin{itemize}
\item
  Finns många olika sätt att se på \imp{hur} program ska skrivas.
\item
  Olika programmeringsspråk följer olika paradigmer.
\item
  Pratar ofta om två för R:

  \begin{itemize}
  \tightlist
  \item
    Funktionell programmering
  \item
    Objektorienterad programmering
  \end{itemize}
\end{itemize}
\end{frame}

\begin{frame}{Funktionell programmering}
\phantomsection\label{funktionell-programmering}
\begin{itemize}
\tightlist
\item
  I funktionell programmering pratar man om två saker.

  \begin{itemize}
  \tightlist
  \item
    ``Rena'' funktioner.

    \begin{itemize}
    \tightlist
    \item
      Output beror bara på inargumenten.
    \item
      Funktionen har inga sidoeffekter.
    \end{itemize}
  \item
    ``First-class functions''.

    \begin{itemize}
    \tightlist
    \item
      Funktioner ska bete sig som alla andra objekt
    \item
      Kunna spara funktioner, returnera funktioner etc.
    \end{itemize}
  \end{itemize}
\item
  Exempel:

  \begin{itemize}
  \tightlist
  \item
    Scala, Erland, Haskell
  \end{itemize}
\end{itemize}
\end{frame}

\begin{frame}{Objektorientering}
\phantomsection\label{objektorientering}
\begin{itemize}
\tightlist
\item
  Objektorientering åtgår från klasser:

  \begin{itemize}
  \tightlist
  \item
    En klass är en mall som används för att skapa objekt
  \item
    Kopplar samman objekt med funktioner
  \item
    Objekt är en specifik realisering av en klass
  \item
    Objekt skapas med en konstruktor
  \end{itemize}
\item
  Viktiga koncept:

  \begin{itemize}
  \tightlist
  \item
    \textbf{Inkapsling}: Gömma och ordna relaterad data.
  \item
    \textbf{Polymorfism}: Generiska funktioner som hanterar olika
    objekt.
  \item
    \textbf{Arv}: Underlättar specialisering av data och metoder via
    underklasser.
  \end{itemize}
\end{itemize}
\end{frame}

\begin{frame}{Objektorientering}
\phantomsection\label{objektorientering-1}
\begin{itemize}
\tightlist
\item
  i R:

  \begin{itemize}
  \tightlist
  \item
    Alla objekt har en klass
  \item
    Kan undersöka ett objekts klass med \texttt{class( )}
  \item
    Variabeltyper är ``atomära'' klasser
  \item
    Klasser har en konstruktor
  \end{itemize}
\item
  Olika typer av klasser i R:

  \begin{itemize}
  \tightlist
  \item
    Basklasser
  \item
    S3 (informellt): ``lista med klassattribut''
  \item
    S4 (formellt): element väljs ut med @
  \item
    Reference classes
  \item
    S7 (nytt system): "blanding" av S3 och S4
  \end{itemize}
\end{itemize}
\end{frame}

\begin{frame}{Generiska funktioner (S3)}
\phantomsection\label{generiska-funktioner-s3}
\begin{itemize}
\tightlist
\item
  Objektorientering i R utgår från generiska funktioner
\item
  Funktioner som gör olika saker beroende på objektets klass
\item
  Ex:

  \begin{itemize}
  \tightlist
  \item
    \texttt{plot( )}
  \item
    \texttt{mean( )}
  \item
    \texttt{summary( )}
  \item
    \texttt{names( )}
  \end{itemize}
\item
  Funktionerna anropar metoden för objektet när de används.
\end{itemize}
\end{frame}

\begin{frame}{Metoder}
\phantomsection\label{metoder}
\begin{itemize}
\tightlist
\item
  En funktion för en specifik klass
\item
  Metoderna utgår från generisk funktion
\item
  Alla metoder för en generisk funktion hittas med \texttt{methods( )}
\item
  T.ex. funktionen \texttt{mean( )}

  \begin{itemize}
  \tightlist
  \item
    Defaultmetoden finns med \texttt{mean.default( )}
  \item
    Specifik med \texttt{mean.difftime( )}
  \end{itemize}
\item
  Metoder kan också anropas direkt som vanliga funktioner.
\end{itemize}
\end{frame}

\begin{frame}{Demo}
\phantomsection\label{demo}
\begin{block}{Demo: Objektorientering}
\phantomsection\label{demo-objektorientering}
\end{block}
\end{frame}

\section{Datum och tid}\label{datum-och-tid}

\begin{frame}{Datum och tid}
\phantomsection\label{datum-och-tid-1}
\begin{itemize}
\tightlist
\item
  Datum är klurigt att arbeta med, men används extremt mycket.
\item
  Två typer av tid:

  \begin{itemize}
  \tightlist
  \item
    Relativ tid
  \item
    Exakt tid
  \end{itemize}
\item
  Enklare funktioner för datum finns i \texttt{base}
\end{itemize}
\end{frame}

\begin{frame}[fragile]{Datum och tid - Exempel}
\phantomsection\label{datum-och-tid---exempel}
\begin{Shaded}
\begin{Highlighting}[]
\NormalTok{my\_date }\OtherTok{\textless{}{-}} \StringTok{"2024{-}05{-}06"}
\FunctionTok{class}\NormalTok{(my\_date)}
\end{Highlighting}
\end{Shaded}

\begin{verbatim}
## [1] "character"
\end{verbatim}

\begin{Shaded}
\begin{Highlighting}[]
\NormalTok{my\_date\_as\_date }\OtherTok{\textless{}{-}} \FunctionTok{as.Date}\NormalTok{(my\_date)}
\NormalTok{my\_date\_as\_date}
\end{Highlighting}
\end{Shaded}

\begin{verbatim}
## [1] "2024-05-06"
\end{verbatim}

\begin{Shaded}
\begin{Highlighting}[]
\FunctionTok{class}\NormalTok{(my\_date\_as\_date)}
\end{Highlighting}
\end{Shaded}

\begin{verbatim}
## [1] "Date"
\end{verbatim}
\end{frame}

\begin{frame}{lubridate}
\phantomsection\label{lubridate}
\begin{itemize}
\tightlist
\item
  Paket för enkel och effektiv datumhantering
\item
  Sammansatt av ``lubricant'' och ``date''
\end{itemize}

Tre huvudsakliga delar:

\begin{enumerate}
\tightlist
\item
  Läsa in datum
\item
  Ändra inlästa datum
\item
  Göra beräkningar med datum
\end{enumerate}
\end{frame}

\begin{frame}[fragile]{lubridate - Exempel}
\phantomsection\label{lubridate---exempel}
\begin{Shaded}
\begin{Highlighting}[]
\FunctionTok{library}\NormalTok{(lubridate)}
\NormalTok{idag }\OtherTok{\textless{}{-}} \FunctionTok{ymd}\NormalTok{(}\StringTok{"2024{-}05{-}06"}\NormalTok{)}
\FunctionTok{print}\NormalTok{(idag)}
\end{Highlighting}
\end{Shaded}

\begin{verbatim}
## [1] "2024-05-06"
\end{verbatim}

\begin{Shaded}
\begin{Highlighting}[]
\FunctionTok{week}\NormalTok{(idag)}
\end{Highlighting}
\end{Shaded}

\begin{verbatim}
## [1] 19
\end{verbatim}

\begin{Shaded}
\begin{Highlighting}[]
\NormalTok{idag }\SpecialCharTok{+} \FunctionTok{weeks}\NormalTok{(}\DecValTok{2}\NormalTok{)}
\end{Highlighting}
\end{Shaded}

\begin{verbatim}
## [1] "2024-05-20"
\end{verbatim}
\end{frame}

\begin{frame}[fragile]{lubridate}
\phantomsection\label{lubridate-1}
\begin{longtable}[]{@{}ll@{}}
\toprule\noalign{}
Elementordning & Funktion \\
\midrule\noalign{}
\endhead
år, månad, dag & \texttt{ymd()} \\
år, dag, månad & \texttt{ydm()} \\
månad, dag, år & \texttt{mdy()} \\
timme, minut & \texttt{hm()} \\
timme, minut, sekund & \texttt{hms()} \\
år, mån, dag, timme, min, sek & \texttt{ymd\_hms()} \\
\bottomrule\noalign{}
\end{longtable}

Källa: Grolemund and Wickham (2011, Table 4)
\end{frame}

\begin{frame}[fragile]{lubridate}
\phantomsection\label{lubridate-2}
För att ``plocka ut'' eller ändra delar av ett datum används följande
funktioner

\begin{longtable}[]{@{}llll@{}}
\toprule\noalign{}
Datum & Funktion & Tidsdel & Funktion \\
\midrule\noalign{}
\endhead
år & \texttt{year()} & timme & \texttt{hour()} \\
månad & \texttt{month()} & minut & \texttt{minute()} \\
vecka & \texttt{week()} & sekund & \texttt{second()} \\
årsdag & \texttt{yday()} & tidszon & \texttt{tz()} \\
månadsdag & \texttt{mday()} & & \\
veckodag & \texttt{wday()} & & \\
\bottomrule\noalign{}
\end{longtable}

Källa: Grolemund and Wickham (2011, Table 5)
\end{frame}

\begin{frame}[fragile]{lubridate - Exempel}
\phantomsection\label{lubridate---exempel-1}
\begin{Shaded}
\begin{Highlighting}[]
\NormalTok{idag }\OtherTok{\textless{}{-}} \FunctionTok{ymd}\NormalTok{(}\StringTok{"2024{-}05{-}06"}\NormalTok{)}
\FunctionTok{week}\NormalTok{(idag)}
\end{Highlighting}
\end{Shaded}

\begin{verbatim}
## [1] 19
\end{verbatim}

\begin{Shaded}
\begin{Highlighting}[]
\FunctionTok{wday}\NormalTok{(idag, }\AttributeTok{label =} \ConstantTok{TRUE}\NormalTok{)}
\end{Highlighting}
\end{Shaded}

\begin{verbatim}
## [1] Mon
## Levels: Sun < Mon < Tue < Wed < Thu < Fri < Sat
\end{verbatim}

\begin{Shaded}
\begin{Highlighting}[]
\FunctionTok{year}\NormalTok{(idag) }\OtherTok{\textless{}{-}} \DecValTok{2019}
\NormalTok{idag}
\end{Highlighting}
\end{Shaded}

\begin{verbatim}
## [1] "2019-05-06"
\end{verbatim}
\end{frame}

\begin{frame}{lubridate}
\phantomsection\label{lubridate-3}
\begin{itemize}
\tightlist
\item
  För att räkna med datum finns det fyra olika objekt i
  \texttt{lubridate}

  \begin{itemize}
  \tightlist
  \item
    \texttt{instant}
  \item
    \texttt{interval}
  \item
    \texttt{duration}
  \item
    \texttt{period}
  \end{itemize}
\end{itemize}
\end{frame}

\begin{frame}{instant och interval}
\phantomsection\label{instant-och-interval}
\textbf{Instant}

\begin{itemize}
\tightlist
\item
  Ett spcifikt tillfälle i tiden
\item
  Viktiga funktioner:

  \begin{itemize}
  \tightlist
  \item
    \texttt{now( )}
  \item
    \texttt{today( )}
  \end{itemize}
\end{itemize}

\textbf{Interval}

\begin{itemize}
\tightlist
\item
  Tidsspannet mellan två \texttt{instant}
\item
  \texttt{interval(start,end)}
\end{itemize}
\end{frame}

\begin{frame}[fragile]{instant och interval - Exempel}
\phantomsection\label{instant-och-interval---exempel}
\begin{Shaded}
\begin{Highlighting}[]
\NormalTok{inst\_1 }\OtherTok{\textless{}{-}} \FunctionTok{ymd}\NormalTok{(}\StringTok{"2020{-}02{-}29"}\NormalTok{)}
\NormalTok{inst\_2 }\OtherTok{\textless{}{-}} \FunctionTok{today}\NormalTok{()}
\NormalTok{my\_interval }\OtherTok{\textless{}{-}} \FunctionTok{interval}\NormalTok{(}\AttributeTok{start =}\NormalTok{ inst\_1, }\AttributeTok{end =}\NormalTok{ inst\_2)}
\NormalTok{my\_interval}
\end{Highlighting}
\end{Shaded}

\begin{verbatim}
## [1] 2020-02-29 UTC--2024-05-05 UTC
\end{verbatim}
\end{frame}

\begin{frame}{duration}
\phantomsection\label{duration}
\begin{itemize}
\tightlist
\item
  Ett fixt tidsspann som mäts i sekunder
\item
  Tänk kontinuerlig tid
\item
  Absolut tid i sekunder
\item
  Funktioner börjar med \texttt{d}
\item
  Konvertera ett interval med \texttt{as.duration( )}
\item
  Ex:

  \begin{itemize}
  \tightlist
  \item
    \texttt{duration( )}
  \item
    \texttt{dseconds( )}
  \item
    \texttt{dhours( )}
  \end{itemize}
\end{itemize}
\end{frame}

\begin{frame}[fragile]{duration - Exempel}
\phantomsection\label{duration---exempel}
\begin{Shaded}
\begin{Highlighting}[]
\NormalTok{my\_interval }\SpecialCharTok{/} \FunctionTok{ddays}\NormalTok{(}\DecValTok{30}\NormalTok{)}
\end{Highlighting}
\end{Shaded}

\begin{verbatim}
## [1] 50.9
\end{verbatim}

\begin{Shaded}
\begin{Highlighting}[]
\FunctionTok{as.duration}\NormalTok{(my\_interval)}
\end{Highlighting}
\end{Shaded}

\begin{verbatim}
## [1] "131932800s (~4.18 years)"
\end{verbatim}
\end{frame}

\begin{frame}{Period}
\phantomsection\label{period}
\begin{itemize}
\tightlist
\item
  Utgår från den aktuella enheten (dagar, månader, år)
\item
  Tänk diskret tid
\item
  Relativ tid
\item
  Vad vi i dagligt tal menar med ex. två veckor
\item
  Ex:

  \begin{itemize}
  \tightlist
  \item
    \texttt{period(num = , units = )}
  \item
    \texttt{minutes( )}
  \item
    \texttt{hours( )}
  \item
    \texttt{weeks( )}
  \end{itemize}
\end{itemize}
\end{frame}

\begin{frame}[fragile]{Period - Exempel}
\phantomsection\label{period---exempel}
\begin{Shaded}
\begin{Highlighting}[]
\FunctionTok{months}\NormalTok{(}\DecValTok{3}\NormalTok{) }\SpecialCharTok{+} \FunctionTok{weeks}\NormalTok{(}\DecValTok{2}\NormalTok{) }\SpecialCharTok{+} \FunctionTok{days}\NormalTok{(}\DecValTok{1}\NormalTok{)}
\end{Highlighting}
\end{Shaded}

\begin{verbatim}
## [1] "3m 15d 0H 0M 0S"
\end{verbatim}

\begin{Shaded}
\begin{Highlighting}[]
\NormalTok{my\_interval }\SpecialCharTok{\%/\%} \FunctionTok{months}\NormalTok{(}\DecValTok{4}\NormalTok{)}
\end{Highlighting}
\end{Shaded}

\begin{verbatim}
## [1] 12
\end{verbatim}

\begin{Shaded}
\begin{Highlighting}[]
\FunctionTok{as.period}\NormalTok{(my\_interval)}
\end{Highlighting}
\end{Shaded}

\begin{verbatim}
## [1] "4y 2m 6d 0H 0M 0S"
\end{verbatim}
\end{frame}

\begin{frame}[fragile]{Att räknamed tid - Exempel}
\phantomsection\label{att-ruxe4knamed-tid---exempel}
\begin{Shaded}
\begin{Highlighting}[]
\NormalTok{course\_start }\OtherTok{\textless{}{-}} \FunctionTok{ymd}\NormalTok{(}\StringTok{"2022{-}01{-}24"}\NormalTok{)}
\NormalTok{course\_start }\SpecialCharTok{+} \FunctionTok{weeks}\NormalTok{(}\DecValTok{5}\NormalTok{)}
\end{Highlighting}
\end{Shaded}

\begin{verbatim}
## [1] "2022-02-28"
\end{verbatim}

\begin{Shaded}
\begin{Highlighting}[]
\NormalTok{course\_start }\SpecialCharTok{+} \FunctionTok{months}\NormalTok{(}\DecValTok{1}\NormalTok{)}
\end{Highlighting}
\end{Shaded}

\begin{verbatim}
## [1] "2022-02-24"
\end{verbatim}

\begin{Shaded}
\begin{Highlighting}[]
\NormalTok{course\_start }\SpecialCharTok{+} \FunctionTok{ddays}\NormalTok{(}\DecValTok{30}\NormalTok{)}
\end{Highlighting}
\end{Shaded}

\begin{verbatim}
## [1] "2022-02-23"
\end{verbatim}
\end{frame}

\begin{frame}{Demo}
\phantomsection\label{demo-1}
\begin{block}{Demo: Datum och Tid}
\phantomsection\label{demo-datum-och-tid}
\end{block}
\end{frame}

\section{Linjär algebra}\label{linjuxe4r-algebra}

\begin{frame}{Matriser}
\phantomsection\label{matriser}
\begin{itemize}
\tightlist
\item
  Matriser är två-dimensionella vektorer
\item
  Har jobbat med matriser förut.
\item
  Skapa en matris med

  \begin{itemize}
  \tightlist
  \item
    \texttt{matrix( data = , nrow = , ncol = , byrow = )}
  \item
    Kom ihåg cirkulering
  \item
    \texttt{byrow = FALSE} per deafult
  \end{itemize}
\item
  Transponat fås med \texttt{t( )}
\item
  Diagonalmatris med \texttt{diag( vektor )}
\item
  Enhetsmatrisen med \texttt{diag( nummer )}
\end{itemize}
\end{frame}

\begin{frame}{Linjär algebra}
\phantomsection\label{linjuxe4r-algebra-1}
\begin{itemize}
\tightlist
\item
  Matrismultiplikation görs med \texttt{\%*\%}
\item
  Invertera en matris med \texttt{solve( )}

  \begin{itemize}
  \tightlist
  \item
    Notera att \texttt{A \%*\% solve(A)} blir en enhetsmatris
  \end{itemize}
\item
  Lösa ekvationssystem \(Ax = b\) med \texttt{solve(a=A,b=b)}

  \begin{itemize}
  \tightlist
  \item
    Notera att om \(A\) inverterbar är svaret \(x = A^{-1}b\).
  \end{itemize}
\item
  Egenvärden och egenvektorer med \texttt{eigen()}

  \begin{itemize}
  \tightlist
  \item
    Returnerar en lista med egenvärden och egenvektorer
  \end{itemize}
\item
  Summera rader eler kolumner med \texttt{rowSums()} och
  \texttt{colSums()}
\item
  Kombinera matriser med \texttt{rbind( , )} eller \texttt{cbind( , )}
\end{itemize}
\end{frame}

\begin{frame}{Linjär algebra - paketet Matrix}
\phantomsection\label{linjuxe4r-algebra---paketet-matrix}
\begin{itemize}
\tightlist
\item
  Matrix är ett paket som implementerar effektivare/snabbare funktioner
  för linjär algebra.
\item
  Grunden är funktionen \texttt{Matrix( )}, vilket är en konstruktor som
  skapar matriser.
\item
  Matriserna i \textbf{Matrix} är S4 klasser.
\item
  I \textbf{Matrix} är det skillnad på glesa och täta matriser.
\item
  Kolla dokumentation för mer information.
\item
  Rekomenderas för avancerade tillämpningar.
\end{itemize}
\end{frame}

\begin{frame}{Demo}
\phantomsection\label{demo-2}
\begin{block}{Demo: Linjär Algebra}
\phantomsection\label{demo-linjuxe4r-algebra}
\end{block}
\end{frame}

\end{document}
