% Options for packages loaded elsewhere
\PassOptionsToPackage{unicode}{hyperref}
\PassOptionsToPackage{hyphens}{url}
%
\documentclass[
  10pt,
  ignorenonframetext,
]{beamer}
\usepackage{pgfpages}
\setbeamertemplate{caption}[numbered]
\setbeamertemplate{caption label separator}{: }
\setbeamercolor{caption name}{fg=normal text.fg}
\beamertemplatenavigationsymbolsempty
% Prevent slide breaks in the middle of a paragraph
\widowpenalties 1 10000
\raggedbottom
\setbeamertemplate{part page}{
  \centering
  \begin{beamercolorbox}[sep=16pt,center]{part title}
    \usebeamerfont{part title}\insertpart\par
  \end{beamercolorbox}
}
\setbeamertemplate{section page}{
  \centering
  \begin{beamercolorbox}[sep=12pt,center]{part title}
    \usebeamerfont{section title}\insertsection\par
  \end{beamercolorbox}
}
\setbeamertemplate{subsection page}{
  \centering
  \begin{beamercolorbox}[sep=8pt,center]{part title}
    \usebeamerfont{subsection title}\insertsubsection\par
  \end{beamercolorbox}
}
\AtBeginPart{
  \frame{\partpage}
}
\AtBeginSection{
  \ifbibliography
  \else
    \frame{\sectionpage}
  \fi
}
\AtBeginSubsection{
  \frame{\subsectionpage}
}
\usepackage{amsmath,amssymb}
\usepackage{lmodern}
\usepackage{iftex}
\ifPDFTeX
  \usepackage[T1]{fontenc}
  \usepackage[utf8]{inputenc}
  \usepackage{textcomp} % provide euro and other symbols
\else % if luatex or xetex
  \usepackage{unicode-math}
  \defaultfontfeatures{Scale=MatchLowercase}
  \defaultfontfeatures[\rmfamily]{Ligatures=TeX,Scale=1}
\fi
% Use upquote if available, for straight quotes in verbatim environments
\IfFileExists{upquote.sty}{\usepackage{upquote}}{}
\IfFileExists{microtype.sty}{% use microtype if available
  \usepackage[]{microtype}
  \UseMicrotypeSet[protrusion]{basicmath} % disable protrusion for tt fonts
}{}
\makeatletter
\@ifundefined{KOMAClassName}{% if non-KOMA class
  \IfFileExists{parskip.sty}{%
    \usepackage{parskip}
  }{% else
    \setlength{\parindent}{0pt}
    \setlength{\parskip}{6pt plus 2pt minus 1pt}}
}{% if KOMA class
  \KOMAoptions{parskip=half}}
\makeatother
\usepackage{xcolor}
\IfFileExists{xurl.sty}{\usepackage{xurl}}{} % add URL line breaks if available
\IfFileExists{bookmark.sty}{\usepackage{bookmark}}{\usepackage{hyperref}}
\hypersetup{
  pdftitle={R-programmering VT2022},
  pdfauthor={Josef Wilzén},
  hidelinks,
  pdfcreator={LaTeX via pandoc}}
\urlstyle{same} % disable monospaced font for URLs
\newif\ifbibliography
\usepackage{color}
\usepackage{fancyvrb}
\newcommand{\VerbBar}{|}
\newcommand{\VERB}{\Verb[commandchars=\\\{\}]}
\DefineVerbatimEnvironment{Highlighting}{Verbatim}{commandchars=\\\{\}}
% Add ',fontsize=\small' for more characters per line
\usepackage{framed}
\definecolor{shadecolor}{RGB}{248,248,248}
\newenvironment{Shaded}{\begin{snugshade}}{\end{snugshade}}
\newcommand{\AlertTok}[1]{\textcolor[rgb]{0.94,0.16,0.16}{#1}}
\newcommand{\AnnotationTok}[1]{\textcolor[rgb]{0.56,0.35,0.01}{\textbf{\textit{#1}}}}
\newcommand{\AttributeTok}[1]{\textcolor[rgb]{0.77,0.63,0.00}{#1}}
\newcommand{\BaseNTok}[1]{\textcolor[rgb]{0.00,0.00,0.81}{#1}}
\newcommand{\BuiltInTok}[1]{#1}
\newcommand{\CharTok}[1]{\textcolor[rgb]{0.31,0.60,0.02}{#1}}
\newcommand{\CommentTok}[1]{\textcolor[rgb]{0.56,0.35,0.01}{\textit{#1}}}
\newcommand{\CommentVarTok}[1]{\textcolor[rgb]{0.56,0.35,0.01}{\textbf{\textit{#1}}}}
\newcommand{\ConstantTok}[1]{\textcolor[rgb]{0.00,0.00,0.00}{#1}}
\newcommand{\ControlFlowTok}[1]{\textcolor[rgb]{0.13,0.29,0.53}{\textbf{#1}}}
\newcommand{\DataTypeTok}[1]{\textcolor[rgb]{0.13,0.29,0.53}{#1}}
\newcommand{\DecValTok}[1]{\textcolor[rgb]{0.00,0.00,0.81}{#1}}
\newcommand{\DocumentationTok}[1]{\textcolor[rgb]{0.56,0.35,0.01}{\textbf{\textit{#1}}}}
\newcommand{\ErrorTok}[1]{\textcolor[rgb]{0.64,0.00,0.00}{\textbf{#1}}}
\newcommand{\ExtensionTok}[1]{#1}
\newcommand{\FloatTok}[1]{\textcolor[rgb]{0.00,0.00,0.81}{#1}}
\newcommand{\FunctionTok}[1]{\textcolor[rgb]{0.00,0.00,0.00}{#1}}
\newcommand{\ImportTok}[1]{#1}
\newcommand{\InformationTok}[1]{\textcolor[rgb]{0.56,0.35,0.01}{\textbf{\textit{#1}}}}
\newcommand{\KeywordTok}[1]{\textcolor[rgb]{0.13,0.29,0.53}{\textbf{#1}}}
\newcommand{\NormalTok}[1]{#1}
\newcommand{\OperatorTok}[1]{\textcolor[rgb]{0.81,0.36,0.00}{\textbf{#1}}}
\newcommand{\OtherTok}[1]{\textcolor[rgb]{0.56,0.35,0.01}{#1}}
\newcommand{\PreprocessorTok}[1]{\textcolor[rgb]{0.56,0.35,0.01}{\textit{#1}}}
\newcommand{\RegionMarkerTok}[1]{#1}
\newcommand{\SpecialCharTok}[1]{\textcolor[rgb]{0.00,0.00,0.00}{#1}}
\newcommand{\SpecialStringTok}[1]{\textcolor[rgb]{0.31,0.60,0.02}{#1}}
\newcommand{\StringTok}[1]{\textcolor[rgb]{0.31,0.60,0.02}{#1}}
\newcommand{\VariableTok}[1]{\textcolor[rgb]{0.00,0.00,0.00}{#1}}
\newcommand{\VerbatimStringTok}[1]{\textcolor[rgb]{0.31,0.60,0.02}{#1}}
\newcommand{\WarningTok}[1]{\textcolor[rgb]{0.56,0.35,0.01}{\textbf{\textit{#1}}}}
\usepackage{longtable,booktabs,array}
\usepackage{calc} % for calculating minipage widths
\usepackage{caption}
% Make caption package work with longtable
\makeatletter
\def\fnum@table{\tablename~\thetable}
\makeatother
\setlength{\emergencystretch}{3em} % prevent overfull lines
\providecommand{\tightlist}{%
  \setlength{\itemsep}{0pt}\setlength{\parskip}{0pt}}
\setcounter{secnumdepth}{-\maxdimen} % remove section numbering
\usetheme[progressbar=frametitle,block=fill]{metropolis} %numbering=none

%%% USEFUL PACKAGES
%\usepackage{showframe} % For debugging positioning
\usepackage{etex} % If too many packages
% Encoding and language
\usepackage[utf8]{inputenc}
\usepackage{babel}
\usepackage{amsmath, amssymb}
\usepackage{natbib}
%\usepackage{booktabs}
%\usepackage{algorithmic}
\usepackage{algorithm}
\usepackage{caption}
%\usepackage{animate} % Animations
\usepackage{bm} % Bold math
\usepackage{bbm}
%\usepackage{url}
%\usepackage{pifont}
%\usepackage{ulem} % Used for strikeouts \sout
%\usepackage{stackengine}
%\usepackage{enumitem}
%\setlist[description]{leftmargin=\parindent,labelindent=\parindent}
%\usepackage{colortbl} % Used for colored rows in tables


%%% GRAPHICS
\usepackage{graphicx}
\graphicspath{{./figs/}}


%%% COLORS
\setbeamercolor{background canvas}{bg=white}
\def\BlankFrame{
	\bgroup
	%\pdfpageheight 29.7cm
	\setbeamercolor{background canvas}{bg=}
	\begin{frame}[plain]
	\end{frame}
	%\makeatletter
	%\pdfpageheight \beamer@paperheight
	%\makeatother
	\egroup}

\usepackage{xcolor}
\definecolor{DarkGreen}{HTML}{00B200}
\definecolor{LightBlue}{HTML}{0090D9}
\definecolor{gold}{rgb}{.812,.710,.231}
% Text markup
%\setbeamercolor{alerted text}{fg=red}
\newcommand{\blue}[1]{\textcolor{blue}{#1}}
\newcommand{\red}[1]{\textcolor{red}{#1}}
\newcommand{\grey}[1]{\textcolor{gray}{#1}}
\newcommand{\orange}[1]{\textcolor{mLightBrown}{#1}}
\newcommand\myheading[1]{\textbf{#1}}
\newcommand\myemph[1]{\underline{\emph{#1}}}
\newcommand\textexample[1]{\textit{\textbf{#1}}}

%%% SPACING
\newcommand\vws[1][1]{\vspace{#1\baselineskip}} % vertical white space
%\newcommand\strt[1][1.5ex]{\rule[-.05\baselineskip]{0pt}{#1}} % strut
\newcommand\strt[2]{\rule[-#1ex]{0pt}{#2ex}} % strut
\newcommand\Hrule{\vspace{1ex} \hrule \vspace{1ex}} % Horisontal rule with some space after

%%% MISC
\newcommand\articleref[4]{\noindent\begin{minipage}[t]{0.04\textwidth}
		\vspace{0pt} 
		\pgfuseimage{beamericonarticle}
	\end{minipage}%
	\begin{minipage}[t]{0.96\textwidth}
		\vspace{0pt}
		#1. \textbf{#2.} \textit{#3}, #4.
	\end{minipage}}

%%% METROPOLIS THEME SPECIFIC
\makeatletter
\setlength{\metropolis@progressonsectionpage@linewidth}{1pt}
\makeatother
%\setbeamercolor{progress bar}{fg=red,bg=red!50}


%%% TEXTPOS
\usepackage[absolute,overlay]{textpos} % option showboxes is useful in draft mode
\setlength{\TPHorizModule}{\paperwidth}
\setlength{\TPVertModule}{\paperheight}
\textblockorigin{0pt}{10mm} % start everything at top-left, below gray 


%%% TIKZ/PGFPLOTS
\usepackage{tikz}
\usetikzlibrary{arrows,positioning,calc,shapes.geometric}
%\usetikzlibrary{arrows,calc,shapes.geometric,decorations.pathmorphing,backgrounds,positioning,fit,petri,decorations.pathreplacing}
%\usepackage{pgfplots}
%\pgfplotsset{compat = 1.3}


%%% BLOCKS AND BOXES
% Changing colors of blocks
%\setbeamercolor{block title alerted}{bg=UURed,fg=palette primary.fg}
%\setbeamercolor{block body alerted}{bg=UURed!15}
\setbeamercolor{block title alerted}{bg=mLightBrown,fg=palette primary.fg}
\setbeamercolor{block body alerted}{bg=mLightBrown!15}
%\setbeamercolor{block title example}{bg=UUGreen,fg=palette primary.fg}
%\setbeamercolor{block body example}{bg=UUGreen!10}
% \mybox is a rectangular box
\usepackage{boxedminipage}
\setlength\fboxrule{2pt}
\setlength\fboxsep{2\fboxsep}
\newcommand\mybox[3][\textwidth]{
  {\color{#2}
    \begin{boxedminipage}{#1}
      {\color{palette primary.bg} #3}
    \end{boxedminipage}}%
}   
\usepackage{tcolorbox}
\tcbset{arc=1mm,grow to left by=3mm,grow to right by=3mm,left=2mm}
%\newenvironment{redbox}{%
%	\begin{tcolorbox}[colback=UURed!15,colframe=UURed]}{%
%	\end{tcolorbox}}
%\newenvironment{greenbox}{%
%	\begin{tcolorbox}[colback=UUGreen!15,colframe=UUGreen]}{%
%	\end{tcolorbox}}
\newenvironment{redbox}{%
	\begin{tcolorbox}[colback=red!15,colframe=red]}{%
	\end{tcolorbox}}
\newenvironment{greenbox}{%
	\begin{tcolorbox}[colback=DarkGreen!15,colframe=DarkGreen]}{%
	\end{tcolorbox}}
\newenvironment{graybox}{%
	\begin{tcolorbox}[colback=mDarkTeal!5,colframe=mDarkTeal]}{%
	\end{tcolorbox}}
\newenvironment{orangebox}{%
\begin{tcolorbox}[colback=mLightBrown!15,colframe=mLightBrown]}{%
	\end{tcolorbox}}
\newenvironment{bwbox}{%
	\begin{tcolorbox}[colback=white,colframe=black]}{%
\end{tcolorbox}}
\newenvironment{bluebox}{%
	\begin{tcolorbox}[colback=LightBlue!15,colframe=LightBlue]}{%
\end{tcolorbox}}


\hypersetup{
    colorlinks,
    linkcolor={red!50!black},
    citecolor={blue!50!black},
    urlcolor={blue!80!black}
}



%%%%%%%%% NEW MACROS

\newcommand\imp[1]{\alert{\textbf{#1}}}
\newcommand\bfit[1]{\textbf{\textit{#1}}}
\newcommand\good{\color{DarkGreen}{$\blacktriangle$}} % used in lists
\newcommand\bad{\color{red}{$\blacktriangledown$}} % used in lists

\ifLuaTeX
  \usepackage{selnolig}  % disable illegal ligatures
\fi

\title{R-programmering VT2022}
\subtitle{Föreläsning 1}
\author{Josef Wilzén}
\date{2023-01-23}
\institute{Linköpings Universitet}

\begin{document}
\frame{\titlepage}


%%%%%%%%% slide %%%%%%%%% 
\begin{frame}{Föreläsning 1:}
\protect\hypertarget{fuxf6reluxe4sning-1}{}
\begin{itemize}
\tightlist
\item
  Introduktion till kursen
\item
  R, RStudio
\item
  Introduktion till R-programering

  \begin{itemize}
  \tightlist
  \item
    Miniräknare
  \item
    Variabler
  \item
    Vektorer
  \item
    Hjälp
  \item
    Funktioner
  \item
    Logik
  \end{itemize}
\end{itemize}
\end{frame}


%%%%%%%%% slide %%%%%%%%% 

\begin{frame}{Vilka är vi?}
\protect\hypertarget{vilka-uxe4r-vi}{}
\textbf{Föreläsare och examinator}:

\begin{itemize}
\tightlist
\item
  Josef Wilzén
\end{itemize}

\textbf{Labbassistenter}:

\begin{itemize}
\tightlist
\item
  Jack Hagerstål
\item
  Oskar Storberg
\item
  William Wiik
\end{itemize}
\end{frame}

%%%%%%%%% slide %%%%%%%%% 

\begin{frame}{Kursens mål}
\protect\hypertarget{kursens-muxe5l}{}
Information om kursen finns i \href{https://studieinfo.liu.se/kurs/732G33/vt-2023}{kursplanen}

\textbf{Lärandemål}

\begin{itemize}
\tightlist
\item
  skapa enkla program i programspråket R med hjälp av grundläggande
  programmeringstekniker som inläsning och utskrift av data, tilldelning
  och manipulation av datastrukturer, skriva egna funktioner,
  upprepningar och villkorsstyrda satser.
\end{itemize}

\pause Vi sammanfattar detta till

\begin{itemize}
\tightlist
\item
  Bli bekväm med att använda R
\item
  Hantera data med R
\item
  Skriva program i R
\end{itemize}
\end{frame}

%%%%%%%%% slide %%%%%%%%% 

\begin{frame}{Tidigare år}
\protect\hypertarget{tidigare-uxe5r}{}
kursutvärderingar 2021:

\begin{enumerate}
\tightlist
\item
  Kursens ämnesinnehåll har gett mig möjlighet att uppnå kursens
  lärandemål. 3.54
\item
  Kursens olika undervisnings- och arbetsformer har varit relevanta i relation till
kursens lärandemål. 3.86
\item
  Kursens examinerande moment har varit relevanta i relation till
  kursens lärandemål. 3.54
\item
  Vilket helhetsbetyg ger du kursen? 3.29
\end{enumerate}

Annan föreläsare och examinator förra året.
\begin{itemize}
\tightlist
\item
  Mindre förändringar kommer att ske i
föreläsningar/laborationer
\item
  Minska ner omfånget på inlämningsuppgifter något
\item
  Minska ner antalet uppgifter på tentan:  4 istället för 5
\item
  Seminarier (från kursvecka 2)
\end{itemize}


\end{frame}

%%%%%%%%% slide %%%%%%%%% 

\begin{frame}{Kursupplägg}
\protect\hypertarget{kursuppluxe4gg}{}
\begin{block}{Kursen består av \textbf{två} delar:}
\protect\hypertarget{kursen-bestuxe5r-av-tvuxe5-delar}{}
\begin{itemize}
\tightlist
\item
  Del 1: Grundläggande programering
\item
  Del 2: Tillämpningar relaterade till statistik, grafik och
  datahantering
\end{itemize}

\pause
\end{block}

\begin{block}{Varje vecka}
\protect\hypertarget{varje-vecka}{}
\begin{itemize}
\tightlist
\item
  En föreläsning
\item Datorlaboration
  \begin{itemize}
        \item Övningsuppgifter $\rightarrow$ viktigt!
        \item \textcolor{red}{Obligatoriska} inlämningsuppgifter
  \end{itemize}
  
\item
  4 timmar lärarledd laboration
\item
  Inlämning:

  \begin{itemize}
  \tightlist
  \item
    Labbar: varje \textcolor{red}{söndag kl. 18:00} via LISAM
  \end{itemize}
\item
  Seminarie (från kursvecka 2): koddemo, lösningar på övningsuppgifter, frågor,mm

\end{itemize}
\end{block}
\end{frame}

%%%%%%%%% slide %%%%%%%%% 

\begin{frame}{Del 1: Grundläggande programering}
\protect\hypertarget{del-1-grundluxe4ggande-programering}{}
\begin{itemize}
\tightlist
\item
  Grunderna i R

  \begin{itemize}
  \tightlist
  \item
    Lära sig hantera RStudio
  \end{itemize}
\item
  Fyra föreläsningar
\item
  Fyra inlämningar (datorlaborationer)
\item
  Labbarna görs \textbf{en och en}
\end{itemize}
\end{frame}

%%%%%%%%% slide %%%%%%%%% 

\begin{frame}{Del 2: Tillämpningar}
\protect\hypertarget{del-2-tilluxe4mpningar}{}
\begin{itemize}
\tightlist
\item
  Statistisk analys med R
\item
  Fyra föreläsningar
\item
  Fyra inlämningar
\item
  Labbarna förs genom \textbf{parprogrammering} (grupper om två)
\item
  \textbf{Projekt} i två delar
\end{itemize}

\pause

\textcolor{red}{Jobba med materialet och skriv egen kod!}
\end{frame}

%%%%%%%%% slide %%%%%%%%% 

\begin{frame}{Praktisk information}
\protect\hypertarget{praktisk-information}{}
\begin{block}{Kurslogistik}
\protect\hypertarget{kurslogistik}{}
Kurshemsidan innehåller föreläsningar, labbar m.m. Finns: \href{https://www.ida.liu.se/~732G33/info/courseinfo.sv.shtml}{här}

LISAM används för inlämning av labbar och kompletteringar

Teams används för kommunikation
\end{block}

\begin{block}{Programvara}
\protect\hypertarget{programvara}{}
I denna kurs använder vi \textbf{R} och \textbf{RStudio}
\end{block}
\end{frame}

%%%%%%%%% slide %%%%%%%%% 

\begin{frame}{Kurslitteratur I:}
\protect\hypertarget{kurslitteratur-i}{}
\begin{block}{Kursboken}
\protect\hypertarget{kursboken}{}
The Book Of R av Tilman M. Davies, 2016

Den finns som e-bok via biblioteket.
\end{block}

\begin{block}{Artiklar}
\protect\hypertarget{artiklar}{}
Dessa finns tillgängliga via kurshemsidan

\begin{itemize}
\tightlist
\item
  Dates and Times Made Easy with \texttt{lubridate}
\item
  Handling and processing string in R
\item
  Best practices for scientific computing
\end{itemize}
\end{block}
\end{frame}

%%%%%%%%% slide %%%%%%%%% 

\begin{frame}{Kurslitteratur II:}
\protect\hypertarget{kurslitteratur-ii}{}
\begin{block}{Videoföreläsningar}
\protect\hypertarget{videofuxf6reluxe4sningar}{}
\begin{itemize}
\tightlist
\item
  Google Developers videomaterial
\item
  Roger Pengs föreläsningar
\end{itemize}

Länkar finns på kurshemsidan
\end{block}

\begin{block}{Reference cards:}
\protect\hypertarget{reference-cards}{}
Olika referenskort med funktionsnamn och hjälp finns på kurshemsidan.
\end{block}
\end{frame}

%%%%%%%%% slide %%%%%%%%% 

\begin{frame}{Examination}
\protect\hypertarget{examination}{}
\begin{itemize}
\tightlist
\item
  Datorlaborationer, 8st
\item
  Datortentamen i datorsal

  \begin{itemize}
  \tightlist
  \item
    Hjälpmedel: Inbyggda hjälpen i R/Rstudio, R reference card (digitalt) + några fler reference-cards. Information om
    vilka kommer komma på kurshemsidan. Dessa erhålls digitalt på
    tentamenstillfället, ni ska inte ta med er några papper.
  \end{itemize}
\end{itemize}
\end{frame}

%%%%%%%%% slide %%%%%%%%% 

\begin{frame}{Datorlaborationer}
\protect\hypertarget{datorlaborationer}{}
\begin{itemize}
\tightlist
\item
  Börja direkt
\item
  Har \textcolor{red}{obligatoriska} inlämningsuppgifter
\item
  Ungefär 15 h arbete per vecka.
\item
  Laborationsmall finns på hemsidan
\item
  Laborationer lämnas in via LISAM
\item
  Autorättning används på en del av uppgifterna, se till att följa
  instruktionerna, $\rightarrow$ paketet markmyassignment
\item
  100\% rätt för att bli godkänd
\end{itemize}
\end{frame}

%%%%%%%%% slide %%%%%%%%%

\begin{frame}{Datorlaborationer}
\protect\hypertarget{datorlaborationer-1}{}
\begin{itemize}
\tightlist
\item
  Arbetstakt:

  \begin{itemize}
  \tightlist
  \item
    Kursveckorna går måndag till söndag
  \item
    Kursen går på halvfart \textasciitilde20h/vecka. Ungefär 15h/vecka
    till labbar.
  \item
    Deadline: söndag kl 18:00
  \end{itemize}
\item
  Kompletteringar:

  \begin{itemize}
  \tightlist
  \item
    Labb 1-4: komplettering strax efter halva kursen. Se kurshemsidan
    för info.
  \item
    Komplettering i samband med tentan och omtentor.
  \end{itemize}
\end{itemize}
\end{frame}

%%%%%%%%% slide %%%%%%%%%

\begin{frame}{Programmering}
\protect\hypertarget{programmering}{}
\begin{itemize}[<+->]
\tightlist
\item
  Instruera en dator att utföra uppgifter
\item
  Mjukvaruutveckling - vetenskaplig programmering
\item
  Maskinkod - Lågnivåspråk - Högnivåspråk
\item
  Kompileranmde språk - Interpreterande språk
\end{itemize}
\end{frame}

%%%%%%%%% slide %%%%%%%%%

\begin{frame}{Varför lära sig programera?}
\protect\hypertarget{varfuxf6r-luxe4ra-sig-programera}{}
\begin{itemize}[<+->]
\tightlist
\item
  Lösa problem
\item
  Hantera data av olika sorter: tabeller, databaser, listor, bilder, text
\item
  Replikerbarhet
\item
  Komplexa beräkningar
\item
  Automatisera
\end{itemize}
\end{frame}

%%%%%%%%% slide %%%%%%%%%

\begin{frame}{Programmering och programmeringsspråk}
\protect\hypertarget{varfuxf6r-luxe4ra-sig-programera}{}
\begin{itemize}[<+->]
\tightlist
\item
  Programmering kräver ett programmeringsspråk
\item
  Det finns många olika programmeringsspråk
\item
  Olika språk är bra på olika saker
\item
  Exempel
  \begin{itemize}
    \item Python, Javascript, C/C++, Java, Go, ...
  \end{itemize}
\item
  Språk som är bra för dataanalys, statistik, maskininlärning, data science mm
  \begin{itemize}
    \item R, Python, Julia, SQL, Matlab
  \end{itemize}
\end{itemize}
\end{frame}

%%%%%%%%% slide %%%%%%%%% 

\begin{frame}{Studieteknik}
\protect\hypertarget{datorlaborationer}{}
\begin{itemize}[<+->]
\tightlist
\item
  Ni har ansvar för era egna studier och er egen inlärning
\item
  Inlärning kräver engagemang och arbete
\item
  Rekommenderas att ni planerar era studier: skriv ner vecka för vecka vad ni ska göra och när
\item
  Programmering:
    \begin{itemize}
      \item Teoretisk färighet 
      \item Till stor del en praktisk färdiget
    \end{itemize}
\item Viktigt att skriva kod kontinuerligt under hela kursen
\item Kom förbereda till datorlaborationerna! $\rightarrow$ fråga om hjälp!!!
\end{itemize}
\end{frame}


%%%%%%%%% slide %%%%%%%%% 

\begin{frame}{Studieteknik}
\protect\hypertarget{datorlaborationer}{}
Förslag på upplägg:
\begin{itemize}[<+->]
\tightlist
  \item Måndag: Föreläsning + läsa kurslitteratur $\sim$ 4 h
  \item Försök att arbeta $\sim$ 4 h med datorlaborationen innan ni kommer till labbpasset, per pass (= 8 h)
  \item Gå på de bokade labbpassen $\sim$ 4 h
  \item Egenstudier och seminarie $\sim$ 4 h
  \item totalt: 20 h
\end{itemize}
\end{frame}


%%%%%%%%% slide %%%%%%%%%

\begin{frame}{Vad är R?}
\protect\hypertarget{vad-uxe4r-r}{}
\begin{itemize}[<+->]
\tightlist
\item
  R är ett populärt programmeringsspråk för statistiker $\rightarrow$ finns många funktioner för datahatering, statistik och visualisering  
\item
  Öppen källkod
\item
  Många utvecklare 
\item
  Interpreterande högnivåspråk
\end{itemize}
\end{frame}

%%%%%%%%% slide %%%%%%%%%

\begin{frame}[fragile]{Ett exempel på ett program i R}
\protect\hypertarget{ett-exempel-puxe5-ett-program-i-r}{}
Skapa ett program som skriver ut talen från 10 till 1 och sen skriver
``kör!''.

\pause

I R ser det ut på följande sätt

\begin{Shaded}
\begin{Highlighting}[]
\NormalTok{start }\OtherTok{\textless{}{-}} \DecValTok{10}
\ControlFlowTok{for}\NormalTok{ (i }\ControlFlowTok{in} \DecValTok{1}\SpecialCharTok{:}\DecValTok{10}\NormalTok{) \{}
  \FunctionTok{print}\NormalTok{(start)}
\NormalTok{  start }\OtherTok{\textless{}{-}}\NormalTok{ start }\SpecialCharTok{{-}} \DecValTok{1}
\NormalTok{\}}
\FunctionTok{print}\NormalTok{(}\StringTok{"Kör!"}\NormalTok{)}
\end{Highlighting}
\end{Shaded}
\end{frame}

%%%%%%%%% slide %%%%%%%%%

\begin{frame}[fragile]{Resultatet}
\protect\hypertarget{resultatet}{}
Kör vi koden i R får vi följande resultat

\begin{verbatim}
## [1] 10
## [1] 9
## [1] 8
## [1] 7
## [1] 6
## [1] 5
## [1] 4
## [1] 3
## [1] 2
## [1] 1
\end{verbatim}

\begin{verbatim}
## [1] "Kör!"
\end{verbatim}
\end{frame}

%%%%%%%%% slide %%%%%%%%%

\begin{frame}{R och RStudio}
\protect\hypertarget{r-och-RStudio}{}
\begin{itemize}
\tightlist
\item
  R är både ett program och ett programeringsspråk
\item
  RStudio är en IDE för R
\item
  Båda är gratis och går att ladda ner och installera på er egna dator.
\item
  Se kurshemsidan för information.
\end{itemize}
\end{frame}

%%%%%%%%% slide %%%%%%%%%

\begin{frame}{Demo: RStudio}
\begin{itemize}
\tightlist
\item
  Hur använder man RStudio?
\end{itemize}
\end{frame}


%%%%%%%%% slide %%%%%%%%%

\begin{frame}{Datorsalar: SU}
\begin{itemize}
\tightlist
\item
  Datorlaborationerna är i SU-salarna i B-huset
\item
  Har Linux (Ubuntu) som operativsystem
\item
  Utanför de bokade passen: Om de är ledigt så kan ni använda SU-salar eller PC1-PC5 (E-huset) för självstudier.
\end{itemize}
\end{frame}


%%%%%%%%% slide %%%%%%%%%

\begin{frame}{Datorsalar: SU}
Hur startar jag Rstudio och börjar labbba i en SU-sal?????
\begin{itemize}
\tightlist
\item
 Logga in med Liu-ID och lösenord
\item
  Tryck: \texttt{Ctrl+Alt+T} för att öppna en terminal. Här kan du skriva olika kommandon, dessa aktiveras när du trycker enter.
\item
  Skriv: \texttt{module add courses/732G12} och tryck enter
\item
  Detta läser in kursmodulen, som innehåller de programvaror som behövs i kursen.
\item
  Skriv \texttt{rstudio} i terminalen och tryck enter
\end{itemize}
\end{frame}



%%%%%%%%% slide %%%%%%%%%

\begin{frame}[fragile]{Att hitta hjälp}
\protect\hypertarget{att-hitta-hjuxe4lp}{}
\begin{itemize}
\tightlist
\item
  Inbyggd hjälp i R
\item
  Sök i Google
\item
  Sök på \textcolor{red}{ENGELSKA}
\item
  Kolla på felmeddelandet
\end{itemize}

\begin{verbatim}
## Error in eval(expr, envir, enclos) : object 'x' not found
\end{verbatim}
\end{frame}

%%%%%%%%% slide %%%%%%%%%


\begin{frame}[fragile]{Variabler och vektorer}
\protect\hypertarget{variabler-och-vektorer}{}
\begin{itemize}
\tightlist
\item
  Variabler kan spara värden

  \begin{itemize}
  \tightlist
  \item
    Sätts med \texttt{<-} (eller \texttt{->})
  \end{itemize}
\item
  Vektorer är en samling av likadana element

  \begin{itemize}
  \tightlist
  \item
    Skapas med \texttt{c()}
  \item
    Välj element med \texttt{[ ]}
  \end{itemize}
\end{itemize}

Exempel:

\begin{Shaded}
\begin{Highlighting}[]
\NormalTok{a }\OtherTok{\textless{}{-}} \DecValTok{1}
\NormalTok{a}
\end{Highlighting}
\end{Shaded}

\begin{verbatim}
## [1] 1
\end{verbatim}

\begin{Shaded}
\begin{Highlighting}[]
\NormalTok{testVektor }\OtherTok{\textless{}{-}} \FunctionTok{c}\NormalTok{(}\DecValTok{2}\NormalTok{,}\DecValTok{3}\NormalTok{,}\DecValTok{5}\NormalTok{,}\DecValTok{7}\NormalTok{,}\DecValTok{11}\NormalTok{,}\DecValTok{13}\NormalTok{)}
\NormalTok{testVektor[}\FunctionTok{c}\NormalTok{(}\DecValTok{1}\NormalTok{,}\DecValTok{3}\NormalTok{)]}
\end{Highlighting}
\end{Shaded}

\begin{verbatim}
## [1] 2 5
\end{verbatim}
\end{frame}

%%%%%%%%% slide %%%%%%%%%


\begin{frame}[fragile]{Räkna med vektorer}
\protect\hypertarget{ruxe4kna-med-vektorer}{}
\begin{itemize}
\tightlist
\item
  Beräkningar sker elementvis
\end{itemize}

\begin{Shaded}
\begin{Highlighting}[]
\NormalTok{testVektor }\OtherTok{\textless{}{-}} \FunctionTok{c}\NormalTok{(}\DecValTok{2}\NormalTok{,}\DecValTok{3}\NormalTok{,}\DecValTok{5}\NormalTok{,}\DecValTok{7}\NormalTok{,}\DecValTok{11}\NormalTok{,}\DecValTok{13}\NormalTok{)}
\NormalTok{testVektor}\SpecialCharTok{+}\DecValTok{1}
\end{Highlighting}
\end{Shaded}

\begin{verbatim}
## [1]  3  4  6  8 12 14
\end{verbatim}

\begin{itemize}
\tightlist
\item
  Beräkningar mellan vektorer sker cykliskt
\end{itemize}

\begin{Shaded}
\begin{Highlighting}[]
\NormalTok{testVektor }\OtherTok{\textless{}{-}} \FunctionTok{c}\NormalTok{(}\DecValTok{2}\NormalTok{,}\DecValTok{3}\NormalTok{,}\DecValTok{5}\NormalTok{,}\DecValTok{7}\NormalTok{,}\DecValTok{11}\NormalTok{,}\DecValTok{13}\NormalTok{)}
\NormalTok{testVektor}\SpecialCharTok{+}\FunctionTok{c}\NormalTok{(}\DecValTok{1}\NormalTok{,}\DecValTok{2}\NormalTok{)}
\end{Highlighting}
\end{Shaded}

\begin{verbatim}
## [1]  3  5  6  9 12 15
\end{verbatim}
\end{frame}

%%%%%%%%% slide %%%%%%%%%


\begin{frame}[fragile]{Olika typer av värden}
\protect\hypertarget{olika-typer-av-vuxe4rden}{}
\begin{itemize}
\tightlist
\item
  Värden kan vara en av flera olika typer

  \begin{itemize}
  \tightlist
  \item
    t.ex. heltal, flyttal, textsträngar etc.
  \end{itemize}
\item
  Dessa typer kallas atomära klasser
\item
  Kan kolla vilken typ det är med \texttt{typeof( )}
\item
  Kan konvertera med \texttt{as.}
\end{itemize}

\begin{Shaded}
\begin{Highlighting}[]
\FunctionTok{as.character}\NormalTok{(}\DecValTok{4}\SpecialCharTok{:}\DecValTok{8}\NormalTok{)}
\end{Highlighting}
\end{Shaded}

\begin{verbatim}
## [1] "4" "5" "6" "7" "8"
\end{verbatim}

\pause

\begin{longtable}[]{@{}lllll@{}}
\toprule
Beskrivning & Synonymer & \texttt{typeof()} & Exempel i R & \\
\midrule
\endhead
Heltal (\(\mathbb{Z}\)) & \texttt{int} & \texttt{integer} &
\texttt{-1,\ 0,\ 1} & \\
Reella tal (\(\mathbb{R}\)) & \texttt{real}, \texttt{float} &
\texttt{double} & \texttt{1.03,\ -2.872} & \\
Komplexa tal (\(\mathbb{C}\)) & \texttt{cplx} & \texttt{complex} &
\texttt{1\ +\ 2i} & \\
Logiska värden & \texttt{boolean}, \texttt{bool} & \texttt{logical} &
\texttt{TRUE\ FALSE} & \\
Textsträngar & \texttt{string}, \texttt{char} & \texttt{character} &
\texttt{En\ textsträng} & \\
\bottomrule
\end{longtable}
\end{frame}


%%%%%%%%% slide %%%%%%%%%

\begin{frame}{Demo: Variabler}

\end{frame}

%%%%%%%%% slide %%%%%%%%%


\begin{frame}{Funktioner i R}
\protect\hypertarget{funktioner-i-r}{}
\begin{itemize}
\tightlist
\item
  En funktion utför något
\item
  Tar noll eller flera \textcolor{blue}{argument}
\item
  Funktioner samlas i R-paket
\item
  Många små funktioner, en funktion gör en sak.
\end{itemize}
\end{frame}

%%%%%%%%% slide %%%%%%%%%


\begin{frame}{Funktioner i R II}
\protect\hypertarget{funktioner-i-r-ii}{}
En funktion i R är uppbyggd av

\begin{itemize}
\tightlist
\item
  ett funktionsnamn, t.ex. \texttt{area}
\item
  en funktionsdefinition: \texttt{function( )}
\item
  0 eller flera argument, t.ex. \texttt{hojd} och \texttt{bredd}
\item
  ``måsvingar'' \texttt{ \{ \} }
\item
  kod, t.ex. \texttt{area <- hojd * bredd}
\item
  returnera värde, t.ex. \texttt{return(area)}
\end{itemize}
\end{frame}

%%%%%%%%% slide %%%%%%%%%


\begin{frame}[fragile]{Exempel på funktion i R}
\protect\hypertarget{exempel-puxe5-funktion-i-r}{}
\begin{Shaded}
\begin{Highlighting}[]
\NormalTok{area }\OtherTok{\textless{}{-}} \ControlFlowTok{function}\NormalTok{(hojd, bredd)\{}
\NormalTok{  area }\OtherTok{\textless{}{-}}\NormalTok{ hojd }\SpecialCharTok{*}\NormalTok{ bredd}
  \FunctionTok{return}\NormalTok{(area)}
\NormalTok{\}}
\end{Highlighting}
\end{Shaded}

\begin{Shaded}
\begin{Highlighting}[]
\FunctionTok{area}\NormalTok{(}\AttributeTok{hojd =} \DecValTok{2}\NormalTok{, }\AttributeTok{bredd =} \DecValTok{3}\NormalTok{)}
\end{Highlighting}
\end{Shaded}

\begin{verbatim}
## [1] 6
\end{verbatim}

\begin{Shaded}
\begin{Highlighting}[]
\FunctionTok{area}\NormalTok{(}\AttributeTok{hojd =} \DecValTok{5}\NormalTok{, }\AttributeTok{bredd =} \DecValTok{11}\NormalTok{)}
\end{Highlighting}
\end{Shaded}

\begin{verbatim}
## [1] 55
\end{verbatim}
\end{frame}


%%%%%%%%% slide %%%%%%%%%

\begin{frame}{Demo: Funktioner}

\end{frame}



%%%%%%%%% slide %%%%%%%%%


\begin{frame}[fragile]{Lokal miljö}
\protect\hypertarget{lokal-miljuxf6}{}
``Det som sker i en funktion stannar i funktionen''

\begin{Shaded}
\begin{Highlighting}[]
\NormalTok{f }\OtherTok{\textless{}{-}} \ControlFlowTok{function}\NormalTok{(x, y)\{}
\NormalTok{  z }\OtherTok{\textless{}{-}} \DecValTok{5}
\NormalTok{  svar }\OtherTok{\textless{}{-}}\NormalTok{ z}\SpecialCharTok{*}\NormalTok{x }\SpecialCharTok{+}\NormalTok{ y }
  \FunctionTok{return}\NormalTok{(svar)}
\NormalTok{\}}
\end{Highlighting}
\end{Shaded}

\texttt{z} och \texttt{svar} kan \textcolor{red}{inte} användas utanför
funktionen.
\end{frame}

%%%%%%%%% slide %%%%%%%%%

\begin{frame}[fragile]{Lokal miljö II}
\protect\hypertarget{lokal-miljuxf6-ii}{}
\begin{Shaded}
\begin{Highlighting}[]
\FunctionTok{ls}\NormalTok{()}
\end{Highlighting}
\end{Shaded}

\begin{verbatim}
## [1] "a"      "area"    "f"       "i"       "start"     
## [6] "testVektor"
\end{verbatim}

\begin{Shaded}
\begin{Highlighting}[]
\FunctionTok{f}\NormalTok{(}\DecValTok{1}\NormalTok{,}\DecValTok{2}\NormalTok{)}
\end{Highlighting}
\end{Shaded}

\begin{verbatim}
## [1] 7
\end{verbatim}

\begin{Shaded}
\begin{Highlighting}[]
\FunctionTok{ls}\NormalTok{()}
\end{Highlighting}
\end{Shaded}

\begin{verbatim}
## [1] "a"      "area"    "f"       "i"       "start"     
## [6] "testVektor"
\end{verbatim}
\end{frame}

%%%%%%%%% slide %%%%%%%%%


\begin{frame}{Att tänka på}
\protect\hypertarget{att-tuxe4nka-puxe5}{}
\begin{itemize}
\tightlist
\item
  Funktionen måste läsas in innan den fungerar.
\item
  \texttt{return()} avslutar funktionen
\item
  \textcolor{red}{Skriv funktionen i flera delar}

  \begin{itemize}
  \tightlist
  \item
    Skriv kod som gör det du vill
  \item
    Lyft in koden i funktionen
  \item
    Pröva funktionen
  \end{itemize}
\end{itemize}
\end{frame}

%%%%%%%%% slide %%%%%%%%%

\begin{frame}{Demo: Funktioner}

\end{frame}

%%%%%%%%% slide %%%%%%%%%

\begin{frame}{markmyassignment}
\protect\hypertarget{markmyassignment-i-r}{}
\begin{itemize}
\tightlist
\item
  Ett R-paket
\item
  Autorättar uppgifter
\item
  Används i kursen för att rätta inlämnade funktioner
\item
  Ni använder markmyassignment innan ni lämnar in 
\end{itemize}
\end{frame}

%%%%%%%%% slide %%%%%%%%%

\begin{frame}{Demo: markmyassignment}

\end{frame}



%%%%%%%%% slide %%%%%%%%%

\begin{frame}{Logik}
\protect\hypertarget{logik}{}
\begin{itemize}
\tightlist
\item
  Logik är vanligt i programmering

  \begin{itemize}
  \tightlist
  \item
    Används i \texttt{if}-satser $\rightarrow$ kursvecka 3!
  \end{itemize}
\item
  I R finns de logiska värdena \texttt{TRUE}, \texttt{FALSE}, och
  \texttt{NA}
  \begin{itemize}
  \tightlist
  \item
    \texttt{NA} = not available = saknade värden
  \end{itemize}
\item
  Skapas på två olika sätt

  \begin{itemize}
  \tightlist
  \item
    Som vanliga vektorer
  \item
    Genom relationsoperatorer
  \end{itemize}
\item
  Kan användas för att välja element i vektorer
\end{itemize}
\end{frame}

%%%%%%%%% slide %%%%%%%%%


\begin{frame}[fragile]{Logik i R}
\protect\hypertarget{logik-i-r}{}
Kan skapa en vektor med värdena \texttt{TRUE} och \texttt{FALSE}

\begin{Shaded}
\begin{Highlighting}[]
\NormalTok{testVektor }\OtherTok{\textless{}{-}} \FunctionTok{c}\NormalTok{(}\DecValTok{2}\NormalTok{,}\DecValTok{3}\NormalTok{,}\DecValTok{5}\NormalTok{,}\DecValTok{7}\NormalTok{,}\DecValTok{11}\NormalTok{,}\DecValTok{13}\NormalTok{)}
\NormalTok{boolVektor }\OtherTok{\textless{}{-}} \FunctionTok{c}\NormalTok{(}\ConstantTok{TRUE}\NormalTok{, }\ConstantTok{FALSE}\NormalTok{, }\ConstantTok{FALSE}\NormalTok{, }\ConstantTok{TRUE}\NormalTok{, }\ConstantTok{FALSE}\NormalTok{, }\ConstantTok{TRUE}\NormalTok{)}
\end{Highlighting}
\end{Shaded}

\begin{Shaded}
\begin{Highlighting}[]
\NormalTok{testVektor[boolVektor]}
\end{Highlighting}
\end{Shaded}

\pause

\begin{verbatim}
## [1]  2  7 13
\end{verbatim}

\pause

Kan också skapa vektor genom en relation

\begin{Shaded}
\begin{Highlighting}[]
\NormalTok{testVektor }\SpecialCharTok{\textgreater{}} \DecValTok{5}
\end{Highlighting}
\end{Shaded}

\begin{verbatim}
## [1] FALSE FALSE FALSE  TRUE  TRUE  TRUE
\end{verbatim}
\end{frame}

%%%%%%%%% slide %%%%%%%%%

\begin{frame}{Relationsoperatorer}
\protect\hypertarget{relationsoperatorer}{}
\begin{itemize}
\tightlist
\item
  Relationer används för att jämförelser
\item
  Skapar logiska vektorer
\end{itemize}

\begin{longtable}[]{@{}ll@{}}
\toprule
Beskrivning & Operatorer i R \\
\midrule
\endhead
Lika med & \texttt{==} \\
Inte lika med & \texttt{!=} \\
Större än & \texttt{>} \\
Mindre än & \texttt{<} \\
Större än eller lika med & \texttt{>=} \\
Mindre än eller lika med & \texttt{<=} \\
Finns i & \texttt{\%in\%} \\
\bottomrule
\end{longtable}
\end{frame}

%%%%%%%%% slide %%%%%%%%%

\begin{frame}[fragile]{Logiska operatorer}
\protect\hypertarget{logiska-operatorer}{}
\begin{itemize}
\tightlist
\item
  Boolesk algebra
\item
  Operatorer:
\end{itemize}

\begin{longtable}[]{@{}lll@{}}
\toprule
Operator & Symbol & Operator i R \\
\midrule
\endhead
och & \(\wedge{}\) & \texttt{\&} \\
eller & \(\vee{}\) & \texttt{|} \\
inte & \(\neg{}\) & \texttt{!} \\
\bottomrule
\end{longtable}

\pause

\begin{longtable}[]{@{}llllll@{}}
\toprule
Symbol & \(A\) & \(B\) & \(\neg A\) & \(A \wedge B\) & \(A \vee B\) \\
\midrule
\endhead
i R & \texttt{A} & \texttt{B} & \texttt{!A} & \texttt{A\ \&\ B} &
\texttt{A} \(|\) \texttt{B} \\
& \texttt{TRUE} & \texttt{TRUE} & \texttt{FALSE} & \texttt{TRUE} &
\texttt{TRUE} \\
& \texttt{TRUE} & \texttt{FALSE} & \texttt{FALSE} & \texttt{FALSE} &
\texttt{TRUE} \\
& \texttt{FALSE} & \texttt{TRUE} & \texttt{TRUE} & \texttt{FALSE} &
\texttt{TRUE} \\
& \texttt{FALSE} & \texttt{FALSE} & \texttt{TRUE} & \texttt{FALSE} &
\texttt{FALSE} \\
\bottomrule
\end{longtable}
\end{frame}

%%%%%%%%% slide %%%%%%%%%

\begin{frame}{Demo: Logik}

\end{frame}


%%%%%%%%% slide %%%%%%%%%

\begin{frame}[fragile]{Logik exempel}
\protect\hypertarget{logik-exempel}{}
\begin{Shaded}
\begin{Highlighting}[]
\NormalTok{testVektor }\OtherTok{\textless{}{-}} \FunctionTok{c}\NormalTok{(}\DecValTok{2}\NormalTok{,}\DecValTok{3}\NormalTok{,}\DecValTok{5}\NormalTok{,}\DecValTok{7}\NormalTok{,}\DecValTok{11}\NormalTok{,}\DecValTok{13}\NormalTok{,}\DecValTok{17}\NormalTok{,}\DecValTok{19}\NormalTok{,}\DecValTok{23}\NormalTok{,}\DecValTok{29}\NormalTok{,}\DecValTok{31}\NormalTok{)}
\NormalTok{boolVektor }\OtherTok{\textless{}{-}}\NormalTok{ testVektor }\SpecialCharTok{\textless{}} \DecValTok{6} \SpecialCharTok{|} \SpecialCharTok{!}\NormalTok{(testVektor }\SpecialCharTok{\textless{}} \DecValTok{20}\NormalTok{)}
\end{Highlighting}
\end{Shaded}

Vad blir följande uttryck?

\begin{Shaded}
\begin{Highlighting}[]
\NormalTok{testVektor[boolVektor]}
\end{Highlighting}
\end{Shaded}

\pause

\begin{verbatim}
## [1]  2  3  5 23 29 31
\end{verbatim}
\end{frame}

\end{document}
